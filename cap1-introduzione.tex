\addcontentsline{toc}{chapter}{Introduzione} % Capitolo non numerato

\textbf{\LARGE Introduzione}\newline\newline


%% metti le foto della proteina vault!

%- vault (wiki, organello)

Protein Vaults
have an unusual bell-shaped structure. The vault is made out of two bell shaped subunits with 39 identical chains each. Their exact function is not known, but it is believed to have something to do with protein synthesis and possibly protein transport.


\say{\textit{Il Buddha, Il Divino, dimora nel circuito di un calcolatore o negli ingranaggi del cambio di una moto con lo stesso agio che in cima a una montagna o nei petali di un fiore}}\footnote{\fullcite{pirsig1974zen}} \\

\say{\textit{Seduto in riva all'oceano [..] ebbi la consapevolezza che tutto intorno a me prendeva parte a una gigantesca danza; [..] le mie esperienze [in fisica delle alte energie] presero vita: «vidi» scendere dallo spazio esterno cascate di energia, nelle quali si creavano e distruggevano particelle con ritmi pulsanti; «vidi» gli atomi degli elementi e quelli del mio corpo partecipare a questa danza cosmica di energia}}\footnote{\fullcite{capra1975tao}}


“The view that all aspects of reality can be reduced to matter and its various particles is, to my mind, as much a metaphysical position as the view that an organizing intelligence created and controls reality.”

 All things and events, whether material, mental, or even abstract concepts like time, are devoid of objective independent existence. To possess such independent, intrinsic existence would imply that things and events are somehow complete unto themselves and are therefore entirely contained.”
 
 
 
 Parla di come fosse imporante andare nel dettaglio della questione biologica. Capire sia i meccanismi sia i possibili usi di tale "invenzione" (es. combattere prioni).

\clearpage

