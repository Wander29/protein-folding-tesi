\chapter{Introduzione}

\textit{Cos'è la vita? Da dove viene?} - Fino al XVIII secolo per rispondere a tale quesito si faceva riferimento alla fede nel vitalismo: l'esistenza di una forza vitale non subordinata alle leggi della chimica e della fisica.
Importanti svolte furono gli esperimenti, prima di Redi poi di Spallanzani, per dimostrare l'infondatezza della teoria della \textit{generazione spontanea}, secondo la quale la vita poteva generarsi da materia non vivente. Un'importante passo in avanti, in concomitanza con l'affermarsi della \textit{teoria cellulare}, fu il lavoro di Pasteur che stabilì un collegamento fra processi vitali e reazioni chimiche: per la conversione di zucchero in alcool (fermentazione) era necessaria la presenza di microorganismi.
\par Successivamente vi sono i lavori di Berthelot e Buchner (premio Nobel per la Chimica 1907), il quale dimostrò che era possibile ottenere la fermentazione in assenza di microorganismi, usando solamente sostanze estratte da essi.
Queste sostanze furono chiamate \textit{enzimi} (dal ted. Enzym, letteralmente «dentro il lievito»\supercite{enzimaTreccani}). Non si conosceva la loro natura chimica, si scoprì successivamente che tutti gli enzimi sono \textit{proteine} (dal greco «primario», «che occupa la prima posizione» \supercite{proteinaTreccani}).
Queste proteine agivano da catalizzatori: acceleravano le reazioni chimiche all'interno delle cellule senza cambiare la loro natura, quindi senza consumarsi e senza entrare nei prodotti finali della reazione.

\par La scoperta degli enzimi portò ad un cambio di paradigma nel pensiero scientifico riguardo le origini della vita: veniva ora considerata come la conseguenza di numerosi processi chimici resi possibili dalle proteine \supercite{kessel_ben-tal_2018}.
I fondamenti del pensiero biologico si spostarono dal vitalismo al meccanicismo secondo il quale tutti i fenomeni naturali, vita compresa, sono governati dalle stesse leggi, sia per sostanze organiche che inorganiche.

\par L'inconorazione delle proteine a \textit{macromolecole più importanti della vita} si può legare ad un'altra svolta nel pensiero scientifico avvenuta nella seconda metà del XX secolo: la rivoluzione genetica. Le proteine sono i prodotti finali dei geni e sono anche coinvolte nell'espressione dell'informazione genetica. È sullo sfondo di questa rivoluzione che l'informatica si è inserita all'interno del mondo della biologia.
 

\clearpage

