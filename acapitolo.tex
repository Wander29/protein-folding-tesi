\chapter{Acapitolo - riposiziona il mio contenuto}

\addcontentsline{toc}{chapter}{Introduzione}

\textbf{{\LARGE Introduzione\newline\newline L'informatica: un potente strumento}}
\vspace{1cm}

Illustrare il mio obiettivo e la suddivisione del lavoro, dopo aver esposto la mia posizione sui rischi e le prospettive positive aperte dall'informatica.\\\\


-- continua l;'introduzione dall'aritcolo: Soft computing methods for the prediction of protein tertiary
structures: A survey --



\chapter{AlphaFold}



IPA = This makes sense as IPA’s intended function is to refine the structure coming out of the evoformer. 

DeepMSA 
DeepMSA is a composite approach to generate high quality multiple sequence alignment with large alignment depth and diverse sequence sources by merging sequences from whole-genome sequence databases (Uniclust30 and UniRef90) and from metagenome database (Metaclust). Large-scale benchmark data show that DeepMSA profiles consistently improves contact prediction, secondary structure prediction, and threading over default HHblits or PSI-BLAST profiles. 



\par La sfida principale che questi metodi hanno affrontato, e che AlphaFold ha "vinto", era raggiungere almeno una RMSD di 3\angstrom. Le sfide ancora da superare riguardano l'accuratezza su grandi proteine, su proteine con un contenuto significativo di strutture $\beta$ e la modellazione di proteine multi-dominio e di membrana.


-pearce 2021 confronto con AF1, ottimo in generale

--- wei2019protein
AlphaFold1 nel CASP 13 ha raggiunto i massimi punteggi in 25 dei 43 test.

assembles the most probable
fragments based on the co-evolution
analysis of a multiple sequence alignment.

it makes
use of its formidable computing power
to manage truly deep neural networks
that identify co-evolutionary patterns in
protein sequences4 in terms of contact
distributions and angular restraints.
Furthermore, AlphaFold uses deep
learning to generate a protein-specific
statistical potential using a ‘learned
reference state’1, instead of a physical-based
reference state 5.

AF1

AlphaFold  [12] is a protein structure  prediction  method  developed  by  DeepMind, diction  accuracy  of RaptorX  using  the distance  matrix  was quite better than  that using which  had  one  of  the  best  performances  in  CASP13.  AlphaFold  is  the  other  approach contact matrix on a set of CASP targets. Along with AlphaFold, this approach  helped that championed the idea of using distogram for protein structure prediction. Essentially, the distogram prediction component of AlphaFold uses a convolutional neural network (CNN) that is trained on PDB structures to predict the Cβ–Cβ distances between any pair of residues. Using the amino acid representation of the query sequence and features generated from MSA, the CNN network predicts a discrete probability distribution for every pair. This distribution is found to be similar to the true distances. Then, the predicted backbone torsion angles and pair-wise distance between residues are combined to form \supercite{pakhrin2021deep}

\subsection{PAE}
% - alphafold db varaldi, anche pLDDT





\chapter{Protein Engineering e campi applicativi}

Le proteine sono sì macchine ma non funzionano minimamente come le macchine create dall'uomo. non abbiamo elettromagneti, nessun sistema vivente lo ha, non usiamo batterie. 

C'è tanta tecnologia nel mondo microscopico che potrebbe venire usata per la tecnologia dell'uomo.

Target farmacologici: sostanze chimiche che possono legarsi alle proteine "cattive" e impedire così che possano legarsi ai loro target originari.
Cambiamento di forma.

I motori elettrici utilizzano metà dell'elettricità mondiale, motori a benzina consumano 1/3 dell'energia mondiale disponibile.
Abbiamo più macchine microscopiche nelle nostre dita che macchine grandi in tutto il mondo

È uno dei rari casi nella storia in cui la tecnologia da sviluppare si conosce già in gran dettaglio. Si tratta solamente di capire come trasportarla nel mondo macroscopico.

He didn't say how we might macrosize the principles of natural machines. He just said their intricacy and efficiency might guide our future thinking when we finally get the rules (such as protein folding) behind us. A machine that repairs itself and reproduces itself is far different than what we are doing in industry today. But nature's invention is not unblemished either. It comes with pain and death. So we have to pick up where mom left off and leave the world a better place than we found it.
