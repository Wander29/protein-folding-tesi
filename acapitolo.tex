\chapter{Acapitolo - riposiziona il mio contenuto}

\addcontentsline{toc}{chapter}{Introduzione}

\textbf{{\LARGE Introduzione\newline\newline L'informatica: un potente strumento}}
\vspace{1cm}

Illustrare il mio obiettivo e la suddivisione del lavoro, dopo aver esposto la mia posizione sui rischi e le prospettive positive aperte dall'informatica.\\\\


-- continua l;'introduzione dall'aritcolo: Soft computing methods for the prediction of protein tertiary
structures: A survey --



- Soft computing
I paradigmi del soft computing per la predizione della struttura delle proteine sono:
\begin{itemize}
\item ANNs: artificial neural networks
\item EC: evolutionary computation
\item SVMs: support vector machines
\end{itemize}
Inoltre i metodi per la predizione della struttura delle proteine possono essere ulteriormente classificati in base ad un'approssimazione biologica \cite{marquez2015soft}:
\begin{itemize}
	\item homology-based methods
	\item threading methods
	\item \textit{ab initio } methods
\end{itemize}

La prima teoria sul ripiegamento delle proteine risale agli anni trenta , Wu \cite{wu1931studies}, poi Pauling e altra gentaccia

\section{Background filosofico}
Buttaci un po' di filosofia della scienza e di quali cambiamenti potrebbe apportare alla struttura delle rivoluzioni scientifiche. Cita Fleck in qualche modo!
Trova casi di cambi di paradigma e a "riscoperte" tornate alla ribalta grazie all'informatica. Magari l'informatica è un modo, analizzando tanti dati, di contrastare i bias nella scienza?


\section{Proteins and protein levels}

\section{Amino acids, nucleotides and codons}

\section{Protein structure characteristics}
such as Domains, Motifs, Residues and Turns

\section{Distograms}

\section{Phenotypes and genotypes}

\section{Protein Folding}
È considerato uno dei problemi più imepgnativi degli ultimi 50 anni in biochimica.
Utilizzando ancora le capacità divulgative di \cite{TalksDill2013Oct} si può immaginare una proteina come una collana composta da perle, dove ogni perla è un amminoacido e le perle possono avere 20 diversi colori.

Il punto del \textit{protein folding problem }è capire come la stringa di amminoacidi codifichi la forma della proteina

\chapter{Predizione della struttura delle proteine}
L'analisi della struttura delle proteine è intrinsecamente complessa: "nessun'altra classe di molecole (piccole o grandi) esibisce una varietà di forme, dimensioni, struttura e movimenti comparabili alle proteine"  \parencite{baxevanis2020bioinformatics}.

\section{Determinazione sperimentale della struttura delle proteine}

Ci sono 3 tecniche sperimentali che possono essere usate per generare informazioni a risoluzione atomica sulla struttura delle proteine.

\section{CASP}
CASP (\textit{Critical Assessment of Structure Predictions}) è una sfida biennale dove gruppi di ricerca si sfidano cercando di realizzare predizioni di strutture di proteine la cui sequenza amminoacidica è nota ma non la struttura determinata sperimentalmente, che verrà utilizzata per stabilire l'accuratezza dei metodi in gara. \\

Nel 2020 gli organizzatori del CASP14 hanno rinosciuto AlphaFold come soluzione del \textit{protein–structure–prediction problem}. \\

\subsection{Valutazione dell'accuratezza delle predizioni}

Le tecniche di valutazione della predizione della struttura delle proteine richiede criteri ogettivi sulla similarità tra un modello computazionale e la struttura di riferimento determinata sperimentalmente.\\

La misura di valutazione dell'accuratezza oggi è utilizzata è il lDDT (\textit{local Distance
	Difference Test}) \cite{mariani2013lddt}. \\

Le misure precedenti si basavano su una superposizione globale di atomi di carbonio ed erano fortemente influenzate dai movimenti di dominio e non assicurano l'accuratezza di detagli atomici locali nel modello.

\section{Prima di AlphaFold}

\subsection{Machine Learning e biologia}

\chapter{AlphaFold}
AlphaFold è un sistema di \textit{Artificial Intelligence }(AI) sviluppato da DeepMind che realizza predizioni allo stato dell'arte sulla struttura delle proteine basandosi sulle loro sequenze amminoacidiche.

\section{DeepMind}

DeepMind è un'azienda inglese di Intelligenza Artificiale sussidiaria di Alphabet Inc.; in altre parole DeepMind è una società controllata: Alphabet Inc. detiene la maggioranza dei voti nell'assemblea ordinaria o un'influenza dominante sull'amministrazione.

\par DeepMind è stata fondata nel 2010 da Demis Hassabis, Shane Legg e Mustafa Suleyman. La società ha sede a Londra con centri di ricerca in Canada, Francia e Stati Uniti \cite{deepMindWiki}.

Può risultare interessante osservare la correlazione fra i primi lavori di DeepMind e la vita di Demis Hassabis, una vita ricca di sfaccettature: bambino prodigio nel gioco degli scacchi, programmatore di videogiochi (dai 17 anni) passando per una laurea in \textit{Computer Science}, alla fondazione del proprio studio videoludico (Elixir Studios) per poi ritornare nel mondo accademico per ottenere il suo PhD in neuroscienze cognitive nel 2009, campo nel quale ha coautorato numerosi articoli influenti su memoria e amnesia (es. rappresentazione della memoria episodica tramite \textit{scene construction} \cite{Hassabis2007Jul}) \cite{hassabisWiki}. Per arrivare infine a fondare DeepMind e la nuovissima società Isomorphic Labs, sempre sussidiaria di Alphabet Inc.

\par DeepMind iniziò infatti a focalizzarsi sull'insegnare ad un sistema di AI come giocare a vecchi videogiochi anni '70,'80 (es. Pong, Breakout, Space Invaders), per poi passare a Go

DeepMind è stata acquistata da Google nel 2014 per 500 milioni di dollari \cite{Guardian2014}.

\subsection{Etica}
Dopo l'acquisizione di Google l'azienda ha stabilito un'\textit{AI ethics board}.\\
DeepMind è uno dei membri fondatori di \textit{Partnership on AI} insieme ad Amazon, Google, Facebook, IBM e Microsoft, un'organizzazione dedicata all'interfaccia società-AI \cite{partnershiponai}.

- inserire roba su partnershipai (articoli e pilastri)

DeepMind ha anche aperto una nuova unità denominata DeepMind Ethics and Society e si è concentrata sulle questioni etiche e sociali sollevate dall'intelligenza artificiale avendo come consulente il famoso filosofo Nick Bostrom. Nell'ottobre 2017, DeepMind ha lanciato un nuovo gruppo di ricerca per studiare l'etica dell'IA.[5]

\subsection{Alphabet}
Alphabet è un'azienda statunitense fondata nel 2015 dagli stessi fondatori di Google (Larry Page e Sergey Brin) come \textit{holding} a cui fa capo Google LLC e altre società sussidiarie: oltre a DeepMind vi sono Calico, CapitalG, Waymo, Wing, Intrinsic, Nest Labs, Sidewalk Labs, Isomorphic Labs, ...\\ 
Da dicembre 2019 il CEO di Alphabet è Sundar Pichai \cite{cnbc}.
La fondazione di Alphabet a partire da Google è stata una scelta finalizzata a rendere più trasparenti le attività inerenti a Google e concedere una maggiore autonomia alle società del gruppo che operano in settori diversi da quello dei servizi internet.

\subsection{Isomorphic Labs}



\section{AlphaFold 1}
\section{AlphaFold 2}
\section{AlphaFold-Multimer e futuro}
\section{AlphaFold DB}



\section{Rischi per i metodi omologo}
Rischi anemia falciforme (1 amminoacido diverso).
Obiettivo tesi: come svicolare problemi dovuti a somiglianze sequenze ma funzione dverse. Spaventano! 
\section{Uso}
\section{Visualizzazione 3D del risultato}

\chapter{Sperimentazione di AlphaFold}
\section{proteina BFG-54g????}
\subsection{Confronto con altri metodi}

\chapter{Protein Engineering e campi applicativi}
\section{Enzymes engineering}
\section{Covid e Omicron}

\addcontentsline{toc}{chapter}{Conclusioni}

\textbf{{\LARGE Conclusioni}}
\vspace{1cm}

considerazioni sulle porte aperte dalla bioinformatica
Soddisfazione

Il problema del protein folding è risolto? (No).

