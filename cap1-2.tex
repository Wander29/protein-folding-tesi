\chapter{Proteine}
Le proteine sono una classe di macromolecole con funzioni biologiche vitali permettendo il funzionamento di ogni sistema vivente. Riusciamo a pensare, parlare, a digerire il cibo, a muoverci grazie alle proteine.\\

Per questa ragione sono il target di grandi attività di ricerca e di applicazione biotecnologiche: dal combattere malattie infettive \cite{batool2019structure} al contrastare l'inquinamento ambientale \cite{knott2020characterization}.


Nel nostro corpo abbiamo un numero grandissimo di proteine: $10^{27}$. Per usare una metafora di Ken Dill \cite{TalksDill2013Oct} potremmo dire che se si potesse ingrandire una molecola di proteina alla grandezza di un penny (diametro di 19mm \cite{CentWikipedia}) il numero di protiene che una persona ha nel corpo è lo stesso del numero di penny che riempirebbero l'Oceano Pacifico.\\

Nonostante siano piccole, comparandole alle altre molecole del nostro corpo sono fra le più complesse e grandi.\\

--Foto proteina--\\





\section{The central dogma of biology}

\section{Proteins and protein levels}

\section{Amino acids, nucleotides and codons}

\section{Protein structure characteristics}
such as Domains, Motifs, Residues and Turns

\section{Distograms}

\section{Phenotypes and genotypes}

\section{Protein Folding}
È considerato uno dei problemi più imepgnativi degli ultimi 50 anni in biochimica.
Utilizzando ancora le capacità divulgative di \cite{TalksDill2013Oct} si può immaginare una proteina come una collana composta da perle, dove ogni perla è un amminoacido e le perle possono avere 20 diversi colori.

Il punto del \textit{protein folding problem }è capire come la stringa di amminoacidi codifichi la forma della proteina

\clearpage