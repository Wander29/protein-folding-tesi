\chapter{Predizione della struttura di proteine (PSP)}

- utilità: farmaci e funzione dalla sequenza genica \\

I biochimici conoscono oggi la sequenza amminoacidica per più di 225 milioni di proteine\supercite{proteienDBentries} (UniProt), con circa 4.5-5 milioni aggiunte ogni mese mentre la struttura tridimensionale è conosciuta per quasi 200.000 proteine\supercite{proteienDBentries} (PDB)\footnote{al 25 Gennaio 2022 sono presenti 197.514 strutture di macromolecole quindi non solo proteine ma anche acidi nucleici e altre strutture complesse\supercite{pdbStats}} con più di 10.000 strutture aggiunte ogni anno. Anche quando gli scienziati hanno una proteina correttamente ripiegata fra le mani non è così semplice determinarne la sua esatta conformazione tridimensionale, considerando che si parla di strutture di migliaia di atomi.

\par 

\section{Determinazione sperimentale}

- come le proteine sono studiate [alberts, 4.4 p.158]
- metodi sperimentali 
alberts p.168
pal 6 p.126 ma è molto tecnico
wiki-protein folding, 
baxevanis 12 p.363

• storia dei metodi sperimentali (Pal) \\
• cristallografia a raggi-x \\
• NMR, risonanza magnetica nucleare \\
• Cryo-EM (electron microscopy) \\
\subsection{Predizione guidata sperimentalmente}
kessel 3.5
pal 6.2.5 p.138

\section{Strumenti informatici}

\subsection{Rappresentazione informazioni}
pal 6.4.1 p.145
Gu 10-13 p.468
baxevanis 12 p.367

\subsection{Database}
baxevanis 12 p.373, 1

\subsection{Visualizzazione proteine}
pal 6.4 p.146, ottima
kessel p.174 (2.3 GRAPHIC REPRESENTATIONS OF PROTEINS)
\subsection{CASP}

\section{Workflow PSP, input, output, metriche}
- paper Soft computing
\subsection{Proprietà dei dati di input}
\subsection{Output: modelli, ... [..]}
\subsection{Metriche di valutazione}
- Gu 16 p.655 GENERAL APPROACH TO STRUCTURE COMPARISON AND ALIGNMENT
baxevanis 12 p.386

\section{Paradigmi per approssimazione biologica}
kessel 1.3.4 QM è fondamentale ma 
kessel 3.4 [computational methods for structure prediction]
pal 15.3.2 Computational structure prediction. Molto breve
pal 18.1 [molecular dynamics]
psp-wiki [abbastanza discorsivo]
prot-eng-wiki[molto breve]
Gu 28-33 [serie di articoli messi insieme, sembra stra interessante]
baxevanis 12 p.385

\subsection{Ab initio}
\subsection{Homolgy}
\subsection{Threading}
\subsection{Metodi integrativi [..]}
\subsection{applicazioni specifiche [..]}
\subsubsection{Loop modeling}

- predictive methods
baxevanis 7

- protein sequence alignment (poi MSA)
[Burkowski 6 p.167]
baxevanis 8 p.227

\section{Paradigmi nel soft computing}
\subsection{ANN, EC, SVM, .. [..]}
\subsection{altri approcci [..]}

\section{Storia della comprensione delle proteine}
- alberts p.160
- psp-wiki
\subsection{primi approcci}
\subsection{anni '90, database, omologia, progetto genoma}
\subsection{CASP e AlphaFold}

\clearpage