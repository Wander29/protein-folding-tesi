\chapter{Predizione della struttura delle proteine}
L'analisi della struttura delle proteine è intrinsecamente complessa: "nessun'altra classe di molecole (piccole o grandi) esibisce una varietà di forme, dimensioni, struttura e movimenti comparabili alle proteine"  \parencite{baxevanis2020bioinformatics}.

\section{Determinazione sperimentale della struttura delle proteine}

Ci sono 3 tecniche sperimentali che possono essere usate per generare informazioni a risoluzione atomica sulla struttura delle proteine.

\section{CASP}
CASP (\textit{Critical Assessment of Structure Predictions}) è una sfida biennale dove gruppi di ricerca si sfidano cercando di realizzare predizioni di strutture di proteine la cui sequenza amminoacidica è nota ma non la struttura determinata sperimentalmente, che verrà utilizzata per stabilire l'accuratezza dei metodi in gara. \\

Nel 2020 gli organizzatori del CASP14 hanno rinosciuto AlphaFold come soluzione del \textit{protein–structure–prediction problem}. \\

\subsection{Valutazione dell'accuratezza delle predizioni}

Le tecniche di valutazione della predizione della struttura delle proteine richiede criteri ogettivi sulla similarità tra un modello computazionale e la struttura di riferimento determinata sperimentalmente.\\

La misura di valutazione dell'accuratezza oggi è utilizzata è il lDDT (\textit{local Distance
Difference Test}) \cite{mariani2013lddt}. \\

Le misure precedenti si basavano su una superposizione globale di atomi di carbonio ed erano fortemente influenzate dai movimenti di dominio e non assicurano l'accuratezza di detagli atomici locali nel modello.

\section{Prima di AlphaFold}

\subsection{Machine Learning e biologia}

\clearpage