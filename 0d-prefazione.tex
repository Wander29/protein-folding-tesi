\addcontentsline{toc}{chapter}{Prefazione} % Capitolo non numerato

\textbf{\LARGE Prefazione}\newline\newline

Questa tesi è il frutto di un lavoro di ricerca e studio. È stato per me\footnote{La \textit{prima persona }è utilizzata solo nella prefazione per spiegare alcune scelte personali che perderebbero di "calore" se spiegate impersonalmente. Nel resto dell'elaborato verrà sempre utilizzata una forma impersonale.} un passo importante, nel quale ho messo alla prova le mie capacità di affrontare un argomento interdisciplinare e di interfacciarmi con il mondo della ricerca scientifica, leggendo e ricercando libri e articoli. Allo stesso tempo ho potuto dare spazio alla mia creatività, sia nello studio che durante la stesura di questo elaborato. \\

\par Non è stato immediato affrontare un tema con radici lontane dall'informatica come quello del \textit{protein folding}. Il mio obiettivo è stato quello di acquisire un'ampia comprensione del problema e non solo degli aspetti informatici legati alla predizione della struttura di proteine. Ciò che mi affascina è quello che si può trovare al confine fra i mondi della biologia e dell'informatica, al confine fra i due linguaggi e fra i modi di ragionare. Non avevo precedenti esposizioni alla biologia prima di seguire il corso \textit{elementi di biologia e neuroscienze} del prof. Mario Pirchio, correlatore del presente elaborato. \\

\par Oltre alle mie inclinazioni personali, trovo che considerare un problema biologico in termini esclusivamente matematici o informatici ne riduca l'importanza. Ho provato a scendere nei dettagli su alcune questioni biologiche per portare alla luce (per quanto mi sia stato possibile e avendo cura di non andare fuori tema) alcuni \textit{perché} dell'interesse nel ripiegamento delle proteine, come l'esistenza dei prioni e delle malattie neurodegenerative associate a problemi di ripiegamento delle proteine. Penso che comprendere la realtà biologica dell'argomento trattato e i \textit{perché} possa instradare la ricerca interdisciplinare su vie più efficaci e collaborative.


\clearpage

