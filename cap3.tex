\chapter{AlphaFold}
AlphaFold è un sistema di \textit{Artificial Intelligence }(AI) sviluppato da DeepMind che realizza predizioni allo stato dell'arte sulla struttura delle proteine basandosi sulle loro sequenze amminoacidiche.

\section{DeepMind}

DeepMind è un'azienda inglese di Intelligenza Artificiale sussidiaria di Alphabet Inc.; in altre parole DeepMind è una società controllata \cite{societaControllataWiki}: Alphabet Inc. detiene la maggioranza dei voti nell'assemblea ordinaria o un'influenza dominante sull'amministrazione.

\par DeepMind è stata fondata nel 2010 da Demis Hassabis, Shane Legg e Mustafa Suleyman. La società ha sede a Londra con centri di ricerca in Canada, Francia e Stati Uniti \cite{deepMindWiki}.

Può risultare interessante osservare la correlazione fra i primi lavori di DeepMind e la vita di Demis Hassabis, una vita ricca di sfaccettature: da bambino prodigio nel gioco degli scacchi, programmatore di videogiochi (dai 17 anni) passando per una laurea in \textit{Computer Science}, alla fondazione del proprio studio di videogiochi (Elixir Studios) per poi ritornare nel mondo accademico per ottenere il suo PhD in neuroscienze cognitive nel 2009, campo nel quale ha coautorato numerosi articoli influenti su memoria e amnesia (es. rappresentazione della memoria episodica tramite \textit{scene construction} \cite{Hassabis2007Jul}) \cite{hassabisWiki}. Per arrivare infine a fondare DeepMind e la nuovissima società Isomorphic Labs, sempre sussidiaria di Alphabet Inc.

\par DeepMind iniziò infatti a focalizzarsi sull'insegnare ad un sistema di AI come giocare a vecchi videogiochi anni '70,'80 (es. Pong, Breakout, Space Invaders), per poi passare a Go

DeepMind è stata acquistata da Google nel 2014 per 500 milioni di dollari \cite{Guardian2014}.

\subsection{Etica}
Dopo l'acquisizione di Google l'azienda ha stabilito un'\textit{AI ethics board}.\\
DeepMind è uno dei membri fondatori di \textit{Partnership on AI} insieme ad Amazon, Google, Facebook, IBM e Microsoft, un'organizzazione dedicata all'interfaccia società-AI \cite{partnershiponai}.

- inserire roba su partnershipai (articoli e pilastri)
 
DeepMind ha anche aperto una nuova unità denominata DeepMind Ethics and Society e si è concentrata sulle questioni etiche e sociali sollevate dall'intelligenza artificiale avendo come consulente il famoso filosofo Nick Bostrom. Nell'ottobre 2017, DeepMind ha lanciato un nuovo gruppo di ricerca per studiare l'etica dell'IA.[5]

\subsection{Alphabet}
Alphabet è un'azienda statunitense fondata nel 2015 dagli stessi fondatori di Google (Larry Page e Sergey Brin) come \textit{holding} a cui fa capo Google LLC e altre società sussidiarie: oltre a DeepMind vi sono Calico, CapitalG, Waymo, Wing, Intrinsic, Nest Labs, Sidewalk Labs, Isomorphic Labs...\\ 
Da dicembre 2019 il CEO di Alphabet è Sundar Pichai \cite{cnbc}.
La fondazione di Alphabet a partire da Google è stata una scelta finalizzata a rendere più trasparenti le attività inerenti a Google e concedere una maggiore autonomia alle società del gruppo che operano in settori diversi da quello dei servizi internet.

\subsection{Isomorphic Labs}


 
\section{AlphaFold 1}
\section{AlphaFold 2}
\section{AlphaFold-Multimer e futuro}
\section{AlphaFold DB}



\section{Rischi per i metodi omologo}
Rischi anemia falciforme (1 amminoacido diverso).
Obiettivo tesi: come svicolare problemi dovuti a somiglianze sequenze ma funzione dverse. Spaventano! 
\section{Uso}
\section{Visualizzazione 3D del risultato}

\clearpage