
\vfill
\begin{list}{}{
		\leftmargin=.2\textwidth
		\rightmargin=.2\textwidth
		\listparindent=\parindent
		%\itemindent=\parindent
		\itemsep=0pt
		\parsep=0pt}
	\item\relax
	\say{\textit{Seduto in riva all'oceano [..] ebbi la consapevolezza che tutto intorno a me prendeva parte a una gigantesca danza; [..] le mie esperienze [in fisica delle alte energie] presero vita: «vidi» scendere dallo spazio esterno cascate di energia, nelle quali si creavano e distruggevano particelle con ritmi pulsanti; «vidi» gli atomi degli elementi e quelli del mio corpo partecipare a questa danza cosmica di energia}}\footnote{\fullcite{capra1975tao}} \\ \\
	
	
	\say{\textit{Chiedersi il perché del big-bang, o chiedersi perché la banana sia gialla possono sembrare cose infinitamente diverse
			in importanza e filosofia, ma in fondo non è vero. Se si studia veramente fino in fondo perché la banana è gialla, si arriva probabilmente
			al perché dell’origine delle cose. [..] tutti i perché hanno un che di profondo, il punto essenziale è mettere onestamente a lavoro la mente e la fantasia}}\footnote{Pier Luigi Luisi, comunicazione personale, Luglio 2021} \\ \\
		
	\say{\textit{[..] in Lui, le doti della mente e del cuore armonizzavano singolarmente. [..] è rimasto vivo [..] ad insegnarci ancora come sulla vetta del sapere debba ardere il puro fuoco dell'entusiasmo}}\\
	\say{\textit{Confessava della sua scienza l'impareggiabile funzione morale: destare nei giovani l'amore alle cose naturali, rivelarne la bellezza, discoprirne l'ordine e le leggi voleva dire per Lui potenziare ed accrescere in loro l'ordine morale interiore}}\footnote{\fullcite{ruffiniLambertini}}
	
	
\end{list}
\vfill % equivalent to \vspace{\fill}
\clearpage
