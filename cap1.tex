\chapter{Bioinformatica}
\section{Di cosa si occupa}
\par Una parte importante della bioinformatica si occupa dell'utilizzo di strumenti informatici finalizzati a manipolare, archiviare e confrontare stringhe e sequenze di caratteri.\\
La bioinformatica tuttavia non si ferma all’analisi delle sequenze. Tra le più interessanti applicazioni bioinformatiche odierne vi sono quelle incentrate sull’analisi strutturale. \\
Difatti la bioinformatica pone le sue fondamenta nel campo della \textit{structural bioinformatics}: per portare un esempio il database PDB (\textit{Protein Data Bank}) nasce nel 1977 per archiviare coordinate atomiche e legami derivati dagli studi cristallografici sulle proteine \parencite{bernstein77}. \par

\section{Background filosofico}
Buttaci un po' di filosofia della scienza e di quali cambiamenti potrebbe apportare alla struttura delle rivoluzioni scientifiche. Cita Fleck in qualche modo!
Trova casi di cambi di paradigma e a "riscoperte" tornate alla ribalta grazie all'informatica. Magari l'informatica è un modo, analizzando tanti dati, di contrastare i bias nella scienza?