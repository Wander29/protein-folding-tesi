\chapter{Background}

\textit{Cos'è la vita? Da dove viene?} - Fino al 18° secolo per rispondere a tale quesito si faceva riferimento alla fede nel vitalismo: l'esistenza di una forza vitale non subordinata a leggi della chimica e  della fisica.
Il cambiamento avvenne nel 19° secolo.
Un'importante svolta fu il lavoro di Louis Pasteur che stabilì un collegamento fra processi vitali e reazioni chimiche: la conversione di zucchero in alcool (fermentazione) era un risultato della crescita di microorganismi.
\par Successivamente vi sono i lavori di Berthelot e Buchner (premio Nobel in Chimica 1907) il quale dimostrò che era possibile ottenere la fermentazione in assenza di microorganismi, usando solamente sostanze estratte da essi.
Queste sostanze furono chiamate \textit{enzimi} (dal ted. Enzym, letteralmente «dentro il lievito»\cite{enzimaTreccani}). Non si conosceva la loro natura chimica, si scoprì successivamente che tutti gli enzimi sono proteine (dal greco «primario», «che occupa la prima posizione»).
Queste proteine agivano da catalizzatori: acceleravano le reazioni chimiche all'interno delle cellule e nei tessuti senza cambiare la loro natura, quindi senza consumarsi, e senza entrare nei prodotti finali della reazione.
\par La scoperta degli enzimi portò ad un cambio di paradigma nel pensiero scientifico riguardo le origini della vita: veniva ora considerata come la conseguenza di numerosi processi chimici resi possibili dalle proteine\cite{kessel_ben-tal_2018}.
I fondamenti del pensiero biologico si spostarono dal vitalismo al meccanicismo secondo il quale tutti i fenomeni naturali, vita compresa, sono governati dalle stesse leggi, sia per sostanze organiche che inorganiche.
\par Oltre agli enzimi ci sono altre proteine importanti, uno degli esempi più noti è l'emoglobina, proteina animale adibita a trasportare ossigeno dai polmoni agli organi e ai tessuti del corpo così come a riportare CO$_{2}$ ai polmoni. 
\par L'inconorazione delle proteine a \textit{macromolecole più importanti della vita} si può legare ad un'altra svolta nel pensiero scientifico avvenuta nella seconda metà del 20° secolo: la rivoluzione genetica. 
Le proteine sono ben più che "macchine molecolari": sono i prodotti primari dei geni, responsabili, tra altri, dell'espressione dell'informazione genetica.


\section{Background biologico}
\subsection{Organizzazione molecolare dei viventi: le cellule}


Background biologico
\subsection{DNA, RNA}
• DNA, fenotipo e genotipo
• RNA
\subsection{Dogma centrale della biologia}

• codoni, amminoacidi
\subsection{Proteine: le macromolecole più importanti della vita}

Importante funzione degli enzimi è correlata alla digestione negli animali. Enzimi come le amilasi e le proteasi sono in grado di ridurre le macromolecole (nella fattispecie amido e proteine) in unità semplici (maltosio e amminoacidi), assorbibili dall'intestino

Tutti gli enzimi sono proteine, ma non tutti i catalizzatori biologici sono enzimi, dal momento che esistono anche catalizzatori costituiti di RNA, chiamati ribozimi


\section{Background informatico}

Background informatico
• bioinformatica
• database bioinformatici
• machine learning
• reti neurali, deep learning

\clearpage