Trasformare l’esperienza dell’università in qualcosa di positivo, di progressivo, che può alimentare il fuoco delle mie passioni
Fai qualcosa di specifico, renditi esperto.\newline

Guida il lettore da 0 ad Alphafold facendolo meravigliare davanti alla bellezza della bioinformatica, e della vita.\newline

Medita e poi scrivi qui: non passare da fonti terze. Non perdere il flusso.\\
Tu stai scrivendo qualcosa per te, non per il mondo. Scrivi, poi confrontati. Se ti confronti è normale che ti vedi inferiore. Come puoi invece essere inferiore a te stesso? \newline

Ciò che conta è fare, fare, fare, mettere in pratica.\newline

Hai scelto tu di uscire dall’informatica. Hai paura di risultare ignorante in biologia? Hai paura di esserti immischiato in un campo a te esterno e di sembrare “capiscione”? \\
1. Non ne sa quasi nulla nessuno dei prof 2.Non interessa loro 3. ho Mario Pirchio a cui chiedere aiuto 4. voglio uscire dall’informatica pura. Non mi fido. Non mi interessa. Qui mi interesso 5.affronta la responsabilità. Ho la responsabilità di creare la mia strada e crederci, di laurearmi per mio padre e la mia famiglia.\newline

Mentre disegnavo ho notato che ciò che mi spingeva a a migliorare il disegno era riuscire a intravedere il risultato finale in quello che stavo facendo. Non stavo tracciando una linea su un foglio. Stavo facendo piccoli passi per mettere su carta ciò che vedevo dentro di me (non nella mente, ma nel cuore).\\ Realizzavo una piccola parte di me al di fuori di me. E vedere che ciò che stavo creando si stava avvicinando a ciò che avevo in mente mi dava una soddisfazione immensa. E questa felicità mi spingeva tantissimo a continuare e a migliorarmi.\\
Voglio scrivere questo documento per realizzare una piccola parte di me all’esterno di me.\\
L’obiettivo del disegno era realizzare un ritratto di Thich Nhat Hanh, per esprimere la mia gratitudine nei suoi confronti.\newline

Obiettivo finale: realizzare un documento riguardante il background della bioinformatica e lo studio di AlphaFold per esprimere la mia speranza che l’informatica possa essere usata per il bene della Vita, che ci possa avvicinare ad una comprensione maggiore di essa e di quanto ogni fenomeno sia interrelato.\newline

La tesi serve a dimostrare una ipotesi che avete elaborato dall’inizio, non a mostrare che voi sapete tutto
\clearpage