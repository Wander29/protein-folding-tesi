\documentclass[12pt]{report}
\usepackage[top=2.5cm, bottom=2.5cm, left=3cm, right=2.5cm, centering]{geometry}

% Interlinea
\linespread{1.2}
\usepackage[italian]{babel} % applicazione regole di scrittura
\usepackage{float} % per il posizionamento delle immagini
\usepackage{listings} % per il codice di programmazione
% \usepackage[utf]{inputenc}
\usepackage{fontspec}
\usepackage{graphicx}
\usepackage[colorlinks=true,linkcolor=black,anchorcolor=black,citecolor=blue,filecolor=black,menucolor=black,runcolor=black,urlcolor=black]{hyperref}

\usepackage[backend=biber,style=ieee,sorting=none]{biblatex}
\addbibresource{bibliografia.bib}

\begin{document}
	
\author{Ludovico Venturi}
\title{Tesi Triennale in Informatica}
	
% Frontespizio
\begin{titlepage}
	\begin{figure}
		\centering\includegraphics[scale=0.2] {images/UNIPI_Logo.png}
	\end{figure}
	
	\begin{center}
		{\LARGE{ Corso di Laurea Triennale in Informatica (L-31) \\}}
		\vspace{2cm}
		{\Large { TESI DI LAUREA }}\\
		\vspace{2cm}
		{\LARGE { \textbf{Protein Folding: dai metodi classici alla rivoluzione di AlphaFold} }}
	\end{center}
	
	\vspace{2cm}
	
	\begin{minipage}[t]{0.6\textwidth}
		{\large{\textbf{Relatore}\\ Prof. \textbf{Paolo Milazzo}}}
		\vspace{0.5cm}
	\end{minipage}\hfill\begin{minipage}[t]{0.4\textwidth}\raggedleft
		{\large{\textbf{Candidato} \\ Ludovico \textbf{Venturi}\\ }}
	\end{minipage}
	
	\vspace{25mm}
	
	\centering{\large{\bf ANNO ACCADEMICO 2020/2021 }}
\end{titlepage}
% Fine frontespizio

\tableofcontents
\clearpage

% Trasformare l’esperienza dell’università in qualcosa di positivo, di progressivo, che può alimentare il fuoco delle mie passioni
Fai qualcosa di specifico, renditi esperto.\newline

Guida il lettore da 0 ad Alphafold facendolo meravigliare davanti alla bellezza della bioinformatica, e della vita.\newline

Medita e poi scrivi qui: non passare da fonti terze. Non perdere il flusso.\\
Tu stai scrivendo qualcosa per te, non per il mondo. Scrivi, poi confrontati. Se ti confronti è normale che ti vedi inferiore. Come puoi invece essere inferiore a te stesso? \newline

Ciò che conta è fare, fare, fare, mettere in pratica.\newline

Hai scelto tu di uscire dall’informatica. Hai paura di risultare ignorante in biologia? Hai paura di esserti immischiato in un campo a te esterno e di sembrare “capiscione”? \\
1. Non ne sa quasi nulla nessuno dei prof 2.Non interessa loro 3. ho Mario Pirchio a cui chiedere aiuto 4. voglio uscire dall’informatica pura. Non mi fido. Non mi interessa. Qui mi interesso 5.affronta la responsabilità. Ho la responsabilità di creare la mia strada e crederci, di laurearmi per mio padre e la mia famiglia.\newline

Mentre disegnavo ho notato che ciò che mi spingeva a a migliorare il disegno era riuscire a intravedere il risultato finale in quello che stavo facendo. Non stavo tracciando una linea su un foglio. Stavo facendo piccoli passi per mettere su carta ciò che vedevo dentro di me (non nella mente, ma nel cuore).\\ Realizzavo una piccola parte di me al di fuori di me. E vedere che ciò che stavo creando si stava avvicinando a ciò che avevo in mente mi dava una soddisfazione immensa. E questa felicità mi spingeva tantissimo a continuare e a migliorarmi.\\
Voglio scrivere questo documento per realizzare una piccola parte di me all’esterno di me.\\
L’obiettivo del disegno era realizzare un ritratto di Thich Nhat Hanh, per esprimere la mia gratitudine nei suoi confronti.\newline

Obiettivo finale: realizzare un documento riguardante il background della bioinformatica e lo studio di AlphaFold per esprimere la mia speranza che l’informatica possa essere usata per il bene della Vita, che ci possa avvicinare ad una comprensione maggiore di essa e di quanto ogni fenomeno sia interrelato.\newline

La tesi serve a dimostrare una ipotesi che avete elaborato dall’inizio, non a mostrare che voi sapete tutto

Ludo non dimenticarti quanta luce hai, sei ricco di una bellezza tanto speciale, non lasciare che altri te la nascondano\\
... \textit{ti ringrazio per ciò che sei}...\\
Tutto ciò che sei, che dici, che fai è meraviglioso\\
Dovresti sentirti in colpa con te stesso se invece abbandonassi tutto e tornassi a casa per paura di sbagliare\\
Ludo non hai bisogno di me, nè di tua madre nè di nessun altro. Tu sei una persona davvero meravigliosa, sei forte, hai tanta luce in te. Una cosa che penso di sapere è che potresti fare qualsiasi cosa, andare da qualsiasi parte. E se lo vorrai io ci sarò in ogni caso, non hai bisogno del mio appoggio per raggiungere quello che vuoi, ma io sono qui, e ci resto per tutto il tempo che vorrai.\\
E potessi starti vicina ogni notte e risvegliarmi accanto a te la mattina farei il tifo per te direttamente dalla prima fila ;")\\
ci credo davvero nel risultato positivo che potrai scoprire tra un po', non demordere prima o poi arriverà esattamente quella cosa che stavi aspettando e tutto andrà a posto da sè [..Sophie..]\\ \\

Una cosa per volta. Svuota il cervello. Adesso il mondo ti sembra pieno di problemi. Ne esiste solo uno per te: il tuo obiettivo. Se pensi a tutte le possibilità rimani fermo. Va solo in direzione dell'obiettivo. Poi al prossimo ci si penserà una volta raggiunto. Nessuno ti mette i bastoni fra le ruote, siamo con te. [papà] \\ \\


Non solo hai dato voce alla tua vita, ma sei stato in grado di renderla poesia, sono veramente orgoglioso di te e di come ti ho rivisto dopo tanto tempo perché, te lo dico con tutto il cuore, hai fatto dei passi da gigante, dei passi enormi e veramente complimenti. [..diego] \\ \\
Quando l'ansia bussa alla porta ringhio contro di lei: "ti affronto". Sono qui, avanti. Mi fermo e la guardo negli occhi. Affronto la vita, senza scappatoie. \\

Non abbiate paura di rischiare per non sbagliare. Mordete la vita. Sporcatevi le mani [Mattarella] \\

in Lui, le doti della mente e del cuore armonizzavano singolarmente.\\

L'amore all'indagine fu veramente la sola grande passione della Sua vita\\

Egli è rimasto vivo [..] ad insegnarci ancora come sulla vetta del sapere debba ardere il puro fuoco dell'entusiasmo. \\

tanto lo scienziato e l'uomo erano fusi in Lui; tanto la scienza era entrata ad invadergli la vita; tanto la sua umanità si era interamente votata alla
scienza. Al rigorismo scientifico disposava un caldo sentimento di artista, un'ispirazione profonda, intuitiva, che Lo guidava nel trovare i temi di ricerca, che Lo confortava nella dura fatica del suo lavoro quotidiano. \\

 era nemico nei suoi studi dell'arida arte meccanica. Se un giovane faceva ricorso a Lui desideroso di approfondire problemi o di battere la via della ricerca scientifica, Egli era pronto a divenirgli non solamente maestro, ma, più che maestro, padre. Si rallegrava con tanta spontanea compiacenza dei successi dei suoi allievi, come se questi fossero altrettanti suoi figliuoli: ed era pronto a
sorreggerli non solamente dinanzi agli ostacoli degli studi, ma ancora di fronte alle difficoltà della vita. Dal conforto che Egli recava al cuore dei suoi allievi ritraeva il suo medesimo conforto.\\

Confessava della sua scienza l'impareggiabile funzione morale: destare nei giovani l'amore alle cose naturali, rivelarne la bellezza, discoprirne 1' ordine e le leggi voleva dire per Lui potenziare ed accrescere in loro l'ordine morale interiore


[Lambertini in onore ad Angelo Ruffini, 1930]

\clearpage
% \addcontentsline{toc}{chapter}{Riassunto} % Capitolo non numerato
	
	\vspace*{\fill}
	\begin{center}
		\textbf{\LARGE Riassunto}\newline\newline
	\end{center}
	\begin{list}{}{%
			\leftmargin=.2\textwidth
			\rightmargin=.2\textwidth
			\listparindent=\parindent
			%\itemindent=\parindent
			\itemsep=0pt
			\parsep=0pt}
		\item\relax
			Va posto al centro della seconda pagina e non dovrebbe superare le 20 righe.\\
			Va posto al centro della seconda pagina e non dovrebbe superare le 20 righe.
			Va posto al centro della seconda pagina e non dovrebbe superare le 20 righe.\\
			\par Va posto al centro della seconda pagina e non dovrebbe superare le 20 righe.
			Va posto al centro della seconda pagina e non dovrebbe superare le 20 righe.\\
			Va posto al centro della seconda pagina e non dovrebbe superare le 20 righe.\\
			Va posto al centro della seconda pagina e non dovrebbe superare le 20 righe.
		
	\end{list}
	\vfill % equivalent to \vspace{\fill}
	\clearpage



\clearpage % Riassunto

% \chapter{Acapitolo - riposiziona il mio contenuto}

\addcontentsline{toc}{chapter}{Introduzione}

\textbf{{\LARGE Introduzione\newline\newline L'informatica: un potente strumento}}
\vspace{1cm}

Illustrare il mio obiettivo e la suddivisione del lavoro, dopo aver esposto la mia posizione sui rischi e le prospettive positive aperte dall'informatica.\\\\


-- continua l;'introduzione dall'aritcolo: Soft computing methods for the prediction of protein tertiary
structures: A survey --


- Soft computing
I paradigmi del soft computing per la predizione della struttura delle proteine sono:
\begin{itemize}
\item ANNs: artificial neural networks
\item EC: evolutionary computation
\item SVMs: support vector machines
\end{itemize}
Inoltre i metodi per la predizione della struttura delle proteine possono essere ulteriormente classificati in base ad un'approssimazione biologica \cite{marquez2015soft}:
\begin{itemize}
	\item homology-based methods
	\item threading methods
	\item \textit{ab initio } methods
\end{itemize}


\section{Background filosofico}
Buttaci un po' di filosofia della scienza e di quali cambiamenti potrebbe apportare alla struttura delle rivoluzioni scientifiche. Cita Fleck in qualche modo!
Trova casi di cambi di paradigma e a "riscoperte" tornate alla ribalta grazie all'informatica. Magari l'informatica è un modo, analizzando tanti dati, di contrastare i bias nella scienza?



\section{Determinazione sperimentale della struttura delle proteine}

Ci sono 3 tecniche sperimentali che possono essere usate per generare informazioni a risoluzione atomica sulla struttura delle proteine.

\section{CASP}
CASP (\textit{Critical Assessment of Structure Predictions}) è una sfida biennale dove gruppi di ricerca si sfidano cercando di realizzare predizioni di strutture di proteine la cui sequenza amminoacidica è nota ma non la struttura determinata sperimentalmente, che verrà utilizzata per stabilire l'accuratezza dei metodi in gara. \\

Nel 2020 gli organizzatori del CASP14 hanno rinosciuto AlphaFold come soluzione del \textit{protein–structure–prediction problem}. \\

\subsection{Valutazione dell'accuratezza delle predizioni}

Le tecniche di valutazione della predizione della struttura delle proteine richiede criteri ogettivi sulla similarità tra un modello computazionale e la struttura di riferimento determinata sperimentalmente.\\

La misura di valutazione dell'accuratezza oggi è utilizzata è il lDDT (\textit{local Distance
	Difference Test}) \cite{mariani2013lddt}. \\

Le misure precedenti si basavano su una superposizione globale di atomi di carbonio ed erano fortemente influenzate dai movimenti di dominio e non assicurano l'accuratezza di detagli atomici locali nel modello.



\chapter{AlphaFold}
AlphaFold è un sistema di \textit{Artificial Intelligence }(AI) sviluppato da DeepMind che realizza predizioni allo stato dell'arte sulla struttura delle proteine basandosi sulle loro sequenze amminoacidiche.

\section{DeepMind}

DeepMind è un'azienda inglese di Intelligenza Artificiale sussidiaria di Alphabet Inc.; in altre parole DeepMind è una società controllata: Alphabet Inc. detiene la maggioranza dei voti nell'assemblea ordinaria o un'influenza dominante sull'amministrazione.

\par DeepMind è stata fondata nel 2010 da Demis Hassabis, Shane Legg e Mustafa Suleyman. La società ha sede a Londra con centri di ricerca in Canada, Francia e Stati Uniti \cite{deepMindWiki}.

Può risultare interessante osservare la correlazione fra i primi lavori di DeepMind e la vita di Demis Hassabis, una vita ricca di sfaccettature: bambino prodigio nel gioco degli scacchi, programmatore di videogiochi (dai 17 anni) passando per una laurea in \textit{Computer Science}, alla fondazione del proprio studio videoludico (Elixir Studios) per poi ritornare nel mondo accademico per ottenere il suo PhD in neuroscienze cognitive nel 2009, campo nel quale ha coautorato numerosi articoli influenti su memoria e amnesia (es. rappresentazione della memoria episodica tramite \textit{scene construction} \cite{Hassabis2007Jul}) \cite{hassabisWiki}. Per arrivare infine a fondare DeepMind e la nuovissima società Isomorphic Labs, sempre sussidiaria di Alphabet Inc.

\par DeepMind iniziò infatti a focalizzarsi sull'insegnare ad un sistema di AI come giocare a vecchi videogiochi anni '70,'80 (es. Pong, Breakout, Space Invaders), per poi passare a Go

DeepMind è stata acquistata da Google nel 2014 per 500 milioni di dollari \cite{Guardian2014}.

\subsection{Etica}
Dopo l'acquisizione di Google l'azienda ha stabilito un'\textit{AI ethics board}.\\
DeepMind è uno dei membri fondatori di \textit{Partnership on AI} insieme ad Amazon, Google, Facebook, IBM e Microsoft, un'organizzazione dedicata all'interfaccia società-AI \cite{partnershiponai}.

- inserire roba su partnershipai (articoli e pilastri)

DeepMind ha anche aperto una nuova unità denominata DeepMind Ethics and Society e si è concentrata sulle questioni etiche e sociali sollevate dall'intelligenza artificiale avendo come consulente il famoso filosofo Nick Bostrom. Nell'ottobre 2017, DeepMind ha lanciato un nuovo gruppo di ricerca per studiare l'etica dell'IA.[5]

\subsection{Alphabet}
Alphabet è un'azienda statunitense fondata nel 2015 dagli stessi fondatori di Google (Larry Page e Sergey Brin) come \textit{holding} a cui fa capo Google LLC e altre società sussidiarie: oltre a DeepMind vi sono Calico, CapitalG, Waymo, Wing, Intrinsic, Nest Labs, Sidewalk Labs, Isomorphic Labs, ...\\ 
Da dicembre 2019 il CEO di Alphabet è Sundar Pichai \cite{cnbc}.
La fondazione di Alphabet a partire da Google è stata una scelta finalizzata a rendere più trasparenti le attività inerenti a Google e concedere una maggiore autonomia alle società del gruppo che operano in settori diversi da quello dei servizi internet.


\section{Rischi per i metodi omologo}
Rischi anemia falciforme (1 amminoacido diverso).
Obiettivo tesi: come svicolare problemi dovuti a somiglianze sequenze ma funzione dverse. Spaventano! 


\chapter{Protein Engineering e campi applicativi}
Le proteine vivono in un ambiente acquoso, per questa ragione vengono continuamente a scontrarsi con molecole d'acqua che rendono il processo di ripiegamento molto movimentato, e poco lineare, un po' come andare in bicicletta in una tempesta \cite{}, si vorrebbe andare in una direzione definita ma si viene continuamente spinti da una parte e dall'altra.

Se conosco la sequenza degli amminoacidi, come posso prevedere la struttura finale della proteina?

Abbiamo circa 20.000 tipi di proteine.

Le proteine possono essere considerate realmente come delle macchine: usano energia per ruotare, pompare, spostare, causare movimento..

Le proteine svolgono la loro funzione in base alla loro forma e ai loro cambiamenti di forma.

(es. proteine rotanti, motorie, canali ionici..)
Non fanno rumore. 

Le proteine sono sì macchine ma non funzionano minimamente come le macchine create dall'uomo. non abbiamo elettromagneti, nessun sistema vivente lo ha, non usiamo batterie. 

C'è tanta tecnologia nel mondo microscopico che potrebbe venire usata per la tecnologia dell'uomo.

Target farmacologici: sostanze chimiche che possono legarsi alle proteine "cattive" e impedire così che possano legarsi ai loro target originari.
Cambiamento di forma.

I motori elettrici utilizzano metà dell'elettricità mondiale, motori a benzina consumano 1/3 dell'energia mondiale disponibile.
Abbiamo più macchine microscopiche nelle nostre dita che macchine grandi in tutto il mondo

È uno dei rari casi nella storia in cui la tecnologia da sviluppare si conosce già in gran dettaglio. Si tratta solamente di capire come trasportarla nel mondo macroscopico.

He didn't say how we might macrosize the principles of natural machines. He just said their intricacy and efficiency might guide our future thinking when we finally get the rules (such as protein folding) behind us. A machine that repairs itself and reproduces itself is far different than what we are doing in industry today. But nature's invention is not unblemished either. It comes with pain and death. So we have to pick up where mom left off and leave the world a better place than we found it.


 %roba scritta di getto che poi riposizionerò. Eoh.

\chapter{Introduzione}

\say{\textit{Il Buddha, Il Divino, dimora nel circuito di un calcolatore o negli ingranaggi del cambio di una moto con lo stesso agio che in cima a una montagna o nei petali di un fiore}}\footnote{\fullcite{pirsig1974zen}} \\

\say{\textit{Seduto in riva all'oceano [..] ebbi la consapevolezza che tutto intorno a me prendeva parte a una gigantesca danza; [..] le mie esperienze [in fisica delle alte energie] presero vita: «vidi» scendere dallo spazio esterno cascate di energia, nelle quali si creavano e distruggevano particelle con ritmi pulsanti; «vidi» gli atomi degli elementi e quelli del mio corpo partecipare a questa danza cosmica di energia}}\footnote{\fullcite{capra1975tao}}


“The view that all aspects of reality can be reduced to matter and its various particles is, to my mind, as much a metaphysical position as the view that an organizing intelligence created and controls reality.”

 All things and events, whether material, mental, or even abstract concepts like time, are devoid of objective independent existence. To possess such independent, intrinsic existence would imply that things and events are somehow complete unto themselves and are therefore entirely contained.”
 
 

\clearpage


\chapter{Background}

\textit{Cos'è la vita? Da dove viene?} - Fino al 18° secolo per rispondere a tale quesito si faceva riferimento alla fede nel vitalismo: l'esistenza di una forza vitale non subordinata a leggi della chimica e  della fisica.
Il cambiamento avvenne nel 19° secolo.
Un'importante svolta fu il lavoro di Louis Pasteur che stabilì un collegamento fra processi vitali e reazioni chimiche: la conversione di zucchero in alcool (fermentazione) era un risultato della crescita di microorganismi.
\par Successivamente vi sono i lavori di Berthelot e Buchner (premio Nobel per la Chimica 1907), il quale dimostrò che era possibile ottenere la fermentazione in assenza di microorganismi, usando solamente sostanze estratte da essi.
Queste sostanze furono chiamate \textit{enzimi} (dal ted. Enzym, letteralmente «dentro il lievito»\supercite{enzimaTreccani}). Non si conosceva la loro natura chimica, si scoprì successivamente che tutti gli enzimi sono \textit{proteine} (dal greco «primario», «che occupa la prima posizione» \supercite{proteinaTreccani}).
Queste proteine agivano da catalizzatori: acceleravano le reazioni chimiche all'interno delle cellule e nei tessuti senza cambiare la loro natura, quindi senza consumarsi, e senza entrare nei prodotti finali della reazione.

\par La scoperta degli enzimi portò ad un cambio di paradigma nel pensiero scientifico riguardo le origini della vita: veniva ora considerata come la conseguenza di numerosi processi chimici resi possibili dalle proteine \supercite{kessel_ben-tal_2018}.
I fondamenti del pensiero biologico si spostarono dal vitalismo al meccanicismo secondo il quale tutti i fenomeni naturali, vita compresa, sono governati dalle stesse leggi, sia per sostanze organiche che inorganiche.

\par L'inconorazione delle proteine a \textit{macromolecole più importanti della vita} si può legare ad un'altra svolta nel pensiero scientifico avvenuta nella seconda metà del 20° secolo: la rivoluzione genetica. 
Le proteine sono ben più che "macchine molecolari": sono i prodotti primari dei geni, responsabili, fra altri, dell'espressione dell'informazione genetica. È sullo sfondo di questa rivoluzione che l'informatica si è inserita all'interno del mondo della biologia.

\clearpage

\section{Background biologico}
\subsection{Organizzazione della vita: dagli atomi alle cellule}
Nonostante le grandi differenze in dimensione, dieta, riproduzione, morfologia, comportamento, vi è un tratto comune a tutti gli organismi viventi: sono composti di cellule. Tutte le cellule sono caratterizzate da una stupefacente somiglianza chimica poiché utilizzano molecole simili e hanno ereditato tutte le stesse intuizioni genetiche. Si pensa quindi vi sia un antenato comune a tutti i viventi: una cellula vissuta circa 3,5 miliardi di anni fa che conteneva un prototipo del macchinario universale della vita sulla Terra oggi \supercite{alberts2018essential}. \\

\par Prima di parlare di cellule è opportuno richiamare l'attenzione sulle strutture biologiche. L'organizzazione biologica si basa su una gerarchia di livelli strutturali \footnote{Questa sezione di background biologico si basa in larga parte sui personali \fullcite{EBN}, frequentato nell'a.a. 2020/21 come esame a libera scelta.}, ognuno dei quali poggia su un gradino sottostante: 

\begin{figure}[h]
	\centering
	\includegraphics[scale=0.45] {images/strutture_biologiche.png}
\end{figure}


\par Tutta la materia è costituita da 94 elementi chimici in natura (tralasciando quelli non stabili). La materia organica è composta per il 96\% da atomi di C, O, N, H (carbonio, ossigeno, azoto, idrogeno). Un atomo ha un nucleo composto da neutroni e protoni circondato da una nube di elettroni in rapido movimento. Il Dalton (Da) è l'unità della massa atomica, corrisponde al peso di un protone o neutrone: $1 Da = 1.7 \times 10^{-24}g$. Un elettrone pesa $0.0005 Da$. Gli elettroni più esterni sono chiamati \textit{elettroni di valenza} e determinano il comportamento chimico di un atomo.

\par Lo scheletro dei composti organici è formato da catene carboniose, lunghe catene di atomi di carbonio legati fra loro da legami covalenti (il tipo di legame chimico più forte). Salendo di un livello nella gerarchia strutturale si arriva alle macromolecole biologiche, fondamentali per le cellule: carboidrati, lipidi, acidi nucleici e proteine. I carboidrati sono combustibili cellulari e materiale da costruzione, i lipidi sono sia depositi di energia che gusci protettivi, gli acidi nucleici permettono di codificare l'informazione genica e le proteine sono alla base delle funzioni vitali.\\

La cellula è la più piccola unità in grado di vivere. Per \textit{vivente} si intende un essere dotato di: organizzazione interna, metabolismo, omeostasi, interazione con l'ambiente, adattamento, crescita e riproduzione.

\par Le cellule hanno dimensioni che variano dai 2$\mu m$ ai $cm$ delle uova di rana, gallina o struzzo ai $m$ di neuroni con lunghi assoni:

\begin{figure}[!h]
	\centering
	\includegraphics[scale=0.5] {images/cellule-dimensioni.png}
	\caption{(A) disegno di un neurone. (B) Paramecium. (C) superficie di un petalo di fiore di bocca di leone. (D) Macrofago. (E) Un lievito di fissione viene catturato nell'atto di divisione cellulare. Fonte: \cite{alberts2018essential}}
	\label{fig:cellule-dimensioni}
\end{figure}

\par È possibile dividere gli esseri viventi in due domini: \textit{procarioti} ed \textit{eucarioti}. Il primo include i due regni Bacteria e Archaea. Sono caratterizzati da cellule piccole, circa 1$\mu m$. Il secondo dominio include cinque regni: animali, piante, funghi, protisti e cromisti. Dispongono di cellule più grandi (circa 10-100 $\mu m$) dotate di compartimenti interni che dividono i processi cellulari.

La struttura tipica di una cellula animale è mostrata nella seguente figura:

\begin{figure}[!h]
	\centering
	\includegraphics[scale=0.17]{images/cellula-eucariotica2.png}
	\caption{Cellula animale. Fonte: \cite{eukaryoteBritannica}}
	\label{fig:cellula-animale}
\end{figure}

L'interno delle cellule è composto da ambiente acquoso per il 70-95\%. \\

- completa con vita e funzionamento basilare, senza DNA \\

Il ciclo di vita delle cellule si basa su 4 fasi: crescita, sintesi del DNA, crescita completa e mitosi (divisione cellulare). Le cellule dei mammiferi impiegano da 18 a 24 ore per completare un ciclo di mitosi, mentre i lieviti solamente 90 minuti. Per questa ragione il lievito da fornaio (\textit{Saccharomyces cerevisiae}) è usato come organismo modello in citologia e genetica: il suo genoma è stato il primo ad essere sequenziato completamente tra gli eucarioti \supercite{lievitoWiki}. 

\par Le cellule hanno una durata di vita molto variabile, ad esempio alcuni organismi unicellulari come le spore possono vivere anche decenni, così come i nostri neuroni, mentre i globuli bianchi vengono ricambiati ogni 2 giorni. \\

\par Gli strumenti utilizzati per indagare nel mondo microscopico riescono a mostrare dettagli che vanno dal limite di 200$nm$ del microscopio ottico (limite imposto dalla natura ondulatoria della luce) alla precisione di 1$nm$ del microscopio a trasmissione elettronica (che usa fasci di elettroni invece di fasci di luce):

\begin{figure}[!h]
	\centering
	\includegraphics[scale=0.6]{images/grandezze.png}
	\caption{(A) Il grafico elenca le dimensioni dei livelli strutturali biologici, le unità di misura relative e gli strumenti necessari per visualizzarli. (B) Uno stesso dettaglio a varie scale di grandezza: pollice, pelle, cellule, mitocondrio, ribosomi, insieme di atomi che formano parte di una proteina. I dettagli molecolari sono oltre la potenza del microscopio elettronico. Fonte: \cite{alberts2018essential}}
	\label{fig:microscopi-grandezze}
\end{figure}

\subsection{Concetti fondamentali in biologia}

\begin{itemize}
	\item \textit{Proprietà emergenti }\\
			Ad ogni livello di indagine, ovvero passando da un livello della gerarchia strutturale al superiore, si palesano nuove proprietà non riconducibili ai livelli più semplici: le proprietà emergenti. Una singola molecola d'acqua non è né solida né liquida.
	\item \textit{Teoria cellulare} \\
			Le cellule rappresentano le unità strutturali e funzionali degli organismi.
	\item \textit{Geni} \\
			Il perpetuarsi della vita è possibile grazie alla trasmissione dei geni.
	\item \textit{Forma e funzione} \\
			Forma e funzione sono correlate a tutti i livelli biologici. Se le ali degli uccelli non fossero così come sono essi non potrebbero volare, se i mitocondri non avessero striature non potrebbero svolgere la respirazione cellulare, se i neuroni non avessero lunghi assoni non riuscirebbero a comunicare oppure si pensi al \textit{paramecium} che si muove come un sommergibile grazie alle sue ciglia (vedi figura \ref{fig:cellule-dimensioni}B).
	\item \textit{Evoluzione} \\
			L'evoluzione rappresenta il tema centrale ed unificante della biologia, come si è già accennato sopra. Gli organismi sono sistemi aperti che interagiscono continuamente con l'ambiente, dotati di variabilità individuale e finalizzati alla competizione per la sopravvivenza. 
	\item \textit{Diversità e unità} \\
			Vi sono da 5 a 30 milioni di specie differenti eppure scendendo sempre di più nella struttura degli organismi si osserva una similitudine quasi sconcertante. Un esempio che ci riguarda è la somiglianza fra le ciglia di \textit{paramecium} e le ciglia di una cellula epiteliale delle vie aeree degli esseri umani: presentano la stessa sezione trasversale. Diversità e unità della vita sulla Terra sono due facce della stessa medaglia.
			- somiglianza nel DNA e implicazioni per informatica - 
			
\end{itemize}

\subsection{Dogma centrale della biologia}
• DNA, fenotipo e genotipo
• RNA

Il DNA nel nucleo è associato a delle proteine con cui forma un materiale fibroso chiamato cromatina, mostrandosi "sfilacciato" in modo da poter essere letto. Quando la cellula si riproduce tali fibre si ispessiscono divenendo visibili come strutture compatte e singole: i cromosomi. Il nucleolo non è provvisto di membrana e serve per la sintesi di RNA ribosomiale, cioè l'RNA che uscendo dai pori dell'involucro nucleare andrà nel citoplasma a formare i ribosomi. Dall'involucro nucleare può uscire RNA e proteine ma non il DNA attraverso i pori presenti su di esso. L'RNA uscirà e andrà nel citoplasma per dirigere la sintesi proteica.

genoma: indica patrimonio complessivo del DNA di una cellula

\par

\begin{figure}[!htb]
	\minipage{0.5\textwidth}
	\includegraphics[scale=0.45]{images/central-dogma.png}
	\caption{Dogma centrale in biologia}\label{fig:awesome_image1}
	\endminipage\hfill
	\minipage{0.5\textwidth}
	\includegraphics[width=\linewidth]{images/vita-autocatalitica.png}
	\caption{La vita è un processo autocatalitico}\label{fig:awesome_image2}
	\endminipage\hfill
\end{figure}


• codoni, amminoacidi
\subsection{Proteine: le macromolecole più importanti della vita}

\par Oltre agli enzimi ci sono altre proteine importanti, uno degli esempi più noti è l'emoglobina, proteina animale adibita a trasportare ossigeno dai polmoni agli organi e ai tessuti del corpo così come a riportare CO$_{2}$ ai polmoni. 

Importante funzione degli enzimi è correlata alla digestione negli animali. Enzimi come le amilasi e le proteasi sono in grado di ridurre le macromolecole (nella fattispecie amido e proteine) in unità semplici (maltosio e amminoacidi), assorbibili dall'intestino

Tutti gli enzimi sono proteine, ma non tutti i catalizzatori biologici sono enzimi, dal momento che esistono anche catalizzatori costituiti di RNA, chiamati ribozimi

In un organismo, nonostante tutte le cellule condividano gli stessi geni, cellule afferenti a organi o tessuti diversi esprimono geni differenti (\textit{espressione genica}).

\section{Background informatico}

Background informatico
• bioinformatica
• database bioinformatici
• machine learning
• reti neurali, deep learning

\clearpage
\chapter{Protein Folding}

\say{\textit{la forma è l'immagine plastica	della funzione}}\footnote{\fullcite{ruffini1925fisiogenia}}\\

La correlazione tra forma e funzione si rivela fondamentale nel caso delle proteine. Un canale ionico neuronale permette il passaggio di ioni grazie alla sua forma a canale; una ferritina cattura e immagazzina gli ioni ferro grazie alla sua forma a sfera cava. 

\par Il ripiegamento delle proteine (\textit{protein folding}) è il processo di ripiegamento molecolare attraverso il quale le proteine ottengono la loro struttura tridimensionale che permette loro di svolgere la loro funzione biologica. Il ripiegamento avviene sia contemporaneamente alla sintesi proteica nei ribosomi sia al termine di questa.

\par La prima teoria del ripiegamento proteico è stata proposta negli anni trenta del 20° secolo da Hsien Wu\supercite{wu1931studies}, legata al processo di denaturazione.  

\par robe \\


--- Biochemists now know the amino acid sequence for about
160 million proteins, with about 4.5–5 million added each
month, and the three-dimensional shape for about 40,000.
Researchers have tried to correlate the primary structure of
many proteins with their three-dimensional structure to dis-
cover the rules of protein folding. Unfortunately, however,
the protein-folding process is not that simple. Most proteins
probably go through several intermediate structures on their
way to a stable shape, and looking at the mature structure
does not reveal the stages of folding required to achieve that
form

Cos'è stu cazz e protein folding?

• cos’è il problema del protein folding: non solo la struttura finale\\
• 3 sottoproblemi: folding code, protein structure prediction e folding process\\
• le domande di principio del protein folding\\
• struttura delle proteine (4 livelli)\\
• Domini, Residui, Motivi, Giri\\
• postulato di Anfinsen, paradosso di Levinthal\\
• interazioni (covalenti, polari, van der Waals, ..)\\
• una forza dominante o tante piccole forze?\\
• Cambio di paradigma da metà anni ‘80\\
• chaperonine, sintesi\\
• misfolding (denaturazione)\\
• struttura e funzione, malattie e prioni\\
• Limiti al ripiegamento: angoli di tersione e piano di Ramachandran\\
• processo spontaneo: energia di Gibbs, entalpia, entropia\\


[ soft computing articolo]
-- strutture --
pri-
mary structure of proteins consist of linear sequences of twenty
natural amino acids joined together by peptide bonds. The sec-
ondary structure of a protein refers to the interactions due to
a regular arrangement of hydrogen bonds between CO and NH
groups (carboxyl and amino) of its amino acids, forming different
motifs ( ̨-helix, ˇ-sheet, loops and turns). The tertiary structure is
a description of the complex and irregular folding of the polypep-
tide chain in three dimensions. These complex structures are held
together by a combination of several molecular interactions (e.g.
ionic, hydrophobic or hydrogen bonds) that involve the amino acids
of the chain. The quaternary structure is the final dimensional struc-
ture formed by all the polypeptide chains making up a protein [13].

---- native state  ---
A protein spontaneously folds into a 3-dimensional structure
after having been manufactured in the ribosomes. A specific pro-
tein will fold in the same way and will end up with the same 3D
structure. This phenomenon is called the native state of the protein 
A folded protein can
have more than one stable folded state or conformation. Each con-
formation has its own biological activity. Anfinsen’s experiment
discovered that the amino acid sequence determines the native
structure of a protein

Anfinsen \supercite{anfinsen1972formation}

-- misfolding --
Sometimes, a protein can fold into a wrong shape. A single miss-
ing or incorrect amino acid could cause such a misfold. As already
stated, protein function is determined by its structure, which can
be inferred from the sequence of amino acids, therefore a mis-
fold implies that a protein can not fulfill its function correctly.
Alzheimer’s disease, Cystic fibrosis and other neurodegenerative
diseases, mad cow disease are now attributed to protein misfolding (are associated with an accumulation
of misfolded proteins.). The knowledge
of the misfolding factors and understanding the protein folding
process, would help in developing cures for these diseases. There-
fore, the knowledge of the structure of the protein provides a great
advantage for the development of new drugs and the design of new
proteins.

-- IDP ---
The structure of some proteins is difficult to determine
for a simple reason: A growing body of biochemical research
has revealed that a significant number of proteins, or regions
of proteins, do not have a distinct 3-D structure until they
interact with a target protein or other molecule. Their flexibil-
ity and indefinite structure are important for their function,
which may require binding with different targets at different
times. These proteins, which may account for 20–30% of
mammalian proteins, are called intrinsically disordered proteins
and are the focus of current research.


-- strumenti ---
Even when scientists have a correctly folded protein in
hand, determining its exact three-dimensional structure is
not simple, for a single protein has thousands of atoms. The
method most commonly used to determine the 3-D structure
of a protein is X-ray crystallography

\begin{figure}[h]
	\centering
	\includegraphics[scale=0.4]{images/aminoacid-tipi.jpeg}
	\caption{I 20 amminoacidi universali. Fonte: \cite{aminoacidTipi}}
	\label{fig:amminoacidi-tipi}
\end{figure}

\clearpage
\chapter{Predizione della struttura di proteine}

Il protein folding problem ha sia guidato che tratto beneficio dagli avanzamenti nei metodi sperimentali e computazionali\supercite{dill2008protein}. Uno dei maggiori obiettivi della biologia computazionale è proprio il Protein Structure Prediction (PSP), ovvero la predizione della struttura nativa tridimensionale di una proteina a partire dalla sua sequenza amminoacidica. Il PSP è il problema opposto al \textit{protein design} (la progettazione di nuove sequenze proteiche aventi delle specifiche attività).

\par Grazie al CASP\footnote{Critical Assessment of Structure Predictions, vedi la sezione \ref{sec:CASP}.}, alla crescita dei database sulle proteine, allo sviluppo dei metodi per omologia e di allineamento di sequenze e all'utilizzo del Deep Learning, i metodi computazionali hanno registrato incredibili progressi, come il livello raggiunto da AlphaFold può dimostrare. 

\par La predizione della struttura di proteine è uno strumento fondamentale: in medicina per la comprensione delle malattie da misfolding, nell'industria farmaceutica per risparmiare anni di laboriosi e costosi esperimenti correntemente richiesti per lo sviluppo di un singolo farmaco (\textit{drug design}), in biotecnologia per il design di nuovi enzimi e in generale per acquisire maggior conoscenza sul protein folding in tutti i suoi lati.

\section{Metodi e strumenti informatici}

La piccola percentuale di strutture determinate e il gap che continua a crescere con le sequenze conosciute (vedi sotto la sezione \ref{sec:database}) è una conseguenza della lentezza e della dispendiosità dei metodi sperimentali (e in parte anche dei progressi delle tecnologie di sequenziamento). I metodi computazionali, significativamente più veloci ed economici, potrebbero fornire una possibile soluzione a questo problema.

\subsection{Workflow e classificazione dei metodi per il PSP} \label{sec:workflow-psp}
{
	
Quando si parla di metodi per la predizione della struttura di proteine esistono due \textit{paradigmi fondamentali} per affrontare il problema:
\begin{itemize}
	\item paradigma \textit{ab initio}
	\item paradigma \textit{data-based}
\end{itemize}

Il paradigma \textit{ab initio} (o \textit{de novo}) si basa su un approccio puramente fisico, nel quale la struttura è predetta da zero simulando principi fisici.

\par Nel paradigma \textit{data-based} invece si fa uso di informazioni estratte da database di sequenze o strutture di proteine. 

\par Le proteine che esistono in natura oggi si sono sviluppate attraverso lunghi processi evolutivi, progredendo attraverso mutazioni casuali e selezione naturale. La rivoluzione genetica degli anni '50, consentendo la determinazione delle sequenze amminoacidiche, ha permesso il nascere di metodi di confronto delle sequenze. È per questo che si possono ricavare informazioni sulla struttura 3D di una sequenza amminoacidica cercando altre proteine con proprietà nella sequenza simili e una struttura nota.\\

\par È bene chiarire sin dall'inizio che metodi basati totalmente sul primo paradigma non sono computazionalmente trattabili. Per questa ragione, i metodi odierni per la PSP di sequenze senza struttura nota, sono sempre in qualche misura \textit{data-based}. Possono essere quasi totalmente basati sui dati come nel caso della modellazione per \textit{omologia} e \textit{fold recognition} oppure parzialmente basati sui dati negli altri casi.

\par L'utilizzo, anche parziale, di tecniche \textit{data-based} è necessario per guidare la ricerca nello spazio conformazionale tramite rappresentazioni più grossolane, in modo da superare il paradosso di Levinthal.

\par Varie osservazioni evolutive e strutturali supportano questo approccio. Si è visto che la struttura è più conservata della sequenza: un'identità anche solo del 50\% può implicare un'identità di ripiegamento ed è possibile ricavare informazioni da mutazioni coevolute.

\par L'approccio \textit{data-based} è anche supportato dall'osservazione che, sebbene il numero di famiglie di proteine multi-dominio cresca rapidamente, la scoperta di nuovi domini singoli sembra stabilizzarsi. Ciò suggerisce che la maggioranza delle proteine possa ripiegarsi in un numero limitato di domini strutturali, forse non più di 10.000-20.000. Per molte famiglie a singolo dominio la struttura di almeno un membro è conosciuta: si stima che ciò permetta di avere informazioni su più di 3/4 delle sequenze nei database\supercite{alberts2018essential}. \\

\par Una \textit{pipeline} standard per la previsione della struttura delle proteine è basata su fasi di previsione intermedie, nelle quali vengono dedotte delle astrazioni che, risultando più semplici della struttura 3D completa, rivelano delle informazioni importanti per guidare le successive ricerche e modellazioni. Queste informazioni possono essere chiamate "annotazioni della struttura delle proteine" (Protein structure annotations, PSA). Le annotazioni sono divise in 2 categorie a seconda delle informazioni che forniscono:
\begin{itemize}
	\item \textit{Annotazioni 1D}: informazioni sulla backbone, caratteristiche strutturali locali (es. formazione di strutture secondarie, accessibilità al solvente)
	\item \textit{Annotazioni 2D}: vincoli spaziali (es. contact map)
\end{itemize}

\begin{figure}[!htb]
	\centering
	\includegraphics[scale=1]{images/psa.jpg}
	\caption{Pipeline generica per la predizione della struttura 3D di una proteina. Questo schema vuole mettere in risalto gli step intermedi relativi alle annotazioni. Fonte\cite{torrisi2020deep}}
	\label{fig:psa}
\end{figure}

Ci sono due possibili situazioni in cui ci si può trovare quando si vuole modellare una proteina: si riesce a trovare almeno una proteina omologa (o con caratteristiche simili) oppure no. Nel primo caso la struttura trovata verrà chiamata \textit{template} e si affronterà una predizione di tipo \textit{template-based modeling} (TBM), più semplice, mentre nell'altro caso si affronterà una predizione \textit{template-free modeling} (FM).\\


\par Come già accennato, nel panorama attuale molti degli gli approcci oggi utilizzati per il PSP sono prevalentemente \textit{data-based}: i metodi puri \textit{ab initio} vengono raramente utilizzati per la predizione della struttura di proteine (per ragioni che verranno spiegate in dettaglio nella sezione \ref{sec:ab-initio}). Tuttavia, nella pratica, alcune intuizioni del paradigma \textit{ab initio} vengono usate in metodi prevalentemente \textit{data-based}, ad esempio la funzione euristica di valutazione che simula il campo di forza per calcolare l'energia potenziale. Le varie tecniche vengono utilizzate in combinazione: non vi è una singola tecnica principale e i metodi migliori sono proprio quelli che riescono ad integrare vari approcci. \\

\par  Quando si ha di fronte una sequenza di una proteina senza struttura nota e si vuole predire la sua forma tridimensionale, un metodo per il PSP attuale agirebbe nel seguente modo (vedi fig. \ref{fig:fm-tbm})\footnote{Ogni argomento o metodo citato verrà spiegato nel dettaglio successivamente, in questa sezione l'obiettivo è di fornire una visione globale degli argomenti.}:

\begin{figure}[!htb]
	\centering
	\includegraphics[scale=0.95]{images/FM-TBM.jpg}
	\caption{Step tipici negli approcci al PSP di tipo TBM e FM. Fonte\cite{pearce2021deep}}
	\label{fig:fm-tbm}
\end{figure}

\begin{enumerate}
	\item viene generata una MSA per ottenere informazioni evolutive ed	identificare sequenze omologhe
	\item viene profilata la sequenza, sfruttando anche i risultati dell'MSA, e viene usata per dedurre le annotazioni 1D e 2D
	\item viene scelto il tipo di modellazione adeguato
	\item vengono assemblati i frammenti o i template
	\item viene scelto il modello fra vari candidati, che sarà poi raffinato a livello atomico
\end{enumerate}

\par Nel 1° passo l'obiettivo è ottenere informazioni evolutive che serviranno sia per identificare sequenze omologhe che per dedurre le annotazioni. Se si riesce a trovare almeno una proteina omologa allora sarà possibile procedere alla \textit{modellazione per omologia} (\textit{homology modeling}). \\

\par Nel 2° passo si applicano delle deduzioni sulla sequenza target al fine di ricavare delle annotazioni sulla struttura, per due scopi:
\begin{itemize}
	\item impostare dei vincoli spaziali e strutturali per guidare la modellazione
	\item nel caso in cui non siano state trovate proteine omologhe: per identificare template globali al fine di applicare protocolli di \textit{fold recognition}; se non se ne trovano, tali informazioni verranno utilizzate per valutare i frammenti nella ricerca all'interno di una \textit{fragment library}
\end{itemize}

\par Si rientrerà nel caso del TBM sia che venga eseguita una modellazione per \textit{omologia} che una modellazione basata sul \textit{fold recognition}.
Si rientra invece nel caso del MF quando non si riescono a trovare dei template globali. In questo caso vengono principalmente utilizzate tecniche di modellazione \textit{fragment-based}, ovvero basate su frammenti di proteine che verranno poi integrati. Nel 3° passo, a seconda che ci si trovi nel caso TBM o MF vengono attuate le rispettive modellazioni.\\ 

\par Nel 4° passo l'assemblaggio è eseguito sotto la guida di una funzione euristica del campo di forza, che può essere \textit{energy-based} e/o \textit{knowledge-based}, combinata con una rete neurale profonda per la predizione di determinate caratteristiche. Nel caso di una modellazione TBM si hanno anche delle restrizioni spaziali sul modello. Vengono utilizzate tecniche specifiche per regioni non allineate, come i loop (\textit{loop modeling}). \\

\par Nel 5° passo vengono valutati i modelli, viene eseguita una valutazione della qualità (\textit{quality assessment, QA}) stimando l'accuratezza del modello (\textit{estimation of model accuracy}, \textit{EMA}) e infine viene eseguito il raffinamento. Tipicamente viene scelto il modello con minore energia.

\subsubsection{Nota sulla classificazione dei metodi}

La predizione della struttura di proteine è stata ed è tutt'ora un campo in evoluzione. Per tale ragione risulta difficile classificare e raggruppare i metodi in nette categorie. Agli albori i metodi erano divisi in \textit{ab initio} e \textit{comparative modeling}. Oggi il confine non è più così marcato (come si è potuto vedere dalla panoramica nella sezione precedente). Il CASP divide la modellazione in due classi principali in base alla difficoltà: TBM ed FM\supercite{kryshtafovych2021critical}, in base all'utilizzo o meno di informazioni ricavate da template (proteine con struttura 3D nota). 

\par Nonostante questa divisione possa risultare efficace, in questo lavoro si è ritenuto opportuno seguire una classificazione diversa, che provasse a delineare invece le idee alla base degli approcci odierni e raggrupparli in più livelli secondo questo principio. La motivazione risiede nel fatto che nessun metodo preso singolarmente può dare risultati soddisfacenti: negli anni si è assistito infatti ad un utilizzo combinato dei tanti metodi trovati sfocando sempre più i margini fra le categorie. Tutto questo ha creato confusione ed un utilizzo improprio dei termini. Come si vedrà, \textit{ab initio} indica il "puro" approccio fisico, mentre con la divisione operata dal CASP si tende ad utilizzare come sinonimi \textit{ab initio} e \textit{template-free modeling}\footnote{Un esempio è in \fullcite{torrisi2020deep}. Nonostante ciò è proprio da tale lavoro che si è preso spunto per l'idea delle annotazioni.}, cosa che, in fase di stesura della tesi, è stata reputata equivoca e forse addirittura erronea. Allo stesso tempo i metodi si sono evoluti negli ultimi anni, specialmente con l'avvento del Deep Learning, e le vecchie classificazioni \footnote{Compresa quella di \fullcite{kessel_ben-tal_2018}, su cui il presente capitolo si basa in parte, precisamente sul capitolo 3.4.} non rispecchiano più l'attuale struttura dei metodi usati. Si è deciso di focalizzare l'attenzione sulla struttura dei metodi per il PSP odierni e da qui provare ad astrarre verso l'alto. Si è scelto per tale ragione di dividere la fase di annotazione da quella di modellazione e di rimarcare la differenza basilare tra i due grandi paradigmi per il PSP. Il quadro di riferimento è il \textit{workflow} della figura \ref{fig:fm-tbm}, che delinea la struttura tipica di un metodo attuale (tralasciando i metodi end-to-end come AlphaFold2). 

}

\subsection{Soft computing e Deep Learning}

Nel corso degli anni il PSP è stato affrontato anche con approcci di \textit{soft computing}. Si sta parlando di approcci perlopiù \textit{data-based}. I principali metodi di \textit{soft computing }utilizzati fanno capo a queste tecniche\supercite{marquez2015soft}:

\begin{itemize}
	\item Machine Learning:
	\begin{itemize}
		\item ANN (Artificial neural network)
		\item SVM (Support vector machines), es. SVM-SEQ 
		\item k-Nearest Neighbors
		\item linear regression
		\item HMM (Hidden Markov Models)
		\item Support vector regression
	\end{itemize}
	
	\item EC (Evolutionary computing), es. MECoMaP
	\item approcci \textit{statistici}, basati principalmente sull'omologia e sul fold recognition
	\item modelli \textit{matematici}, come un adattamento della programmazione lineare intera
\end{itemize}

I limiti della computazione evolutiva sono: la difficoltà di trovare un criterio di arresto e la possibilità di convergere verso un massimo locale come risultato di una configurazione sfavorevole dei parametri. Utilizzando questa tecnica è necessario tenere conto di una corretta scelta della rappresentazione del problema, della funzione fitness, della dimensione della popolazione e del tasso degli operatori genetici. Ad esempio, una piccola dimensione della popolazione può far sì che l'EA non possa esplorare lo spazio sufficiente per trovare una soluzione corretta.

\par Come limiti della tecnica SVM invece si può parlare del fatto che i modelli del kernel overfittino il criterio di selezione del modello, della difficoltà nella selezione dei parametri ottimali della funzione del kernel e della complessità algoritmica e gli ampi requisiti di memoria nei compiti su larga scala.

\par Le reti neurali invece offrono un elevato grado di flessibilità. Oltre ai vettori di input codificanti di coppie di amminoacidi, è possibile includere neuroni con informazioni aggiuntive, ad es. lunghezza della sequenza, valori di idrofobicità dell'ambiente o informazioni evolutive, nonostante la codifica dei dati di input necessariamente restringa le possibili informazioni codificabili. Le reti neurali presentano comunque delle limitazioni di cui tenere conto, ad esempio l'uso di parametri appropriati e l'overfitting.

\subsubsection{L'arrivo del Deep Learning}
{
Il campo del PSP ha assistito a numerosi avanzamenti grazie ad approcci basati sul Deep-Learning (DL) come evidenziato dal successo di AlphaFold nell'ultimo CASP. Il DL sta diventando una delle tecnologie principali per vari domini scientifici: computer vision, natural language processing, speech recognition, guida autonoma, ecc. 

Anche se le reti neurali di tipo FFNN sono state usate per prevedere annotazioni 1D sin dagli anni '80\supercite{torrisi2020deep}\footnote{Queste reti erano tipicamente utilizzate nella loro cosiddetta versione "a finestra", in cui ogni segmento, composto da un numero fisso di amminoacidi in una sequenza, veniva trattato come input per un esempio separato. L'obiettivo del segmento era l'annotazione di interesse per uno degli amminoacidi in esso (solitamente quello centrale).}, è però solo negli ultimi 10 anni (specialmente negli ultimi 2 CASP) che si sta assistendo a vari avanzamenti nel PSP grazie al DL, in particolare nei seguenti campi\supercite{pakhrin2021deep}:

\begin{itemize}
	\item generazione di MSA (es. DeepMSA)
	\item predizione di contatti (contact map, es. TripletRes)
	\item predizione di distogrammi (es. RaptorX)
	\item predizione della distanza fra i residui (es. PDNET)
	\item guidare l'assemblaggio iterativo di frammenti
	\item valutazione dei modelli e raffinamento (es. QDeep)
	\item pipeline generale del PSP (es. trRosetta o AlphaFold1)
	\item approcci DL-based end-to-end (es. AlphaFold2)
	\item pulizia dei dati nel cryo-EM (es. PIXER)
	\item predizione guidata sperimentalmente dalla cryo-EM (es. DeepTracer)
	\item predizione di strutture multidominio (es. FUpred)
\end{itemize}

Tra i modelli di Deep Learning maggiormente utilizzati nei metodi odierni vi sono le ResNet\supercite{pakhrin2021deep}.
}
\subsection{Output e misure di valutazione}

\subsubsection{Modelli di output}
I modelli dei dati di output hanno lo scopo di rappresentare la struttura terziaria predetta di una proteina. I principali modelli sono\supercite{marquez2015soft}:

\begin{itemize}
	\item \textit{modello ad angolo di torsione}. Gli angoli di torsione ($\phi, \psi$) sono legati alle possibilità della catena polipeptidica di assumere determinate conformazioni e data una conformazioni ogni angolo di torsione è ben definito. Per tale ragione una possibile rappresentazione è
	\[ [(\phi_{1}, \psi_{1}), ..., (\phi_{n}, \psi_{n})] \]
	dove $n$ è il numero dei residui. Il grafico di Ramachandran consente di evitare le possibili collisioni fra gli atomi. \\
	
	\item \textit{modello reticolare}, nel quale ogni amminoacido può essere rappresentato come una coppia (x,y) dove $x$ e $y$ sono le coordinate in un reticolo 2D. Considerando i possibili movimenti un'altra rappresentazione potrebbe essere tramite vettori di direzione:
	
	\[ (L_{1}, L_{2}, ..., L_{n}) \]
	
	dove $L_{i} \in \{UP,DOWN,LEFT,RIGHT\}$.\\
	
	\item \textit{binary contact map}, nel quale vengono rappresentati i contatti fra i residui tramite una matrice $L\times L$ dove $L$ rappresenta il numero dei residui. Un elemento $(i,j)$ nella matrice rappresenta una coppia di amminoacidi che possono essere in contatto (1) o no (0). Si definisce contatto una distanza tra i residui inferiore ad una determinata soglia, tipicamente 8\angstrom. Gli atomi di riferimento per tale calcolo sono in genere $C_{\alpha}$ o $C_{\beta}$.
	
	\par Data una \textit{contact map} è possibile ricostruire il modello 3D di una proteina risolvendo il Molecular Distance Geomtry Problem (MDGP). Quando usate come modello di rappresentazione delle proteine, le mappe di contatto sono	utili anche per confrontare le strutture. \\
	
	\item \textit{distance matrix}, simile alla mappa dei contatti ma rappresenta le distance a valori reali invece del contatto binario\\
	\item \textit{hydrophobic-polar} (HP), nel quale una sequenza è rappresentata come una stringa $s\in (H,P)^{+}$, dove $H$ rappresenta un amminoacido idrofobico e $P$ un amminoacido idrofilo.
	
\end{itemize}

\subsubsection{Metriche di valutazione}
Le misure di qualità valutano l'affidabilità dei modelli 3D rispetto alla struttura della proteina determinata sperimentalmente, le principali sono\supercite{marquez2015soft}:

\begin{itemize}
	\item RMSD (Root mean square deviation, indica la deviazione standard) rappresenta la deviazione assoluta (in \angstrom) dei singoli atomi $C_{\alpha}$ tra il modello e la struttura conosciuta:
	
	\[  RMSD = \sqrt{\frac{1}{N} \sum_{i=1}^{N} \left|r_{i}^{model}-r_{i}^{real}\right|^{2}}  \]
	
	dove $r_{i}^{model}$ indica la posizione dell'i-esimo atomo $C_{\alpha}$ nel modello. È stata la metrica di riferimento dal CASP1 al CASP4. È stato abbandonato poiché:
	
	\begin{itemize}
		\item il punteggio è dominato da valori anomali in regioni scarsamente previste mentre allo stesso tempo è insensibile alle parti mancanti 
		\item dipende fortemente dalla sovrapposizione del modello con la struttura di riferimento
	\end{itemize}
	
	\item GDT\_TS (Global distance test\_total score), è usato come maggior criterio di valutazione nel CASP e descrive le percentuali di residui ben modellati nel modello rispetto al target:
	
	\[ GDT\_TS = 100 \times \frac{\sum_{d_{i}} \frac{GDT_{i}}{NT}} {4} \]
	
	dove $GDT_{i}$ è il numero di atomi $C_{\alpha}$ di una predizione che non deviano più di una soglia stabilita $d_{i}$ (in \angstrom) dai $C_{\alpha}$ della struttura conosciuta, dopo sovrapposizione ottima. $NT$ è il numero degli amminoacidi della proteina e $d_{i} \in \{1,2,4,8\}$. Essendo un criterio basato su sovrapposizione globale degli atomi $C_{\alpha}$ anch'esso soffre di limitazioni quando applicato a proteine flessibili e/o multi-dominio e non considera l'accuratezza nelle differenze fra atomi che non siano $C_{\alpha}$.\\
	
	\item lDDT (local Distance difference test), è una metrica di valutazione non basata su sovrapposizione globale che valuta differenze di distanze locali di tutti gli atomi in un modello, includendo la validazione di plausibilità stereochimica\supercite{mariani2013lddt}. lDDT misura quanto sia stato riprodotto l'ambiente di una struttura di riferimento in un modello di una proteina. È calcolato su tutti gli atomi. La struttura di riferimento può essere una singola struttura o un insieme di strutture equivalenti. 
	
	\par Valutando tutti gli atomi è in grado di catturare l'accuratezza, ad esempio, della geometria locale di un sito di legame o il corretto ripiegamento del nucleo di una proteina. È stato introdotto nel CASP9. \\
	
	\item TM\_score (Template modeling score), misura la somiglianza globale tra la struttura modello e quella conosciuta in base alla distanza tra ogni paio di residui. Il punteggio è compreso tra $(0, 1]$, dove 1 indica una corrispondenza perfetta tra due strutture. Generalmente punteggi inferiori a 0,20 corrispondono a proteine non correlate mentre le strutture con un punteggio superiore a 0,5 si pensa abbiano all'incirca lo stesso ripiegamento.\\
	
	\item per la valutazione delle \textit{contact map} vengono usate 3 misure:
	\begin{itemize}
		\item \textit{accuracy}, che rappresenta il numero di contatti correttamente predetti
		\item \textit{coverage}, che riflette la proporzione di contatti predetti diviso i contatti reali
		\item $X_{d}$, distribuzione dell'accuratezza della predizione dei contatti
		
					\[ accuracy=\frac{C}{C_{p}}; \quad coverage=\frac{C}{C_{t}}; \quad X_{d}=\sum_{i=1}^{15} \frac{P_{i}-P_{a}}{i} \]
		dove $C_{t}$ rappresenta il numero dei contatti reali, $C$ il numero di predizioni corrette, $C_{p}$ il numero totale dei contatti predetti, $P_{i}$ riflette il numero di coppie stimate la cui distanza è nel range $(4(i-1), 4i)$ e $P_{a}$ rappresenta il numero di coppie reali la cui distanza è nello stesso range.
	\end{itemize}
\end{itemize}


\subsection{Database e rappresentazione delle informazioni} \label{sec:database}
% pal 6.4.1 p.145
% Gu 10-13 p.468
% baxevanis 12 p.367

Come si è già visto è possibile descrivere una proteina attraverso la sua sequenza amminoacidica. Il risultato è una stringa di lettere alfabetiche, poiché ogni amminoacido corrisponde ad una determinata lettera (vedi fig. \ref{fig:codici-amminoacidi}). Si ricorda anche che ad un amminoacido può corrispondere uno o più codoni (3 paia di basi azotate nel DNA). \\

\par Per quanto riguarda la struttura delle proteine, il formato standard è il PDB (Protein Data Bank).

\subsubsection{Database di proteine}


--PDB--
The Protein Data Bank (PDB) is an archive of experimentally determined three-dimensional
structures of biological macromolecules that serves a global community of researchers, educators,
and students. The data contained in the archive include atomic coordinates, crystallographic
structure factors and NMR experimental data. Aside from coordinates, each deposition also
includes the names of molecules, primary and secondary structure information, sequence
database references, where appropriate, and ligand and biological assembly information, details
about data collection and structure solution, and bibliographic citations.


\par La predizione della struttura è importante per un semplice motivo: i biochimici conoscono oggi la sequenza amminoacidica per più di 225 milioni di proteine\supercite{proteienDBentries} (UniProt, con circa 4.5-5 milioni aggiunte ogni mese) ma sono state determinate solamente circa 160.000 strutture tridimensionali di proteine\supercite{proteienDBentries} (PDB\footnote{Al 3 Febbraio 2022 sono presenti 162.913 strutture di proteine nella versione dell'RCSB. Nel PDB ci sono anche strutture di altre macromolecole (complessi di acidi proteici-nucleici, DNA e RNA) per un totale, incluse le proteine, di 186.670\supercite{pdbStats}. Nel wwPDB (database globale) vi sono in totale 197.961 strutture di macromolecole\supercite{wwpdbStats}.}, con poco più di 10.000 strutture aggiunte ogni anno). 

\begin{figure}[!htb]
	\centering
	\includegraphics[scale=0.3]{images/pdb-statistica.png}
	\caption{Crescita complessiva del numero di strutture di proteine pubblicate nel PDB. Fonte\cite{pdbStats}}
	\label{fig:pdb-statistica}
\end{figure}


%baxevanis 12 p.373, 1



\subsection{Rappresentazione grafica}
%pal 6.4 p.146, ottima
%kessel p.174 (2.3 GRAPHIC REPRESENTATIONS OF PROTEINS)

QuteMol -> cignoni!

\subsection{CASP} \label{sec:CASP}
{
Dal 1994 il campo del PSP è stato stimolato, monitorato e quantitativamente valutato dalla competizione biennale CASP (\textit{Critical Assessment of Structure Predictions}). CASP è una sfida nella quale gruppi di ricerca si sfidano cercando di realizzare predizioni di strutture di proteine. È nota la sequenza amminoacidica di queste proteine target ma non la struttura sperimentale. Queste sequenze di proteine provengono da laboratori  congiunti: è pianificata la determinazione delle loro strutture native in vitro, che verrà infine utilizzata per stabilire l'accuratezza dei metodi in gara.

\par Questi esperimenti a livello di comunità sono cresciuti significativamente nel corso delle edizioni: dai 33 target e i 100 modelli sottomessi nel 1994 (CASP1) agli 82 target e più di 55.000 modelli sottomessi nel 2018 (CASP13)\supercite{abbass2020enhancing}.

\par Ogni due anni un insieme di sequenze di proteine sono rilasciate gradualmente nel corso di un paio di mesi, durante i quali gruppi di ricerca da tutto il mondo tentano di predire le loro strutture 3D e inviano i loro modelli (fino a 5 per target).

\par Nei primi 6 round (CASP1-6), i target erano classificati in 3 categorie: \textit{comparative modeling}, \textit{fold recognition} e \textit{ab initio}. Da allora i target sono stati classificati solo in 2 classi: le famose \textit{template-based modeling} e \textit{template-free modeling}. I target nella categoria TBM sono considerati "facili" mentre quelli in FM sono considerati difficili. È presenta anche la divisione fra metodi completamente automatici e previsioni ottenute usando intervento umano.

\par I risultati sono valutati sulla base di vari criteri, come il numero di residui la cui posizione è stata predetta con un certo livello di accuratezza, identificazione di strutture secondarie, limiti dei domini, contatti fra residui, regioni disordinate, ecc.

\par Analizzando i risultati del CASP negli anni si può notare che i metodi basati su un approccio fisico, nonostante l'aumento della potenza computazionale, hanno subito solo lievi miglioramenti.

\par L'importanza del CASP per la biologia strutturale è alta anche per il contributo che ha dato alla creazione dei metodi automatici, accessibili anche ai non esperti.

\par Nel 2020 gli organizzatori del CASP14 hanno riconosciuto AlphaFold come soluzione del \textit{protein–structure–prediction problem} ???. \\

}

\section{Annotazioni 1D sulla struttura} \label{sec:predizione-structural-features}
%- praveen 2014
%- pal 

Le "annotazioni monodimensionali sulla struttura delle proteine" (1D PSA) sono astrazioni che descrivono la disposizione della backbone proteica.

Le proprietà delle sequenze possono essere raggruppate in due tipi:
\begin{itemize}
	\item proprietà esplicitamente correlate con la struttura degli amminoacidi
	\item proprietà statistiche
\end{itemize}

Proprietà del 1° tipo sono collegate a: formazioni di determinate strutture secondarie, esposizione al solvente circostante (\textit{solvent accessibility}), determinati angoli di torsione, interazioni con alcuni amminoacidi, ecc. (es. GenTHREADER, SPARKS-X).

\par Le proprietà statistiche vengono invece identificate tramite le frequenze degli amminoacidi in vari MSA. Un modo di rappresentare queste tendenze è tramite \textit{profili}, i quali denotano amminoacidi che frequentemente appaiono in certe posizioni dell'allineamento. Un profilo del genere è più informativo di una semplice sequenza amminoacidica: contiene informazioni evolutive (es. livelli di conservazione di specifiche posizioni nella sequenza). È questa l'utilità dell'utilizzo di profili. Il profilo della proteina target è poi confrontato con i profili corrispondenti di tutte le proteine con ripeigamenti conosciuti (\textit{profile-profile alignemnt}) e le proteine sufficientemente simili alla proteina target possono essere usate come template globali. Un esempio di procedura profile-matching è HHPred, basata su confronti a coppie di profili HMM tra target e template.

\subsection{Allineamento di sequenze} \label{sec:MSA}
{
	Un allineamento di sequenze multiple (MSA) è una disposizione di più di due sequenze di amminoacidi o nucleotidi allineate in modo da posizionare i residui delle diverse sequenze in colonne verticali in una maniera appropriata. I metodi di MSA sono utilizzati in una grande varietà di analisi e pipeline nell'analisi del proteoma e del genoma e sono un passo iniziale essenziale nella maggior parte dei confronti filogenetici. Sono ampiamente utilizzati per aiutare a ricercare caratteristiche comuni nelle sequenze e possono essere usati per aiutare a prevedere le strutture bi e tridimensionali di proteine e acidi nucleici. Un MSA mostra il grado di similarità tra le sequenze.
	
	\par Di solito, si dovrebbe solo tentare di allineare sequenze filogeneticamente correlate e, quindi, omologhe. In questo caso, l'allineamento ideale avrà residui omologhi allineati nelle colonne.
	
	\begin{figure}[!htb]
		\centering
		\includegraphics[scale=0.5]{images/msa.jpeg}
		\caption{Multiple sequence alignment schematica di una sequenza proteica di varie specie. Fonte\cite{msaBioNinja}}
		\label{fig:msa}
	\end{figure}
	
	\par Le sequenze di proteine in genere risultano avere un grado di somiglianza maggiore rispetto alle sequenze nucleotidiche. Ciò è dovuto alla degenerazione del codice genetico e al fatto che quasi per ogni amminoacido esistano vari codoni che lo codificano, perciò differenti sequenze nucleotidiche possono codificare esattamente la stessa sequenze di amminoacidi.
	
	\par Un allineamento di sequenze multiple (MSA) può essere utilizzato per tracciare l'entità della divergenza evolutiva tra sequenze correlate. Rispetto a una singola sequenza, l'MSA fornisce informazioni sulle tendenze evolutive degli amminoacidi in ciascuna posizione della sequenza, il che caratterizza un "profilo" della sequenza.
	
}

\subsection{Predizione della struttura secondaria}
%-wiki
%-sejonwski 1989
%-deOliveira 2021

\section{Annotazioni 2D sulla struttura}

Le annotazioni riguardano predizioni dei contatti fra residui sulla base di informazioni co-evolutive (\textit{correlated mutation}), generando o una \textit{contact map}, o un \textit{distogramma} o una \textit{distance map} attraverso una \textit{deep residual neural network} (\textit{ResNet}). Il fine è ricavare restrizioni spaziali come contatti inter-residuo a lunga distanza, distanze specifiche, legami idrogeno e angoli di torsione.

Le "annotazioni monodimensionali sulla struttura delle proteine" (1D PSA) sono astrazioni che descrivono la disposizione della backbone proteica.

\begin{figure}[!htb]
	\centering
	\includegraphics[scale=0.6]{images/FM-template.png}
	\caption{Schema generale di un workflow FM. Il DL è un passo fondamentale nel predizione di vincoli spaziali. Fonte\cite{pakhrin2021deep}}
	\label{fig:fm-template-dl}
\end{figure}


Data la natura "black-box" dei metodi di DL è stata ad esempio proposta InterPretContactMap che fa riferimento al campo della xAI (explainable AI); combina reti neurali profonde con meccanismi di attenzione per aumentare la comprensibilità della predizione dei contatti.

\subsection{\textit{correlated mutation}}
{
	L'apparizione di mutazioni nelle sequenze delle proteine durante la loro evoluzione dipende da aspetti sia strutturali che funzionali. È interessante notare come l'analisi delle sequenze di famiglie di proteine mostri che certe posizioni tendono a \textit{coevolvere}. In altre parole l'apparizione di una mutazione in una posizione è accompagnata da una mutazione in un'altra posizione. 
	
	\par È stato suggerito che un tale collegamento possa avvenire in posizioni vicine nello spazio tridimensionale e che i residui considerati interagiscano fra loro. Se una mutazione in una posizione porta a una distruzione della sua interazione con la posizione adiacente, una mutazione compensatoria di quest'ultima potrebbe rimediare al problema.
	
	\par Ad esempio, se due posizioni erano originariamente polari e coinvolte in un'interazione elettrostatica favorevole, la mutazione di una posizione da polare a non polare distruggerebbe l'interazione. Se però la posizione adiacente muta anch'essa da polare a non polare, l'interazione originale può tramutarsi in un'altra interazione favorevole, stavolta non polare.
	
	\par Ciò suggerisce che sia possibile inferire posizioni a contatto nella struttura ripiegata di una proteina analizzandone le variazioni nella sequenza attraverso la sua evoluzione e osservando quali posizioni sono \textit{coevolute}. Gli accoppiamenti evoluzionisticamente inferiti possono essere usati come vincoli durante la modellazione.
	
	\begin{figure}[!htb]
		\centering
		\includegraphics[scale=0.65]{images/correlated-mutation.jpg}
		\caption{In alto, sequenza della proteina da predire, in cui ogni amminoacido è un cerchio. Sotto, un MSA con le sequenze di una famiglia correlata, tutte con lo stesso ripiegamento. A destra l'inferenza del contatto delle posizioni coevolute. Fonte\cite{marks2011protein}}
		\label{fig:correlated-mutation}
	\end{figure}
	
	\par Ci sono 3 principali ostacoli a questo approccio: rumore statistico (molte correlazioni sono solo frutto di rumore e quindi insignificanti), correlazioni fra residui distanti e numero insufficiente di posizioni correlate. Il metodo non risulta applicabile quando ci sono poche sequenze omologhe. Nonostante l'idea non sia nuova, solo nell'ultimo decennio tali metodi sono divenuti abbastanza accurati da poter essere usati nel PSP; metodi noti sono EVfol, Rosetta-GREMLIN, FILM3, MetaPSICOV, ecc.
	
}

The hypothesis behind the approach was that if
mutations that occur at two positions in an MSA are
correlated, these positions are more likely to form a
contact in 3D space. This is because there is evolutionary
pressure to conserve the structures of proteins and a
mutation at one position may be rescued by a correspond-
ing mutation at a nearby residue. The accuracy of co-
evolution-based contact map prediction remained low for
many years due to the inability to distinguish between
direct and indirect interactions, where indirect interac-
tions occur when residues appear to co-evolve but do not
actually form contacts. For example, if Residues A and B
are both in contact with Residue C, A and B often appear
as if they co-evolve even when there is no physical
contact between them. There is evidence showing that
such co-evolution may have a functional cause [22] rather
than a structural one, which resulted in the failure of
structure-based contact derivation.\supercite{pearce2021deep}

Progress in contact prediction remained stagnant for some
time. However, a leap in contact prediction accuracy took
place when algorithms started utilizing global prediction
approaches. Early methods mainly predicted contacts
between residue pairs one-at-a-time using techniques
such as mutual information, thus ignoring the interactions
with other residue pairs and the global context in which
the interactions took place; this is largely why it was
difficult for these local methods to distinguish between
direct and indirect interactions. The introduction of
global statistical models determined through the use of
direct coupling analysis (DCA) was more successfully
able to distinguish between these direct and indirect
interactions [23,24]. 


\subsection{Distogrammi}


\section{Predizione della struttura 3D}

\subsection{\textit{homology modeling}}
{
	Nella modellazione per omologia ci si affida a somiglianze nella sequenza tra la proteina target e i template. I metodi per omologia sono perciò basati sul paradigma: \\
	\say{\textit{la sequenza codifica per la struttura}}.
	
	\par Sono metodi basati anche sull'osservazione che la struttura terziaria è più conservata della sequenza amminoacidica. Di conseguenza ci si aspetta una significativa somiglianza nella struttura fra proteine che condividono una notevole somiglianza tra le sequenze, anche non totale.
	
	\par In altri termini, due sequenze amminoacidiche molto simili (\textit{omologhe}) , in due proteine differenti ma evoluzionisticamente collegate, dovrebbero acquisire la stessa struttura locale.
	
	\par Un approccio che utilizzi la modellazione per omologia consiste tipicamente nei seguenti passi:
	\begin{enumerate}
		\item ricerca e selezione del template
		\item costruire un MSA (Multiple Sequence Alignment) che includa la proteina target e i template
		\item assegnare le coordinate spaziali dei template alla sequenza della proteina target
		\item raffinamento della struttura modello
		\item valutazione e validazione della struttura risultante
	\end{enumerate}
	
	\begin{figure}[!htb]
		\minipage{0.48\textwidth}
		\includegraphics[scale=0.43]{images/homology1.png}
		\caption{Diagramma di flusso della modellazione per omologia. Fonte: \cite{sliwoski2014computational}}
		\label{fig:omologia-flusso}
		\endminipage\hfill
		\minipage{0.48\textwidth}
		\centering
		\includegraphics[scale=0.5]{images/homology2.jpg}
		\caption{Schema esemplificativo di una modellazione per omologia. Fonte \cite{UNIL-homology}}
		\label{fig:omologia-esempio}
		\endminipage\hfill
	\end{figure}
	
	Nel 1° step si cerca una struttura modello, almeno una, tra le strutture conosciute avente un'alta somiglianza di sequenza. È più semplice se la struttura di un una proteina omologa molto simile è stata già risolta. Ci sono però altri gruppi di proteine, come le proteine di membrana, le cui strutture risolte sono scarse. Trovare i giusti template e caratterizzare la loro omologia è ciò che determina in genere il successo dell'intera predizione (una somiglianza minore del 30\% avrà risultati molto scarsi, mentre sopra al 50\% la predizione ha buona probabilità di essere di buon livello). È possibile in ogni caso che vi sia una somiglianza \textit{locale} anche quando la somiglianza globale è scarsa. Le principali tecniche si dividono nelle seguenti categorie:
	\begin{itemize}
		\item pattern recognition e ricerche euristiche, ad esempio BLAST (Basic Local Alignment Search Tool) e FastA
		\item profile and iterative alignment methods, come PSI-BLAST e HMM (Hidden Markov Models), quest'ultimo molto usato
		\item structure based threading, come THREADER e FUGUE
	\end{itemize}
	
	\par Nel 2° step vengono sfruttate informazioni evoluzionistiche per migliorare l'allineamento tra le sequenze dei template e del target. È difficile stabilire allineamenti fra omologhi distanti, come nel caso di target \textit{eucarioti} e template \textit{procarioti}. La tecnica più usata oggi è l'MSA, altre tecniche usate sono: programmazione dinamica, MSA, threading, HMM ecc. Ci sono vari tool per creare MSA, ad esempio: Clustal$\Omega$, MAFFT, MUSCLE, T-Coffee. Possono venire utilizzati più MSA insieme per sopperire a problemi di disallineamento di piccole regioni. Il MSA viene illsutrato dopo la sezione \ref{sec:MSA}.
	
	\par Nel 3° step ad ogni segmento della sequenza target viene assegnato un insieme di coordinate spaziali in accordo ai risultati del MSA. Tool noti sono MODELLER (soddisfazione di vincoli spaziali), NEST, COMPOSER e SWISS-Model. La struttura ottenuta potrebbe essere però deformata a causa dell'utilizzo di più template e numerosi inserzioni e cancellazioni. Possono essere presenti lunghezze e angoli dei legami non ottimali e atomi sovrapposti.
	
	\par Per ovviare a tali problemi nello step 4 si applica un processo di raffinamento, specialmente per quanto riguarda i loop (\textit{loop modeling}, vedi sez. \ref{sec: loop-modeling}), regioni fodamentali per i siti di legame, e i residui (\textit{side chain modeling}). Vengono applicati algoritmi che confrontano caratteristiche geometriche ed effettuano calcoli energetici che identificano configurazioni atomiche sfavorevoli. 
	
	\par Nel 5° step si valuta l'affidabilità della predizione. Si dice che un modello è affidabile in generale quando è basato su un template corretto e un allineamento approssimativamente corretto. Si può valutare tale affidabilità in vari modi:
	
	\begin{itemize}
		\item alcune qualità della struttura costruita possono essere confrontate con delle tendenze statistiche 
		\item se ci sono vari modelli predetti si calcola l'energia libera e si sceglie la struttura con minor energia libera (ad es. tramite ProSa)
		\item stereochimica (relativa alle proprietà spaziali delle molecole), ad esempio con PROCHECK
		\item la conservazione evoluzionistica a livello amminoacidico può essere correlata con il loro stato "esposto" o "seppellito" (l'idea di partenza è che il nucleo della proteina rimanga inalterato e la superficie sia variabile), ad esempio Profiles3D
		\item se si hanno a disposizione dati sperimentali della struttura nativa della proteina si può validare il modello con la consistenza a essi.
	\end{itemize}
	
	\subsubsection{Efficienza e limiti}
	
	Con una somiglianza maggiore del 50\% si registra una RMSD tra 1 e 2 \angstrom, ma è importante notare che non sempre proteine omologhe (vicine sequenzialmente) condividono la stessa funzione e struttura. Un esempio sono le proteine del lievito Gal1 e Gal3: 73\% di identità e 92\% di somiglianza. Queste due proteine hanno però sviluppato differenti funzioni, con Gal1 che è una galattochinasi mentre Gal3 è un induttore trascrizionale\supercite{platt2000insertion}.
	
	\par Non c'è quindi una soglia che assicuri una sicura predizione della struttura: molte proteine con una lontana somiglianza possono svolgere la stessa funzione mentre altre altamente simili possono svolgere funzioni diverse. Una regola empirica è considerare sequenze con più del 30-40\% di somiglianza come sequenze con una struttura o funzione simile.
	
	\begin{figure}[!htb]
		\centering
		\includegraphics[scale=1.2]{images/homology-grafico.jpg}
		\caption{Risultati dei metodi per omologia alla variazione dell'identità nella sequenza. (A) Dimostrazione schematica dell'uso di confronti di sequenze per identificare omologie strutturali. 'X' indica un qualsiasi amminoacido. (i) identità > 70\%: semplici allineamenti di sequenza sono sufficienti per trovare il corretto ripiegamento. (ii) Tra il 20 e il 30\% non sempre è possibile trovare il corretto ripiegamento; è necessario effettuare ulteriori raffinamenti. (iii) a bassi livelli l'utilità di questo metodo è molto bassa. (B) Somiglianza strutturale (in GDT\_TS score) al variare dell'identità della sequenza. Anche al 30\% il livello di somiglianza è significativo. Fonte\cite{joseph2014local}}
		\label{fig:omologia-grafico}
	\end{figure}
	
	\par Un'osservazione fondamentale risiede sulle basi in sé del metodo: dato l'affidamento pressoché totale nella modellazione comparativa, la struttura modello è condizionata necessariamente a essere più simile ai template che alla reale struttura nativa della sequenza target, nonostante i vari processi successivi di raffinamento che, data la loro natura approssimativa, non sono perfetti. 
	
	\par I problemi maggiori risultano nelle regioni con bassa somiglianza, come ci si può aspettare. Si sta parlando specialmente dei \textit{loop}, soggetti a mutazioni considerevoli durante l'evoluzione.
	
	\par Si incorrono in problemi con la modellazione per omologia quando si trattano proteine che non hanno omologhe tra le strutture conosciute, come le proteine di membrana, le quali sono difficili da da cristallizzare. \\
	
	\subsubsection{Conclusioni sulla \textit{modellazione per omologia}}
	
	\par Nonostante tutte le osservazioni fatte, \textit{la modellazione per omologia è correntemente il miglior metodo computazionale per predire la struttura delle proteine} e la sua applicabilità è destinata a crescere con l'aggiunta di nuove strutture determinate sperimentalmente da poter essere usate come template.
	
	\par La sfida principale che questi metodi hanno affrontato, e che AlphaFold ha "vinto", era raggiungere almeno una RMSD di 3\angstrom. Le sfide ancora da superare riguardano l'accuratezza su grandi proteine, su proteine con un contenuto significativo di strutture $\beta$ e la modellazione di proteine multi-dominio e di membrana.
	
	\par Oltre alla PSP i metodi per omologia sono anche usati nel drug design (per studiare le differenze strutturali fra le proteine bersagliate dallo stesso farmaco) e nello studio dei meccanismi catalitici.
	
}

\subsection{\textit{fold recognition via threading}}
{

Il \textit{threading} consiste nell'allineare la sequenza della proteina target con la struttura di proteine template, che possono essere trovate sulla base di proprietà condivise (vedi sez \ref{sec:predizione-structural-features} sulla predizione di caratteristiche strutturalI a partire dalla sequenza).

\begin{figure}[!htb]
	\centering
	\includegraphics[scale=0.53]{images/threading-mappa.png}
	\caption{. Fonte\cite{joseph2014local}}
	\label{fig:fold-recognition}
\end{figure}

an align-
ment between the sequence and the template structure
(threading).

Identifying templates, i.e., proteins of known structure whose fold should be similar to
that of the target protein. This can be done by threading, i.e., by aligning the sequence
of the target protein with the structure of each fold in the PDB on the basis of shared
sequence properties, and then choosing the best match. The sequence properties can
be grouped into two types:

Come si è già detto la struttura delle proteine è maggiormente conservata rispetto alle sequenze. Questo significa che proteine con differenti sequenze possono ancora formare strutture simili grazie a certe proprietà condivise codificate nelle loro sequenze. Se queste proprietà o ricorrenze statistiche potessero essere identificare, sarebbe possibile predire la struttura di una nuova proteina basandosi su un template che condivide le stesse proprietà, anche se loro sequenze sono diverse. Questa è l'idea su cui si basano i metodi di \textit{fold recognition}. 
In altri termini, in questo approccio si cerca una proteina con struttura conosciuta (nel PDB) che abbia alcune proprietà nella sequenza o tendenze condivise con la proteina target: le due probabilmente hanno un ripiegamento simile o motivi strutturali simili.
}

\subsection{\textit{ab initio}} \label{sec:ab-initio}
{
Il metodo più lineare e a prima vista ovvio per predire la struttura nativa di proteine è seguire la natura, simulando accuratamente come le forze fisiche guidino la proteina a ripiegarsi e usare questa simulazione per riprodurre il processo di ripiegamento su proteine con strutture sconosciute. \textit{Ab initio}, termine latino, significa infatti "dall'inizio". Questo approccio si basa sul postulato di Anfinsen.

\par Il primo problema che sorge è superare il paradosso di Levinthal. Per farlo si assume un profilo energetico a imbuto del ripiegamento, ovvero la premessa termodinamica che la forma nativa di una proteina sia lo stato in cui risulta avere più bassa energia libera, o più precisamente (richiamando la definizione di struttura nativa data nella sez. \ref{sec:energetica}) quella conformazione avente minore energia libera tale da mantenere il livello di dinamicità richiesto alla proteina per svolgere la sua funzione biologica.

\par Le predizioni nell'approccio \textit{ab initio} sono pertanto \textit{energy-based}, ovvero guidate dall'idea di minimizzare l'energia. Si può vedere il PSP secondo l'approccio \textit{ab initio} come un problema di ottimizzazione dove una funzione di energia gioca il ruolo di \textit{euristica} cercando di raggiungere il minimo globale di energia all'interno dello spazio di ricerca. In quanto \textit{energy-based} usano solo informazioni sul tipo di atomi nel sistema, le loro posizioni relative nello spazio tridimensionale e le loro interazioni con gli altri atomi. Viene poi calcolato l'intero contenuto di energia del sistema e le forze agenti su ogni atomo. 

\par L'energia totale di un sistema (\textit{free energy}) può essere decomposta in varie componenti: cinetica, potenziale, termica ecc. È l'energia libera che determina la stabilità del sistema. Come si vedrà sotto, nell'approccio fisico non viene calcolata tutta l'energia libera ma viene approssimata con una sua parte per motivi di complessità.

\par Sebbene vi siano differenti metodi in questo approccio, tutti condividono due caratteristiche di base:
\begin{itemize}
	\item calcolano il contenuto di energia del sistema in una singola configurazione
	\item campionano numerose configurazioni e ne trovano una con la minor energia libera
\end{itemize}

Per \textit{configurazione} si intende la disposizione complessiva di tutti gli atomi di tutti i componenti del sistema (proteina, solvente, ioni, membrana ecc.) mentre la posizione collettiva dei soli atomi della proteina viene chiamata \textit{conformazione}.

\subsubsection{Molecular mechanics \& dynamics}

\par Per descrivere in maniera affidabile tutte le forze fisiche operanti sul sistema tra i differenti atomi bisognerebbe descriverne la distribuzione di tutti gli elettroni, il che richiede però calcoli di meccanica quantistica (QM). Le forze, in un sistema molecolare, risultano dalla distribuzione spaziale degli elettroni attorno agli atomi. Sfortunatamente questi calcoli sono computazionalmente molto costosi e una rigorosa caratterizzazione di un sistema macromolecolare, con milioni di atomi, è al momento insostenibile. Calcoli di QM su una singola conformazione di una piccola proteina possono richiedere mesi, tempi troppo lunghi se si ha l'obiettivo di provare tante configurazioni per sceglierne una finale. 

\subsubsection{Molecular mechanics}

\par Per le ragioni sopra elencate gli scienziati spesso investigano sistemi macromolecolari usando approssimazioni delle reali forze in essi. Il campo da cui i calcoli per le approssimazioni sono presi è chiamato \textit{molecular mechanics} (MM), poiché approssima sistemi molecolari usando espressioni prese dalla meccanica newtoniana classica:
\begin{itemize}
	\item il contenuto di energia è descritto usando un \textit{campo di forza} nel quale gli atomi e i legami covalenti sono trattati come palline e molle
	\item le descrizioni che richiederebbero calcoli di QM vengono ignorate
	\item le rappresentazioni sono \textit{esplicite}: prendono in considerazione tutti gli atomi (vedi fig. \ref{fig:descrizione-esplicita-mm})
\end{itemize}

Il campo di forza sopra accennato descrive l'energia potenziale del sistema. Da notare che l'energia potenziale (intesa come entalpia) è solo una delle due componenti dell'energia libera, vedi sez. \ref{sec:energetica}).

\par Un campo di forza è un'energia di posizione: l'energia di un oggetto in una specifica posizione all'interno di un campo (gravitazionale, elettrico, magnetico ecc.). Nelle molecole l'energia potenziale è la somma di tutti gli effetti dei campi elettrici atomici\footnote{Gli atomi possiedono, in base alla loro eventuale carica, campi elettrici che influenzano gli altri atomi.} in una determinata posizione. Si può approssimare l'energia potenziale all'energia risultante da tutti i legami covalenti e le interazioni non covalenti, escluse quelle non polari \footnote{Un esempio di interazione non polare è l'effetto idrofobico. Vengono escluse poiché coinvolgono principalmente cambiamenti di entropia nel solvente.}, in una singola configurazione del sistema. In genere i campi di forza non sono una singola funzione ma una somma di più termini, ognuno corrispondente a un differente tipo di legame chimico o interazione, un esempio:

\[ U_{tot}=U_{cov}+U_{elst}+U_{vdw} \]

dove per $U_{tot}$ si intende l'energia potenziale totale, per $U_{cov}$ l'energia potenziale dei legami covalenti, per $U_{elst}$ quella delle interazioni elettrostatiche e per $U_{vdw}$ quella delle interazioni di van der Waals.

\par L'energia potenziale delle interazioni elettrostatiche può essere calcolata con la legge di Coulomb, mentre quella delle interazioni di van der Waals tramite l'equazione di Lennard-Jones. Nelle simulazioni di sistemi biologici vengono usati: CHARMM, AMBER, GROMACS, ecc.

\begin{figure}[!htb]
	\centering
	\includegraphics[scale=0.4]{images/esplicita-mm.png}
	\caption{Descrizioni esplicite nei calcoli di MM. (A) una piccola proteina immersa in un solvente composto da molecole d'acqua e ioni (Na$^{+}$, Cl$^{-}$). La proteina è rappresentata come sfere di atomi, l'acqua come bastoncini e gli ioni come piccole sfere magenta e gialle. (B) una proteina trasportatrice in un doppio strato lipidico, circondato da ambiente acquoso. La proteina e le teste dei lipidi sono rappresentate in modo space-fill, mentre l'acqua e le code dei lipidi con rappresentazione wire-frame. Fonte \cite{kessel_ben-tal_2018}}
	\label{fig:descrizione-esplicita-mm}
\end{figure}

\par La descrizione approssimata fornita dal campo di forza permette di calcolare l'energia potenziale di molti sistemi macromolecolari in meno di un secondo. \\

\par Una variante del MM è la \textit{QM/MM} nella quale i calcoli di QM sono indirizzati solamente su una piccola parte della proteina che contiene residui funzionali importanti. Le altre regioni sono soggette invece a MM, con calcoli molto più veloci \footnote{Questo approccio è stato introdotto da Warshel, Levitt e Karplus.}

\subsubsection{Spazio configurazionale}

\par Assumendo l'accuratezza del campo di forza, il calcolo dell'energia potenziale di un sistema consente di determinare (parte del)la stabilità di una configurazione. L'idea iniziale potrebbe essere quella di considerare tutte le possibili locazioni atomiche del sistema, calcolare l'energia potenziale in ogni caso e scegliere quella con la minor energia. Come si può facilmente intuire ciò risulta essere un procedimento troppo oneroso, in quanto si devono considerare anche gli atomi del solvente (ed eventuali ligandi o cofattori). Anche il solo numero delle possibili configurazioni atomiche è difficile da calcolare.

\par Per superare questo problema vengono usate tecniche per ridurre lo spazio di ricerca nello spazio configurazionale. Ci sono vari metodi di ricerca, ad esempio: \textit{systematic search }(grid search basata su dettagli geometrici), \textit{model-building model }(usa frammenti molecolari), \textit{random approach }(movimenti random sul piano cartesiano da una configurazione iniziale), \textit{distance geometry} (usa una matrice di distanze atomiche), \textit{Monte Carlo method} (modifiche random e accettazione probabilistica di configurazioni a livelli energetici maggiori)\supercite{ROY2015151}.

\par Il metodo più semplice è chiamato \textit{energy minimization}:

\begin{enumerate}
	\item si parte da una configurazione arbitraria
	\item si calcola l'energia potenziale. Viene derivato questo valore su differenti posizioni nel sistema in modo da calcolare le forze agenti su ogni atomo dalla rimanente parte del sistema
	\item un piccolo cambiamento è introdotto nella posizione di ogni atomo, in risposta alle forze applicate su ognuno di essi dal resto del sistema (in accordo a quanto calcolato nel precedente step)
	\item se la nuova configurazione ha un'energia minore viene adottata
	\item altrimenti questa viene scartata e viene creata una nuova configurazione
	\item si ritorna allo step 3 finché non si trovano più configurazioni con minor energia
\end{enumerate}

Il metodo passa da una configurazione all'altra scendendo con il gradiente della superficie dell'energia potenziale finché non converge in un \textit{punto di minimo locale}. Tutte le procedure di \textit{energy minimization }tendono a rimanere bloccate in un minimo locale di energia non riuscendo spesso a raggiungere il minimo globale a causa di \textit{barriere energetiche} da scavalcare per raggiungere una configurazione con energia minore (vedi fig. \ref{fig:energy-minimization} e \ref{fig:imbuto}).

\begin{figure}[!htb]
	\centering
	\includegraphics[scale=1]{images/energy-minimzation.jpg}
	\caption{Differenti fasi energetiche di una molecola durante la sua minimizzazione energetica. Fonte\cite{ROY2015151}}
	\label{fig:energy-minimization}
\end{figure}

\subsubsection{Molecular dynamics}

È possibile spingere l'algoritmo di minimizzazione energetica fuori da punti di minimo locale fornendo energia extra, ad esempio innalzando la temperatura del sistema (ovvero aggiungendo calore virtuale). L'energia aggiunta consente agli atomi del sistema di incrementare i loro movimenti e nuove configurazioni fuori dalle barriere energetiche vengono create. Questo metodo è chiamato \textit{Molecular dynamics} (MD) e si focalizza sui movimenti dipendenti dal tempo degli atomi nel sistema. I calcoli sono realizzati in accordo alla meccanica classica. 

\par Agli atomi viene assegnata una velocità iniziale (proporzionale alla temperatura) e continuano a muoversi nello spazio secondo i corrispondenti cambiamenti nell'energia potenziale del sistema. Il movimento di ogni atomo nel sistema è calcolato in base alla sua energia in quel dato momento.

\par Le simulazione di MD sono eseguite in cicli ripetitivi di \textit{riscaldamento} e \textit{raffreddamento} (metodo conosciuto nel mondo informatico come \textit{simulated annealing}, in riferimento al processo di tempra dei metalli). Nella fase di riscaldamento vengono superate le barriere energetiche mentre la fase di raffreddamento (seguita dall'\textit{energy minimization}) consente al sistema di rilassarsi in configurazioni con minor energia.

\par Un metodo comune per rendere la ricerca con MD più efficiente è di spezzarla in due fasi:
\begin{itemize}
	\item ricerca a bassa risoluzione per trovare una collezione di strutture con interazioni non polari (basato sulla nozione che il nucleo delle proteine globulari sia idrofobico)
	\item ricerca ad alta risoluzione fra le strutture selezionate nel primo step
\end{itemize}

\subsubsection{Limiti dell'approccio fisico e parziali soluzioni}
Sebbene simulare il ripiegamento proteico seguendo la meccanica classica possa apparire un approccio attraente, questo è pratico solo per piccole proteine e usando alte risorse computazionali: lo spazio di ricerca è enorme e il problema è computazionalmente intrattabile in modo deterministico (è NP-hard)\supercite{abbass2020enhancing}.

\par I metodi di MM/MD trovano difficilmente impiego in processi biologici rilevanti come il protein folding. Alcuni problemi riguardano l'approssimazione in sé del campo di forza, la sua accuratezza e i possibili doppi conteggi delle forze in gioco (es. interazioni ioniche e legami idrogeno calcolate in due espressioni differenti). 

\par Un altro problema, sempre nell'approssimazione dell'energia libera con campi di forza, è che forniscono sì l'energia potenziale ma non l'entropia. L'unico modo per stimare l'entropia e l'energia libera dai calcoli per l'energia potenziale è eseguire questi calcoli su tutte le possibili configurazioni del sistema e poi integrarli. Il problema risiede quindi nell'impossibilità di compiere la totalità di questi calcoli a causa delle rappresentazioni esplicite usate nelle simulazioni di MD. In particolare è difficile considerare tutte le configurazioni del solvente acquoso. Ciò che si sta calcolando non è l'energia libera ma un \textit{potenziale di forze medie} (PMF). In conclusione le simulazioni di MD non sono consigliate per descrivere gli effetti dei solventi.

\subsubsection{Mean field approach}
Per ovviare parzialmente al problema delle rappresentazioni esplicite è possibile descrivere \textit{implicitamente} parti del sistema che vengono descritte da una proprietà media, per questa ragione tale approccio è chiamato \textit{mean field}. Un esempio è la descrizione del solvente, la parte "meno" interessante in genere, come una massa omogenea descritta dalla sua \textit{dielettricità} \footnote{Proprietà di un mezzo non conduttore di essere sede di un campo elettrostatico.}, conosciuto anche come approccio \textit{continuum-solvent}, vedi fig. \ref{fig:mean-field}.

\begin{figure}[!htb]
	\centering
	\includegraphics[scale=0.4]{images/mean-field.png}
	\caption{Descrizione con approccio mean-field di un sistema, il solvente è descritto implicitamente mentre la proteina esplicitamente. $\epsilon$ indica la dielettricità. (A) proteina in un solvente acquoso altamente dielettrico. (B) proteina di membrana in un ambiente eterogeneo. Il solvente acquoso è altamente dielettrico mentre la lastra semi-trasparente gialla, che rappresenta la regione biologica di doppio strato lipidico, è poco dielettrica. Fonte\cite{kessel_ben-tal_2018}}
	\label{fig:mean-field}
\end{figure}

Ovviamente, essendo una forte approssimazione, alcuni aspetti del sistema reale sono ignorati, come le interazioni fra gli atomi delle proteine e le molecole d'acqua. Tale problema si esacerba quando il solvente è una membrana. \\

\par Un altro compromesso è l'approccio \textit{mixed force fields} che combina calcoli espliciti sulla proteina e calcoli impliciti sul solvente. Viene usata l'equazione di Poisson-Boltzman (PBE) per calcolare accuratamente l'energia libera elettrostatica, legando così l'effetto polarizzante delle cariche con il loro ambiente. Essendo però un calcolo oneroso viene in genere risolta l'equazione generalizzata di Born (GB). A partire da queste due equazioni, che calcolano la componente elettrostatica dell'energia libera, è possibile calcolare l'intera \textit{free energy}, in modo abbastanza accurato, con calcoli che si rifanno alla surface area (SA, vedi parte finale della sez. \ref{sec:termodinamica-forze-idrofobiche}). 

\par Questi metodi sono chiamati \textit{PBSA} e \textit{GBSA} rispettivamente, e come si è visto permettono un calcolo più preciso dell'energia libera. Questi possono a loro volta essere combinati con la MM per rappresentare anche le interazioni del sistema (\textit{MM-PBSA, MM-GBSA}) e sono oggi largamente utilizzati.

\par Un altro limite computazionale è il lasso temporale che si riesce a coprire. La maggior parte delle proteine si ripiega in microsecondi mentre le simulazioni riescono a coprire tempi che vanno dai pico ai nanosecondi. Grazie ad avanzamenti nelle risorse informatiche sono stati fatti dei passi avanti da questo punto di vista. Un caso interessante è \textit{Anton}, un supercomputer progettato specificatamente per ottimizzare simulazioni di MD capace di coprire 85$\mu s$ al giorno per un sistema molecolare di 23.000 atomi (180 volte più veloce di qualsiasi computer general-purpose). Altri progressi sono dovuti alla computazione parallela e alla computazione accelerata dalla GPU. Il calcolo distribuito (\textit{grid computing}), ovvero una larga rete di computer personali dedicati volontariamente al completamento di processi, ha permesso alla rete \textit{Folding@Home} (170.000 computer) di simulare l'intero processo di ripiegamento della proteina di legame dell'acetil coenzima A, composta da 86 residui e che richiede 10 millisecondi per ripiegarsi. Un'altra rete distribuita di calcolo è \textit{Rosetta@Home}, con 86.000 nodi e finalizzata al PSP. \\

\subsubsection{Conclusioni sull'approccio \textit{ab initio}}
I metodi \textit{ab initio} non sono attualmente in grado di predire la struttura della maggior parte delle proteine sulla sola base della loro sequenza. Ma sono molto abili nel farlo quando il punto di partenza della predizione è una struttura vicino a quella nativa. Questi metodi sono infatti ampiamente usati per raffinare le strutture grezze ottenute dalle determinazioni sperimentali (vedi sez. \ref{sec:experimentally-guided-prediction}). 

\par Il loro successo dipende ampiamente dall'accuratezza della funzione di energia, dall'efficienza dell'algoritmo di ricerca nello spazio conformazionale e dall'abilità di discernere strutture native da "esche" energeticamente intrappolate.
Nonostante le loro limitazioni i metodi \textit{ab initio} sono di grande interesse perché sono gli unici, in principio, derivare la vera struttura nativa delle proteine e possono quindi fornire intuizioni importanti per il protein folding problem. Hanno infatti fornito informazioni importanti sulla dinamica delle proteine e sono utilizzati anche nel \textit{protein engineering} e nel \textit{drug discovery}. 

}

\subsubsection{Valutazione \textit{knowledge-based}}
È anche possibile rimpiazzare il campo di forza con una funzione di valutazione \textit{knowledge-based}. Spesso queste funzioni sono composte di espressioni relative alla tendenza statistica di gruppi chimici, amminoacidi o atomi di interagire fra di loro. L'affidabilità della funzione di valutazione dipende dal database su cui si basa. 



\subsection{\textit{fragment-based}}
%-abbass 

\par Un possibile metodo di azione è la \textit{frammentazione delle proteine}, nel quale i calcoli sono realizzati su piccoli frammenti della proteina e poi integrati. Un esempio è il software QUARK. L'approccio \textit{fragment-based} è usato nei metodi che combinano predizioni \textit{template-based} con assemblaggi \textit{ab initio}.

%-- parla dell'assemblaggio dei frammenti. È ab initio?

\subsection{\textit{loop modeling}} \label{sec: loop-modeling}
{
I loop sono regioni della struttura proteica con ruoli spesso cruciali (interazioni con altre proteine, siti di legame con molecole ecc.). Allo stesso tempo sono molto variabili nella loro sequenza e struttura rispetto alle altre regioni. Si trovano generalmente sulla superficie delle proteine e le loro strutture sono note per essere difficili da predire.

\begin{figure}[!htb]
	\minipage{0.48\textwidth}
	\centering
	\includegraphics[scale= 1]{images/fread.png}
	\caption{Disegno di un loop in celeste. Fonte \cite{FREAD}}
	\label{fig:loop-example}
	\endminipage\hfill
	\minipage{0.5\textwidth}
	\centering
	\includegraphics[scale=0.3]{images/loops.png}
	\caption{Omega loop. Sono spesso coinvolti nel riconoscimento molecolare e in funzioni regolatrici. Fonte: \cite{Papaleo2016TheRO}}
	\label{fig:omega-loops}
	\endminipage\hfill
\end{figure}

Il loop modeling non si applica solamente alla fase di raffinamento della modellazione per omologia della predizione di strutture proteiche. È importante anche nella predizione di frammenti mancanti nelle strutture determinate sperimentalmente. È stato stimato che in più della metà delle struttura depositate nel PDB ci siano segmenti mancanti, spesso loop\supercite{karami2018dareus}.

\par Problemi comuni nel loop modeling sono: decidere quale regione del modello sarà un loop; trovare il corretto allineamento di regioni di ancoraggio; la modellazione in sé del loop; le conformazioni di loop multipli, ecc.

\par Nella modellazione per omologia (nel PSP) si registrano spesso grandi deviazioni dai template omologhi: la modellazione dei loop rimane un problema aperto nella modellazione per omologia della struttura delle proteine\supercite{karamiLoop}. Le principali strategie per il loop modeling sono le stesse di quelle per la predizione dell'intera struttura:
\begin{itemize}
	\item \textit{data-based} (o knowledge-based), basati sull'assunzione di similarità sequenza-struttura, ovvero che loop con sequenze simili hanno anche conformazioni simili
	\item \textit{ab initio}, in cui viene esplorato lo spazio conformazionale
	\item approccio ibrido
\end{itemize}

\par Un protocollo comune per la modellazione, partendo da un modello della proteina senza loop e la sequenza del loop, è quello mostrato in figura \ref{fig:loop-modeling-approaches}, con i seguenti step:
\begin{itemize}
	\item generazione di tutti i possibili stati del loop (ab initio, data-based o ibrido)
	\item valutazione e raggruppamento
	\item raffinamento
\end{itemize}

\begin{figure}[!htb]
	\centering
	\includegraphics[scale=1]{images/loop-modeling-approaches.jpg}
	\caption{Approcci al loop modeling. Workflow schematico di un protocollo prototipo per la modellazione dei loop. Fonte\cite{barozet2021current}}
	\label{fig:loop-modeling-approaches}
\end{figure}

La ricerca su database è efficiente per famiglie specifiche ma i loop più lunghi di 4 residui devono essere comunque ottimizzati. I residui ai fianchi della regione del loop sono chiamati residui di \textit{ancoràggio} e sono utilizzati per effettuare la ricerca nei database. ArchPRED ad esempio considera le strutture secondarie ai fianchi del loop mancante, il loro orientamento relativo e il numero di residui mancanti per identificare conformazioni del loop candidate. È possibile usare una funzione di valutazione basata sull'energia per valutare la modellazione dei loop, ad esempio basata sulla stereochimica (come in CHARMM). 

\par Molti dei metodi raggiungono ottimi risultati per la predizione dei loop su strutture sperimentali in ambienti esatti (ovvero strutture cristallizzate a cui mancano le regioni dei loop). Ma nei modelli per omologia non si è ancora riusciti a raggiungere buoni risultati\supercite{karami2018dareus}. Metodi allo stato dell'arte sono in grade di predire conformazioni stabili di loop relativamente corti (fino a 12 residui)\supercite{barozet2021current}.
Seppur con le loro limitazioni, gli approcci correnti sono pronti per essere usati in problemi impegnativi come il loop design in enzimi e anticorpi.

\par I metodi \textit{ab initio} sono dipendenti dalle tecniche di ottimizzazione dell'energia e per questa ragione risultano essere lenti. Metodi \textit{ab initio} per il completamento della struttura cristallizzata allo stato dell'arte sono Rosetta-NGK e GalaxyLoop-PS2. CODA è un esempio di metodo ibrido che combina i due approcci nel loop modeling, così come Sphinx il quale prima esegue una ricerca data-based per trovare frammenti più corti del loop di interesse in modo da ottenere informazioni strutturali, successivamente applica metodi \textit{ab initio} per generare frammenti della corretta lunghezza. \\

\par Pochi metodi sono disponibili come web servers e quindi utilizzabili anche dai non esperti: GalaxyLoopPS2, LoopIng, Sphnix e DaReUS-Loop. Metodi locali sono invece MODELLER, Loopy, OSCAR-loop, Rosetta-NGK, LEAP e M-DISGro. Sono disponibili anche dei tool per la modellazione di loop specifici per gli anticorpi, come quelli offerti da SAbPred\footnote{Collezione di tool sviluppati da Oxford Protein Informatics Group (OPIG).}: Sphinx e FREAD (knowledge-based) che effettua una ricerca su database tenendo in considerazione i vincoli spaziali dei residui di ancoràggio.

\par Possono essere utilizzati anche simulazioni Monte Carlo e MD per investigare proprietà termodinamiche e cinetiche dei loop. Per quanto riguarda l'utilizzo di Deep Learning o Machine Learning per la modellazione dei loop, resta da dimostrare la capacità dei metodi ML/DL di generare modelli significativi di loop flessibili, nonostante in altre aree della bioinformatica strutturale si sia rivelato uno scenario con grande potenziale\supercite{barozet2021current}.

\par Uno dei metodi più recenti e con migliori risultati è DaReUS-Loop, un approccio  \textit{data-based} che identifica loop candidati estraendoli dal completo insieme delle strutture conosciute del PDB. Il filtraggio dei candidati si basa su confronti di conformazioni locali profilo-profilo insieme a una valutazione fisico-chimica. Applicato ai dataset del CASP11 e CASP12 mostra significativi progressi nell'accuratezza della predizione dei loop e propone una misura di confidenza che correla bene con l'accuratezza effettiva dei loop. I loro autori mostrano anche che oltre il 50\% dei modelli ben riusciti sono derivati da proteine non correlate: ciò suggerisce che frammenti di proteine, sotto simili vincoli, tendono ad adottare simili strutture (oltre la mera omologia)\supercite{karami2018dareus}. 

\begin{figure}[!htb]
	\centering
	\includegraphics[scale=0.45]{images/dareus.png}
	\caption{DaReUS-Loop workflow. Da notare che dal 2019\supercite{karamiLoop} nel processo di costruzione del modello non è più usato MODELLER (non free) ma GROMACS. Fonte\cite{karami2018dareus}}
	\label{fig:dareus-workflow}
\end{figure}

I principali step del metodo sono mostrati in figura \ref{fig:dareus-workflow} e sono:
\begin{itemize}
	\item ricerca dei candidati del loop
	\item filtraggio dei candidati
	\item costruzione del modello
	\item model selection
\end{itemize}

Nell'ultimo step sono utilizzate 2 misure per valutare i modelli e vengono ritornati in output come predizioni finali i 5 migliori modelli per ogni metrica.

\begin{figure}[!htb]
	\centering
	\includegraphics[scale=0.63]{images/dareus-confronto.png}
	\caption{Esempi di predizione di un loop lungo (15 residui) della proteina target T0807 del CASP11 a confronto. Blu=DaReUS-Loop, verde=Rosetta NGK, arancione=GalaxyLoop-PS2, rosso=struttura cristallizzata. La RMSD di ogni loop predetto rispetto al loop nativo è riportata di seguito. DaReUS-Loop: 1.3\angstrom, NGK: 3\angstrom, PS2: 2.9\angstrom. Nella colonna a destra sono riportate le catene laterali della struttura nativa e di quella predetta da DaReUS-Loop. Fonte\cite{karami2018dareus}}
	\label{fig:}
\end{figure}

}

\section{Case Study: \textit{TASSER}}
-tasser papers
-kessel
-abbass
-pearce 2021


\section{Studio sperimentale delle proteine} \label{sec:experimentally-guided-prediction}
{

\subsubsection{Come vengono studiate le proteine}
Comprendere come una particolare proteina funzioni richiede dettagli strutturali e analisi biochimiche: entrambe hanno bisogno di una grande quantità di proteine pure. Isolare però un singolo tipo di proteina dalle migliaia presenti in una cellula non è un compito semplice. Per molti anni le proteine sono state purificate direttamente dai tessuti nei quali esse erano abbondanti. Ciò comporta, oltre alla necessità fisica di procurarsi i tessuti, dover riconoscere le proteine da studiare. Queste procedure richiedono settimane e forniscono solamente pochi milligrammi di proteina pura. Oggi le proteine sono generalmente isolate da cellule coltivate in laboratorio e spesso queste sono modificate geneticamente al fine di produrre grandi quantità di una data proteina. \\

\par Il primo step in una procedura di purificazione consiste nel rompere i tessuti e le cellule in modo da far loro rilasciare il contenuto (\textit{omogenizzazione}), chiamato \textit{estratto} (o omogenato cellulare). Ci sono vari meccanismi, come mostrato in fig. \ref{fig:break-cells}. 


\begin{figure}[!htb]
	\centering
	\includegraphics[scale=0.5]{images/break-cells.png}
	\caption{Omogenizzazione attraverso 4 diversi meccanismi, ognuno dei quali finalizzato a rompere le membrane plasmatiche. Fonte\cite{alberts2018essential}}
	\label{fig:break-cells}
\end{figure}

Segue poi una procedura di frazionamento iniziale, tipicamente per \textit{centrifugazione}, per separare l'omogenato in differenti parti. Per raggruppare poi le classi di molecole di interesse si possono usare tecniche di \textit{cromatografia} o \textit{elettroforesi}. Una forma efficiente di cromatografia è quella per \textit{affinità} nella quale vengono utilizzati degli anticorpi specifici per la proteina di interesse. Nel secondo metodo un insieme di proteine viene immerso in un gel e soggetto ad un campo elettrico: le proteine migreranno nel gel a differenti velocità, a seconda del loro peso molecolare e della loro carica. Proteine simili migreranno a velocità simili, pertanto potranno essere visualizzate bande o punti di aggregazione di proteine. \\

\par Una volta separate le proteine di interesse è possibile studiarne la struttura. Per quanto riguarda la struttura primaria, la prima proteina sequenziata è stata l'\textit{insulina} nel 1955, attraverso una procedura chimica diretta. La proteina veniva prima scomposta da una determinata proteasi e successivamente ogni amminoacido, in ogni frammento, veniva determinato sperimentalmente. Un metodo molto più veloce è la \textit{spettrometria di massa}, almeno per organismi di cui sia stato completamente sequenziato il genoma. Questa tecnica determina l'esatta massa di ogni frammento in una proteina purificata, consentendo l'identificazione della proteina nei database di sequenze genomiche.

\par Per eseguire la \textit{spettrometria di massa} la proteina viene "digerita" dall'enzima \textit{tripsina} e frammentata in peptidi o singoli amminoacidi. Questi vengono scaldati con un laser, il che li renderà carichi e li farà evaporare. Viene poi usato un potente campo elettrico per far volare gli ioni peptidici verso un misuratore: il tempo che impiegano per arrivare è legato alla loro massa e carica (più massa = più lenti, più carichi = più veloci). L'insieme delle precise masse dei frammenti serve come "impronta digitale" (\textit{peptide mass fingerprinting}, PMF) che verrà usata per identificare la proteina codificata dall'organismo (e i suoi geni corrispondenti) dai database il cui profilo (massa teorica) corrisponde a questa impronta peptidica.

\subsubsection{Determinazione sperimentale delle strutture}

\par La determinazione sperimentale della struttura delle proteine ha vissuto dei progressi significativi col passare degli anni ed è di grande importanza per i metodi computazionali di PSP, consentendo ai metodi \textit{data-based} di affinare le loro predizioni.

\begin{figure}[!htb]
	\centering
	\includegraphics[scale=0.42]{images/metodi-sper2.png}
	\caption{Differenti livelli di informazione ottenuta lungo il percorso della determinazione della struttura di macromolecole. AFM: atomic force microscopy; Cryo-ET: cryo-electron tomography; EM: electron microscopy; FRET: fluorescence resonance energy-transfer; NMR: nuclear magnetic resonance; SAXS: small-angle X-ray scattering; SPR: surface plasmon resonance. Fonte\cite{sharon2011far}}
	\label{fig:metodi-sper2}
\end{figure}

I metodi per la determinazione sperimentale della struttura delle proteine possono essere divisi in due gruppi:
\begin{itemize}
	\item metodi \textit{indiretti}, ovvero l'osservazione della proteina è possibile solo dopo sofisticate manipolazioni dei dati ottenuti:
	\begin{itemize}
		\item metodi per \textit{diffrazione}, che si basano sulla diffrazione o sulla dispersione di particelle subatomiche o onde elettromagnetiche da parte della proteina
		\item metodi per \textit{spettroscopia}, i quali si affidano all'eccitazione e susseguente rilassamento degli atomi della proteina in risposta alla radiazione elettromagnetica
	\end{itemize}
	\item metodi \textit{diretti}, in cui l'osservazione della proteina è diretta; al momento è possibile con la microscopia crioelettronica (Cryo-EM)
\end{itemize}

Vengono utilizzate principalmente 3 tecniche per generare informazioni strutturali sulle proteine a risoluzione atomica: \textit{X-ray crystallography, nuclear magnetic resonance (NMR) spectroscopy} ed \textit{electron microscopy}\footnote{La traduzione italiana sarebbe: cristallografia a raggi-X, spettroscopia a risonanza magnetica nucleare e microscopia elettronica.}. Una volta ottenute proteine pure, queste devono essere o cristallizzate (cristallografia a raggi X), o piazzate in speciali solventi (spettroscopia NMR) o congelate (microscopia elettronica). \\

\par Il metodo di diffrazione più comune, e più anziano, è la \textit{cristallografia a raggi X}. Produce strutture tridimensionali con la più alta risoluzione ma presenta alcune gravi carenze, la più significativa è la necessità di cristallizzare la proteina studiata. La cristallizzazione è un processo lungo e difficile e produce anche strutture di proteine al di fuori del loro ambiente. Queste strutture sono prive di qualsiasi proprietà dinamica e (raramente) possono risultare deformate. Pertanto non tutte le parti di una proteina (ad es. le parti più mobili) possono essere viste con strutture a raggi-X e di conseguenza queste regioni possono essere ignorate o aperte a interpretazioni.

\begin{figure}[!htb]
	\centering
	\includegraphics[scale=0.4]{images/cristallografia.png}
	\caption{Processo di determinazione della struttura dell'enzima RuBisCO tramite cristallografia a raggi-X. Fonte\cite{alberts2018essential}}
	\label{fig:cristallografia}
\end{figure}

\par I piccoli cristalli di proteine misurano meno di 1mm e sono esposti ad un'intensa esposizione ai raggi-X (i quali hanno una lunghezza d'onda pari a quella di un atomo, $1-2\angstrom$). I raggi X sono dispersi o diffratti dagli atomi proteici nel cristallo. Il modello di diffrazione che ne deriva appare tipicamente come decine di migliaia di minuscoli punti disposti in complessi schemi circolari. Questi modelli di diffrazione sono registrati su una fotocamera a raggi X digitale. La posizione dei punti di diffrazione (insieme ad altre informazioni), sono effettivamente sufficienti per compiere una computazione della mappa della densità elettronica di tutti gli atomi pesanti (carbonio, azoto, ossigeno, zolfo) nella proteina di diffrazione. Da questa mappa, i cristallografi determinano le coordinate x,y,z di tutti gli atomi usando la sequenza nota della proteina. Si noti che, nella cristallografia a raggi X, anche se il pattern di diffrazione deriva da milioni di proteine contenute nel cristallo, il risultato è una struttura per una singola proteina "media".

\par La prima struttura determinata con questa tecnica risale al 1958; risolvere una struttura negli anni '70 richiedeva anche 6-7 anni mentre oggi è a volte possibile in soli 6-7 giorni. Il 90\% delle strutture oggi determinate deriva dalla cristallografia a raggi-X\supercite{baxevanis2020bioinformatics}. \\

\par Altri metodi per diffrazione sono:  \textit{small-angle X-ray scattering }(SAXS), \textit{neutron scattering},  \textit{electron crystallography}. Il metodo SAXS produce strutture a risoluzione inferiore rispetto alla cristallografia a raggi X. Tuttavia, le strutture SAXS sono molto utili nell'impostare vincoli posizionali per complessi proteici di grandi dimensioni, grazie ai quali è possibile "modellare" strutture a raggi X ad alta risoluzione.

\begin{figure}[!htb]
	\centering
	\includegraphics[scale=0.5]{images/metodi-sperimentali.png}
	\caption{Diagramma di flusso dei principali step usati la preparazione e soluzione sperimentale della struttura 3D delle proteine. Fonte\cite{baxevanis2020bioinformatics}}
	\label{fig:metodi-sper1}
\end{figure}

\par Il principale metodo spettroscopico utilizzato per la determinazione della struttura proteica è l'NMR. Questo metodo si basa sull'eccitazione dei nuclei atomici sotto un forte campo magnetico e sul loro successivo rilassamento. Viene misurato come i nuclei atomici (ad es. dell'idrogeno o dell'isotopo carbonio o azoto) assorbano la radiazioni; ciò permette di determinare quanto magnetismo nucleare è trasferito da un atomo all'altro. Questo approccio consente agli scienziati di trattare la proteina nel suo ambiente naturale (soluzione, membrana) e fornisce importanti informazioni sulla sua dinamica. L'NMR è un metodo più recente della cristallografia a raggi-X: la prima struttura determinata risale al 1983. Tuttavia questo metodo ha un limite superiore alla dimensioni di molecole studiabili (40kDa) e non può studiare proteine di membrana. L'EPR, un metodo simile, si basa sull'eccitazione e sul rilassamento degli elettroni attorno agli atomi della proteina e richiede la marcatura dei residui proteici con etichette paramagnetiche. \\

\subsubsection{Cryo-EM}

\par La microscopia crioelettronica (cryo-EM) è un metodo diretto: è possibile osservare direttamente macromolecole. È una versione avanzata di microscopio elettronico, il quale fu inventato negli anni '30. Questi microscopi usano raggi di elettroni piuttosto che di luce (la lunghezza d'onda degli elettroni è molto più corta di quella della luce). Durante la metà degli anni '70 è nata la cryo-EM: l'idea è stata quella di congelare i campioni per preservarne la struttura naturale e ridurre i danni causati dai raggi di elettroni.

\begin{figure}[!htb]
	\minipage{0.5\textwidth}
	\centering
	\includegraphics[scale=1.1]{images/cryo-em.jpg}
	\caption{Funzionamento schematico di un microscopio crio-elettronico. Fonte\cite{cryoEMbasics}}
	\label{fig:cryo-em-basics}
	\endminipage\hfill
	\minipage{0.5\textwidth}
	\centering
	\includegraphics[scale=0.136]{images/cryo-em-classi.png}
	\caption{(A) Micrografia rappresentativa del campione di streptavidina ottenuta tramite cryo-EM. La barra di scala rappresenta 20nm (B) Classi medie 2D rappresentative delle immagini della proteina. La barra di scala rappresenta 5nm. Fonte \cite{fan2019single}}
	\label{fig:cryo-em-classi}
	\endminipage\hfill
\end{figure}

Nella cryo-EM una goccia di acqua contenente pure proteine è inserita in una piccola griglia per EM immersa in una vasca di etano liquido a -180°C. I campioni di proteine vengono congelati velocemente (creando ghiaccio vitreo): questo assicura che le circostanti molecole d'acqua non abbiano tempo per formare cristalli di ghiaccio (che deformerebbero la forma della proteina). I campioni sono esaminati (ancora ghiacciati) da un microscopio a trasmissione elettronica (TEM) e sottoposti quindi a forti raggi di elettroni. Un rilevatore di elettroni cattura le "immagini" proiettate delle molecole e, data l'automazione odierna di simili meccanismi, vengono effettuate migliaia di micrografie per catturare più dettagli possibile delle molecole. Ogni micrografia conterrà centinaia di migliaia di molecole singole, ognuna orientata casualmente. 
\par Successivamente vi è lo step di image processing: le immagini proiettate vengono categorizzate in gruppi e allineate per poi essere sovrapposte in moda da calcolare un'immagine media per ogni gruppo. \\

\par La preparazione è quindi molto più semplice rispetto alla cristallografia a raggi-X e le strutture somigliano maggiormente a quelle viste nel normale ambiente acquoso della cellula. La cryo-EM è limitata nelle dimensioni: c'è un limite inferiore che di anno in anno si abbassa (ad. es 39kDa nel 2019\supercite{fan2019single}). Non sono un problema invece grandi proteine (anche maggiori di 100kDa). Oggi la cryo-EM può generare immagini 3D a risoluzione quasi atomica di virus e complesse macromolecole, come i ribosomi. È possibile utilizzare la cryo-EM anche per le proteine di membrana. 

\begin{figure}[!htb]
	\minipage{0.5\textwidth}
	\centering
	\includegraphics[scale=1.5]{images/cryo-ribosoma.jpg}
	\caption{Complesso di controllo qualità dei ribosomi, basato su dati di cryo-EM. Fonte: \cite{cryoRevolutionUCSF}}
	\label{fig:cryo-ribosoma}
	\endminipage\hfill
	\minipage{0.5\textwidth}
	\centering
	\includegraphics[scale=1.5]{images/cryo-zika.jpeg}
	\caption{La struttura del virus Zika ottenuta tramite cryo-EM. La macromolecola ha un peso di 190kDa ed è composta da 11.000 atomi. Risoluzione di 3.8\angstrom. Fonte \cite{cryoZika}}
	\label{fig:cryo-zika}
	\endminipage\hfill
\end{figure}

 È solo da pochi anni che il metodo ha fatto un grande passo in avanti, grazie ad avanzamenti nel rivelatore e nei software di \textit{image processing}. Nel 2017 è stato assegnato il premio Nobel per la chimica per aver contribuito a sviluppare tale metodologia\footnote{A Richard Henderson, Jacques Dubochet e Joachim Frank, "for developing cryo-electron microscopy for the high-resolution structure determination of biomolecules in solution".}.
Si sta considerando l'adozione della microscopia elettronica come di una rivoluzione nel campo della biologia strutturale\supercite{callaway2020revolutionary, bai2015cryo}. La crescita nel numero di strutture determinate è stata lenta inizialmente a causa della scarsa adozione del metodo, ma da quando si è vista la possibilità di produrre mappe dettagliate per macromolecole come i ribosomi la situazione è cambiata (vedi le figure \ref{fig:cryo-em-grafico} e \ref{fig:cryo-em-resolution}). Circa l'1\% delle strutture è stata determinata con questa tecnica ma i rapidi avanzamenti degli strumenti sia hardware che software potrebbero favorire la rivoluzione di cui si parla. 

\begin{figure}[!htb]
	\minipage{0.6\textwidth}
	\centering
	\includegraphics[scale=0.65]{images/cryo-em-grafico.png}
	\caption{Crescita delle mappe elettroniche rilasciate nell'EMDB. Fonte\cite{pakhrin2021deep}}
	\label{fig:cryo-em-grafico}
	\endminipage\hfill
	\minipage{0.4\textwidth}
	\centering
	\includegraphics[scale=0.35]{images/cryo-em-resolution.png}
	\caption{Grafico della risoluzione di strutture risolte tramite cryo-EM. Fonte \cite{callaway2020revolutionary}}
	\label{fig:cryo-em-resolution}
	\endminipage\hfill
\end{figure}
}

\section{Storia della comprensione delle proteine}
- alberts p.160
- psp-wiki
- levitt 2001, birth of structural biology \\ \\


L'approccio \textit{ab initio} è emerso negli anni '60 a partire dal campo della chimica computazionale. Nel 2013 il premio Nobel per la chimica è stato assegnato proprio a quegli scienziati che hanno contribuito sin da quegli anni al campo della biofisica molecolare computazionale (Warshel, Levitt, Karplus).

\par Le predizioni di strutture basate sui metodi \textit{ab initio} sono emerse nella metà degli anni '80, prima per piccoli peptidi e poi per polipeptidi. Il primo programma per calcolare l'energia potenziale nelle proteine è stato sviluppato nel 1969 da Lifson e Levitt\supercite{levitt1969refinement}.

\par La prima simulazione di MD su una proteina è stata realizzata nel 1977 da McCammon, Gelin e Karplus\supercite{mccammon1977dynamics}, studiando la dinamica di ripiegamento di una proteina di 58 amminoacidi rappresentata esplicitamente ma simulata nel vuoto. Questo studio seguì il lavoro pionieristico di Levitt e Warshel del 1975 (\textit{Computer simulation of protein folding}\supercite{levitt1975computer}) sulla stessa proteina che era però rappresentata in modo più semplicistico: ogni amminoacido era rappresentato da due sfere. 

\subsubsection{anni '90, database, omologia, progetto genoma}

Around the beginning of the 1990s, a new field in biology called ‘bioin-
formatics’ emerged, in which scientists sought to predict the characteristics of new pro-
teins on the basis of properties of their sequences

\subsubsection{CASP e AlphaFold}

A test of homology modeling efficiency carried out in 2007 has
shown that in single-domain proteins comprising 90 residues or fewer, the structures pre-
dicted by this method differed from their corresponding native structures by 2 to 6 \supercite{dill2008protein}.



Interestingly, in the first rounds of CASP, secondary structure prediction was a separate category. This
category was cancelled after the organizers noticed that the winners in this category used a somewhat circular
approach. They predicted the 3D structure and used their model structure to decipher the secondary structure
elements.

\clearpage
\chapter{La rivoluzione di AlphaFold}

Predire la struttura delle proteine è stato un importante problema di ricerca, aperto per più di 50 anni. Nonostante i vari progressi, nessun metodo è riuscito ad arrivare ad una precisione atomica, specialmente nel caso in cui non siano disponibili delle proteine omologhe. AlphaFold2 è il primo metodo computazionale che può regolarmente predire la struttura delle proteine con accuratezza atomica, anche in casi in cui nessuna struttura simile è conosciuta\supercite{jumper2021highly}.
AlphaFold2 (AF2) è la versione interamente ridisegnata del modello basato su reti neurali AlphaFold, entrambi sviluppati da DeepMind. 

\par Nel CASP14 (2020) AF2 viene dichiarato vincitore del "protein structure prediction problem" per la maggior parte delle proteine a singolo dominio, dimostrando accuratezza competitiva con le strutture sperimentali nella maggioranza dei casi e superando di gran lunga tutti gli altri metodi esistenti. Alla base di AF2 c'è un nuovo approccio basato sul Machine Learning, che unisce nel design dell'algoritmo di deep learning conoscenza fisica e biologica sulla struttura delle proteine, facendo leva sugli allineamenti multi-sequenza.

\par Quando è iniziata a circolare la notizia che AF2 avesse risolto il problema del PSP, si pensava che avesse raggiunto un GDT\_TS medio di 80 \supercite{moalqAF2} (intuitivamente significa che in media l'80\% della struttura delle proteine target è stato predetto). Predizioni casuali forniscono un GDT\_TS $\leq 20 \%$, predire la struttura grossolanamente è associato a un GDT\_TS di circa il 50\% mentre predire una topologia accurata porta con se un valore di circa il 70\%. Quando tutti i dettagli, comprese le conformazioni delle side-chain, sono corretti il GDT\_TS supera il 90\%. Alcuni, tra cui Mohammed AlQuraishi\footnote{Assistant Professor, Department of Systems Biology della Columbia University e principale investigatore dell'Alquraishi Laboratory. È stato anche uno dei peer reviewer del paper di AlphaFold2\supercite{moalqAF2}.} suggerivano che ci sarebbero voluti almeno altri 10 anni per arrivare ad un GDT di 85-90\supercite{moAlq}, ma AlphaFold ha riportato una mediana nel GDT\_TS pari a 92.4. Un valore che AlQuraishi definisce come uno degli avanzamenti scientifici più rapidi degli ultimi decenni.

\par Le strutture di AF2 hanno un'accuratezza riguardo la mediana della backbone\footnote{La notazione indica C$_{\alpha}$ mean root square deviation ad una copertura del 95\% dei residui. L'intervallo con 95\% di confidenza riportato in questo caso da AF2 corrisponde a 0.85-1.16 \angstrom.} di 0.96 RMSD$_{95}$, mentre il prossimo miglior metodo ha dimostrato un'accuratezza di 2.8 \angstrom. In figura \ref{fig:z-score} è possibile vedere il confronto dei vari metodi che hanno partecipato al CASP14 valutati secondo il Z-score\footnote{Lo Z-score è la differenza del valore di un campione rispetto alla media della popolazione, divisa per la deviazione standard; un valore alto rappresenta una grande deviazione dalla media ed è comunemente usato come procedura di rilevamento dei valori anomali.}. 

\begin{figure}[!htb]
	\centering
	\includegraphics[scale=0.45]{images/casp_res.png}
	\caption{Risultati del CASP14 in base al Z-score. AlphaFold (G427) è incredibilmente avanti rispetto al secondo gruppo (473, Baker). Fonte\cite{CaspRes}}
	\label{fig:z-score}
\end{figure}

Per quanto riguarda l'accuratezza non solo della backbone ma di tutti gli atomi, AlphaFold ha registrato un'accuratezza di 1.5 \angstrom (per fare un confronto, un atomo di carbonio è largo approssimativamente 1.4 \angstrom).

\begin{figure}[!htb]
	\minipage{0.5\textwidth}
	\centering
	\includegraphics[scale=0.3]{images/af-res.png}
	\caption{Performance di AF2 sul dataset del CASP14 (n=87 domini di proteine) rispetto agli altri migliori 15 metodi (su 146). Fonte\cite{jumper2021highly}}
	\label{fig:performanceAF2}
	\endminipage\hfill
	\minipage{0.48\textwidth}
	\centering
	\includegraphics[scale=0.3]{images/casp-difficuty.png}
	\caption{Confronto degli obiettivi degli ultimi quattro CASP in termini di copertura e identità di sequenza dei modelli disponibili. In entrambi i casi, CASP14 include gli obiettivi di modellazione libera (FM) più difficili mai forniti. Fonte\cite{blopigAF}}
	\label{fig:casp-difficulty}
	\endminipage\hfill
\end{figure}

In alcuni casi le predizioni di AlphaFold erano talmente accurate da superare i risultati sperimentali facendo mettere in discussione agli sperimentatori i risultati da loro ottenuti. Si potrebbe pensare che i target del CASP14 fossero in qualche misura più semplici rispetto a quelli degli altri anni per spiegare il successo di AlphaFold. Ma non è così, anzi, gli organizzatori hanno dimostrato che è stato il CASP più difficile (in quanto a percentuale di identità di sequenze, vedi fig. \ref{fig:casp-difficulty}).

Un esempio in cui AF2 surclassa gli altri metodi è il target T1064. AF2 riesce ad ottenere una similitudine molto alta, con nucleo e strutture secondarie quasi perfette (nonostante una grande regione di loop sia sbagliata, ma questo potrebbe anche indicare che sia una regione flessibile).

\begin{figure}[!htb]
	\minipage{0.48\textwidth}
	\centering
	\includegraphics[scale=0.3]{images/t1064-af2.png}
	\caption{Analisi GDT dei 508 modelli inviati per la sequenza target T1064-D1. L'analisi denota il più grande insieme di atomi di $C_{\alpha}$ (percentuale della struttura modellata) che può rientrare nella distanza cutoff $ \in \{ 0.5\, \angstrom, 1.0\, \angstrom, 1.5\, \angstrom, ... , 10.0\, \angstrom \} $. In viola i modelli di AlphaFold. Fonte: \cite{CaspRes}}
	\label{fig:t1064-chart}
	\endminipage\hfill
	\minipage{0.5\textwidth}
	\centering
	\includegraphics[scale=0.3]{images/t1064_model.png}
	\caption{(rosso) modello di AF2 per il target T1064. (blu) struttura 7JTL\_A. Fonte \cite{blopigAF}}
	\label{fig:t1064-afmodel}
	\endminipage\hfill
\end{figure}

Gli altri metodi predicono questa struttura in modo nettamente peggiore. Prendendo in considerazione i risultati dei gruppi Baker e Zhang (i due migliori subito dopo AF2, basati prevalentemente sulla pipeline del primo AlphaFold)  si può notare che il nucleo della proteina è totalmente sbagliato e ci sono molte differenze con la struttura sperimentale (vedi fig. \ref{fig:altri-modelli-t1064}).

\begin{figure}[!htb]
	\centering
	\includegraphics[scale=0.7]{images/t1064-altri-modelli.png}
	\caption{Modelli con il punteggio più alto per il target T1064 presentati dai gruppi Zhang (nero) e Baker (verde). In basso: modelli allineati con la struttura cristallina. A destra: tutti e tre i modelli (Zhang, Baker e AlphaFold 2) sono allineati con la struttura cristallina. Fonte\cite{blopigAF}}
	\label{fig:altri-modelli-t1064}
\end{figure}

Come è possibile vedere nel grafico in figura \ref{fig:modelli-casp14}, AF2 vince quasi in tutti i target, ci sono addirittura casi in cui il prossimo miglior metodo raggiunge solo il 20\% dell'accuratezza mentre AF è al 90\%.

\begin{figure}[!htb]
	\centering
	\includegraphics[scale=0.4]{images/models1.png}
	\caption{Il grafico mostra la differenza fra AF2 e il prossimo miglior metodo in tutti i target del CASP14. Fonte\cite{moAlq}}
	\label{fig:modelli-casp14}
\end{figure}

In generale, anche quando gli altri modelli predicono bene, è nei dettagli che AF2 si differenzia e porta la predizione ad un livello superiore (vedi fig. \ref{fig:af2-details}).

\begin{figure}[!htb]
	\centering
	\includegraphics[scale=0.5]{images/af2-details.png}
	\caption{Predizioni di AF2 (in blu) sovrapposte alle strutture sperimentali (in verde) sui target: T1049, T1056, T1044. In (c) è possibile notare una corretta predizione di un sito di legame per lo zinco. La struttura in (d) è composta da 2180 residui. Fonte\cite{jumper2021highly}}
	\label{fig:af2-details}
\end{figure}

\par AlphaFold2 è in grado di fornire delle precise stime della sua affidabilità per residuo in modo da consentire un uso consapevole delle sue predizioni. La sua misura di confidenza (pLDDT) predice affidabilmente l'accuratezza lDDT-C$_{\alpha}$ della predizione corrispondente. AF2 si è dimostrato applicabile anche su proteine molto lunghe. 

\par La rivoluzione di AlphaFold è stata assimilata alla rivoluzione avvenuta nell'ImageNet nel 2012. Ma secondo AlQuraishi le due cose non sono paragonabili. In quell'occasione il deep learning ha dimostrato di poter superare gli approcci convenzionali nel riconoscimento delle immagini sconvolgendo il mondo della computer vision. Rispetto all'avanzamento di AF2 vi è però una differenza importante: l'avanzamento nell'ImageNet è stato incrementale, quello di AF2 è invece un balzo in avanti di 10 anni, un cambiamento così profondo da mettere sottosopra un intero campo nel corso di una notte; è stato come avere l'accuratezza nell'ImageNet del 2020 già nel 2012, senza tutti i passi intermedi.

\subsubsection{Utilizzo di AF2}

Il codice sorgente di AlphaFold è stato reso pubblico da DeepMind ed è disponibile su GitHub\footnote{https://github.com/deepmind/alphafold}. Per funzionare AlphaFold ha bisogno di database di supporto (fino a 2.5 TB), di molta memoria e di potenza computazionale. Per questi motivi è verosimile utilizzarlo solo su server dedicati alla computazione, come quello disponibile all'Università di Pisa.

\par Il codice è rilasciato con un'immagine Docker e un \textit{launcher script} associato, in modo da risultare più facilmente accessibile. 

\par È stata anche pubblicata una versione semplificata di AlphaFold (senza uso di template) tramite un Google Colab notebook\footnote{https://colab.research.google.com/github/deepmind/alphafold/blob/main/notebooks/AlphaFold.ipynb}.


\section{Architettura}

Forse l'osservazione più importante da fare su AlphaFold è che DeepMind \textit{non} ha scoperto nessun nuovo e sbalorditivo principio sul protein folding. Non ci sono sorprese di carattere biologico. Tutto si basa su un ottimo design del sistema di DL e sul livello altissimo delle abilità dei membri del team e delle risorse a disposizione. È possibile porsi domande sul perché la struttura sia proprio come è presentata nel paper. La risposta più probabile risiede nell'intensa sperimentazione, attuata grazie ad una grande quantità di risorse computazionali e guidata da una grande capacità di progettazione del team di DeepMind.

\par Uno dei principi su cui si basa AlphaFold è l'immersione delle intuizioni basate sulla conoscenza fisico-chimica delle proteine direttamente nella struttura della rete, non come un processo intorno ad essa. Il bias induttivo del sistema riflette la conoscenza attuale fisica e geometrica delle proteine, sminuendo l'importanza della posizione dei residui nella sequenza ed enfatizzando invece la comunicazione tra i residui vicini nella proteina ripiegata. La rete apprende iterativamente un grafo delle interazioni fra residui, ragionando su questo grafo implicito mentre viene costruito. AlphaFold è un sistema end-to-end che produce direttamente una struttura invece di fornire come output le distanze inter-residuo.

\begin{figure}[!htb]
	\centering
	\includegraphics[scale=0.38]{images/af-archit.png}
	\caption{Schema architetturale di AF2. Le frecce indicano il flusso dell'informazione. Le dimensioni degli array sono riportate fra parentesi (s=numero di sequenze, r=numero di residui, c=numero di canali). Fonte\cite{jumper2021highly}}
	\label{fig:architettura-af2}
\end{figure}

L'architettura principale di AlphaFold può essere suddivisa in 3 componenti principali. Innanzitutto vi è la parte di \textit{preprocessing} dell'input, dove AF2 utilizza la sequenza di amminoacidi in ingresso per interrogare diversi database di sequenze proteiche e costruisce un MSA e quindi una \textit{MSA representation}. AlphaFold2 cerca anche di identificare le proteine che possono avere una struttura simile all'input (template) e costruisce una rappresentazione iniziale dei contatti nella struttura, chiamata \textit{pair representation}. Questo è, in sostanza, un modello di quali amminoacidi è probabile siano in contatto tra loro. La prima parte della struttura di AF2 non aggiunge niente di rivoluzionario alla pipeline dei sistemi di predizione. Vengono utilizzati anche database di metagenomica\footnote{L'applicazione di moderne tecniche di genomica senza la necessità di isolare e coltivare in laboratorio specie singole, studiandole quindi direttamente nel loro ambiente naturale.} come MGnify.

\par Nella seconda parte del diagramma, AF2 prende l'MSA e i template e li passa attraverso un \textit{transformer} (lo si può, per ora, immaginare come un "oracolo" in grado di identificare rapidamente quali informazioni siano più informative). L'obiettivo di questa parte è perfezionare le rappresentazioni sia per l'MSA che per le interazioni di coppia, anche scambiando informazioni tra loro in modo iterativo. Un modello migliore dell'MSA migliorerà la caratterizzazione della geometria della rete, che contemporaneamente aiuterà a perfezionare il modello dell'MSA. Questo processo è organizzato in blocchi che vengono ripetuti in modo iterativo fino a un numero specificato di cicli (48 blocchi nel modello pubblicato).

\par Queste informazioni vengono portate all'ultima parte del diagramma: lo \textit{structure module}. Questo sofisticato componente della pipeline prende la \textit{MSA representation} e la \textit{pair representation} e le sfrutta per costruire un modello tridimensionale della struttura. Il risultato finale è un lungo elenco di coordinate cartesiane che rappresentano la posizione di ciascun atomo della proteina, comprese le catene laterali.

\par L'ultima cosa da notare per farsi un'idea della struttura di AF2 è che funziona iterativamente. Una volta generata la prima struttura finale questa sarà utilizzata per raffinare ulteriormente la predizione, fino a un totale di 4 cicli di predizione.

\begin{figure}[!htb]
	\centering
	\includegraphics[scale=0.4]{images/af2-iterazioni.png}
	\caption{Traiettoria del valore del GDT sui domini di due target del CASP14 (T1024, composta da due domini e T1064) con 4 iterazioni del modello. Da notare che 48 blocchi dell'Evoformer costituiscono un ciclo di iterazione. I due domini T1024 ottengono la struttura corretta presto, mentre il target T1064 richiede praticamente tutta la profondità della rete per raggiungere una buono struttura finale. Fonte\cite{jumper2021highly}}
	\label{fig:af2-iterazioni}
\end{figure}

\subsection{Evoformer}
L'Evoformer è il primo componente della struttura di AF2 a cambiare le "regole" dei sistemi di predizione classici. Il compito dell'Evoformer è di spremere ogni goccia di informazione dall'MSA, dai template e dalle altre informazioni derivanti dall'analisi di sequenze. Sono decenni che vengono estratte informazioni attraverso analisi coevolutive, ma fino al CASP13 erano perlopiù approcci statistici. Molti gruppi hanno però dimostrato che attraverso l'uso di ResNet profonde non c'era bisogno di una robusta e complicata statistica. AlphaFold2 reinventa completamente questo processo di analisi coevolutiva e la porta ad un livello di utilità molto superiore.

\begin{figure}[!htb]
	\centering
	\includegraphics[scale=0.42]{images/evoformer2.png}
	\caption{Rete di AF interrogata sui distogrammi previsti. La previsione dei distogrammi è uno dei principali passi che AF compie per "comprendere" la struttura della proteina. Fonte\cite{AFslide}}
	\label{fig:evoformer-distogram}
\end{figure}

\par L'idea centrale dietro l'Evoformer è che le informazioni fluiscono avanti e indietro attraverso la rete. Prima di AlphaFold 2, la maggior parte dei modelli di deep learning richiedeva un allineamento di sequenze multiple e generava alcune inferenze sulla prossimità geometrica. L'informazione geometrica era quindi un prodotto della rete. Nell'Evoformer, invece, la \textit{pair representation} è sia un prodotto che uno strato intermedio. Ad ogni ciclo, il modello sfrutta l'attuale ipotesi strutturale per migliorare la valutazione dell'allineamento di sequenze multiple, che a sua volta porta a una nuova ipotesi strutturale, e così via. Entrambe le rappresentazioni, sequenza e struttura, si scambiano informazioni finché la rete non raggiunge una solida inferenza.

Il primo passo nella rete è definire gli \textit{embeddings} (incorporamenti) per l'MSA e i template. Gli allineamenti di sequenze multiple sono in ultima istanza sequenze di simboli su un alfabeto finito e quindi un esempio di variabile discreta. Le reti neurali, invece, sono intrinsecamente continue e si basano sulla differenziazione per apprendere dal loro training set. 

\par Un \textit{embedding} è un "trucco" del deep learning che consente la trasformazione di una variabile discreta in uno spazio continuo (\textit{embedded space}) in modo che la rete possa essere addestrata. È un processo molto semplice: c'è solo bisogno di definire uno strato di neuroni che riceve l'input discreto ed emette un vettore continuo. Un \textit{embedding}, più precisamente, è uno spazio di dimensioni relativamente basse in cui è possibile tradurre vettori di dimensioni elevate.  Idealmente, un \textit{embedding} acquisisce parte della semantica dell'input posizionando input semanticamente simili vicini nello spazio di incorporamento.  Un incorporamento può essere appreso e riutilizzato tra i modelli. L'embedding dettagliato che AF2 compie sulle caratteristiche di input può essere osservato in figura \ref{fig:af2-emebddings} \footnote{I dettagli applicativi di ogni variabile e algoritmo sono consultabili nelle informazioni supplementari del paper di AlphaFold: \fullcite{supplementaryjumper2021highly}.}.

\begin{figure}[!htb]
	\centering
	\includegraphics[scale=0.46]{images/af2-input-embeddings.png}
	\caption{Input feature embeddings. Fonte\cite{supplementaryjumper2021highly}}
	\label{fig:af2-emebddings}
\end{figure}

L'architettura dell'Evoformer usa due \textit{transformer}, connessi da un canale di comunicazione tra i due. Ognuno è specializzato in un certo tipo di dati: MSA o interazioni fra coppie di amminoacidi. 

\par L'architettura \textit{transformer} è stata introdotta nel 2017 da un gruppo del Google Brain\supercite{vaswani2017attention} e l'ingrediente chiave 
di tale architettura è chiamata \textit{attenzione}. L'obiettivo dell'\textit{attenzione} è identificare quali parti dell'input sono più importanti per l'obiettivo della rete neurale. I transformer hanno dimostrato empiricamente prestazioni superiori in una varietà di compiti, ad esempio nell'\textit{image captioning} e nel \textit{machine translation}. In quest'ultimo problema aiutano a migliorare il problema del \textit{vanishing gradient}, un ostacolo comune durante l'allenamento. Nei modelli basati su sequenze, possono accelerare significativamente l'addestramento rispetto ai classici modelli di RNN. In particolare, sono alla base della maggior parte dei risultati più clamorosi dell'IA degli ultimi anni: ad esempio, GPT in \textit{GPT-3} sta per "Generative Pre-training Transformer".

\par C'è anche un contro all'uso dei transformer: la costruzione della matrice di attenzione richiede un costo in memoria quadratico. Per questa ragione le Tensor Processing Unit (TPU) introdotte da Google apportano notevoli vantaggi.

\begin{figure}[!htb]
	\centering
	\includegraphics[scale=0.36]{images/evoformer.png}
	\caption{Blocco dell'Evoformer, le frecce indicano il flusso dell'informazione. Fonte\cite{jumper2021highly}}
	\label{fig:evoformer}
\end{figure}

\par Il primo transformer, denominabile anche \textit{MSA transformer}, calcola l'attenzione su una matrice molto ampia. Per ridurre quello che altrimenti sarebbe un problema computazionale intrattabile, l'attenzione "fattorizza" in componenti \textit{riga} e \textit{colonna}. Questo processo prende il nome di \textit{axial-attention}. La struttura a livelli consente di calcolare la maggioranza del contesto in parallelo durante la decodifica senza introdurre alcuna ipotesi di indipendenza. In questo contesto, semplificando, la rete calcola prima l'attenzione in orizzontale, consentendo alla rete di identificare quali coppie di amminoacidi sono più correlate; e poi in direzione verticale, determinando quali sequenze sono più informative. 

\par La caratteristica più importane dell'MSA transformer di AF2 è che il meccanismo di attenzione \textit{row-wise} incorpora informazioni dalla \textit{pair representation} come si può vedere in figura \ref{fig:evoformer}, in modo da focalizzarsi sulle coppie di residui che interagiscono fra loro.

L'altro transformer, denominabile \textit{pair transformer}, si basa su un'impostazione fondamentale: l'attenzione è disposta in termini di triangoli di residui (vedi fig. \ref{fig:triangoli-residui}), con l'obiettivo di sfruttare la disuguaglianza triangolare (fig. \ref{fig:dis-triang}).

\begin{figure}[!htb]
	\centering
	\includegraphics[scale=0.15]{images/disuguaglianza triangolare.PNG}
	\caption{Disuguaglianza triangolare. Fonte\cite{disTriang}}
	\label{fig:dis-triang}
\end{figure}

Quest'idea permette di superare un problema classico degli approcci basati su DL per il PSP: le distribuzioni di distanze non potevano essere \textit{embedded} nello spazio tridimensionale. 

\begin{figure}[!htb]
	\centering
	\includegraphics[scale=0.42]{images/triangoli.png}
	\caption{(b) Pair representation interpretata a grafo. (c) Aggiornamenti moltiplicativi ai triangoli di residui e self-attention. I dati nella pair representation sono illustrati come archi direzionati e in ogni diagramma l'arco aggiornata è "ij". Fonte\cite{jumper2021highly}}
	\label{fig:triangoli-residui}
\end{figure}

Dopo un certo numero di iterazioni, 48 nel paper, la rete ha costruito un modello delle interazioni all'interno della proteina. 


\subsection{Structure Module}

In questo modulo viene rappresentata tridimensionalmente la struttura attraverso un ripiegamento \textit{end-to-end} (invece di un metodo a discesa di gradiente). La proteina viene considerata come un \textit{gas residuo}, la backbone è un gas di corpi rigidi 3D. Gli amminoacidi vengono modellati come triangoli, rappresentando i 3 atomi della backbone. 

\par All'inizio i residui vengono tutti piazzati nell'origine (\textit{black hole inizialization}). Ad ogni step del processo iterativo vengono prodotte delle matrici di \textit{affinità} (via matematica per rappresentare traslazioni e rotazioni in una matrice 4×4).

\par Vengono effettuate 8 iterazioni per ridurre le violazioni stereochimiche e raffinare la struttura. Tuttavia anche dopo il processo di raffinamento è possibile che la struttura presenti delle violazioni. Per questa ragione vengono eseguiti dei raffinamenti ulteriori attraverso una discesa di gradiente vincolata dalle coordinate precedentemente calcolate, usando il campo di forza Amber ff99SB con OpenMM.

\begin{figure}[!htb]
	\minipage{0.5\textwidth}
	\centering
	\includegraphics[scale=0.4]{images/gas-residue.png}
	\caption{Rappresentazione come gas residuo. Fonte\cite{AFslide}}
	\label{fig:gas-residuo}
	\endminipage\hfill
	\minipage{0.48\textwidth}
	\centering
	\includegraphics[scale=0.4]{images/struct-module.png}
	\caption{Miglioramento dell'accuratezza e diminuzione delle violazioni stereochimiche attraverso le iterazioni del modulo. Fonte \cite{AFslide}}
	\label{fig:structure-module-iterazioni}
	\endminipage\hfill
\end{figure}

\par I corpi rigidi vengono aggiornati da un'architettura transformer equivariante 3D, che costruisce anche i gruppi laterali (parametrizzati da una lista di angoli  di torsione). Questa è una nuova architettura basata sull'attenzione ideata specificatamente per lavorare con strutture tridimensionali: \textit{Invariant Point Attention} (IPA). Questo meccanismo di attenzione beneficia del fatto di essere invariante rispetto alle traslazioni e alle rotazioni, necessitando così di meno dati.

\begin{figure}[!htb]
	\centering
	\includegraphics[scale=0.4]{images/structure-module-ipa.png}
	\caption{Structure module compreso IPA. Fonte\cite{jumper2021highly}}
	\label{fig:struct-ipa}
\end{figure}

\par Come rappresentazione iniziale viene usata la prima riga dell'Evoformer (la "single representation" è una copia della prima riga dell'MSA representation, ovvero una sequenza) ed è chiamata $s_{i}$. La \textit{pair representation} influenza le matrici di affinità nelle operazioni di attenzione. Le iterazioni sono 8 perché vi sono 8 strati nel modulo con pesi condivisi. Ogni strato aggiorna la rappresentazione singola astratta ($\{s_{i}\}$) così come la rappresentazione 3D (gas residuo).

\par Uno strato dello structure module è composto da 3 principali operazioni, nelle quali la singola rappresentazione astratta: 

\begin{enumerate}
	\item viene aggiornata dall'Invariant Point Attention (algoritmo 20 riga 6) 
	\item viene aggiornata da un layer di transizione
	\item infine viene mappata su frame di aggiornamenti concreti che sono parti dei frame della backbone
\end{enumerate}

\begin{figure}[!htb]
	\centering
	\includegraphics[scale=0.5]{images/alg20.png}
	\caption{Structure Module, prime 10 righe dello pseudocodice. Le righe qui fatte notare sono la 6 a la 10. Fonte\cite{supplementaryjumper2021highly}}
	\label{fig:alg-20}
\end{figure}

L'aggiornamento dei frame della backbone (algoritmo 20 riga 10) avviene tramite la predizione di un \textit{quaternione} per la rotazione e di un vettore per la traslazione. L'utilizzo dei quaternioni è utile perché rispetto alle usuali matrici di rotazione 3×3 sono rappresentabili con solo 4 numeri (invece di 9) e l'interpolazione di vari angoli è un processo perfettamente "liscio" (cioé non presentano casi speciali, come l'equivalenza degli angoli di 0° e 360°). Questo consente alla parametrizzazione attraverso quaternioni di risultare più semplice da ottimizzare con la discesa del gradiente.

\subsection{Altri dettagli}

La citazione ironica di AlQuraishi raccoglie forse al meglio le scelte strutturali in AlphaFold:

\say{\textit{For AlphaFold2, the apparent answer that DeepMind gave to the question of what they should do
		is... yes. Self-supervision? Yes. Self-distillation? Yes. New loss function? Yes. 3D refinement? Yes.
		Recycling after refinement? Yes. Refinement after recycling? Yes. Templates? Yes. Full MSAs? Yes.
		Tied-weights? Yes. Non-tied weights? Yes. Attention over nodes? Yes. Attention over edges? Yes.
		Attention over coordinates? Yes. The answer, to all the questions, is yes! And this clearly paid off.}}\footnote{\fullcite{moalqAF2}}\\

\subsubsection{Loss}

La rete è allenata end-to-end con gradienti provenienti dalla funzione di \textit{loss} (funzione obiettivo/di costo) FAPE (Frame Aligned Point Error) e da altre loss ausiliarie:

\begin{figure}[!htb]
	\centering
	\includegraphics[scale=0.4]{images/loss.png}
	\label{fig:loss}
\end{figure}

dove per \textit{aux} si intende "auxiliary loss", per \textit{dist} "cross-entropy loss for distogram prediction", per \textit{msa} "cross-entropy loss for masked MSA prediction", per \textit{conf} "model confidence loss". La loss totale nella fasi di inferenza comprende anche le loss \textit{exp resolved} che sta per "experimentally resolved loss" e \textit{viol} che sta per "violation loss".\\

\par La loss finale in AF2 è quindi una somma pesata di loss ausiliarie multiple, che non sono necessariamente correlate con le performance ma possono fornire informazioni aggiuntive. Non viene calcolata la loss solo dell'ultima struttura finale dopo le iterazioni, ma viene calcolata per ogni iterazione. È presente anche una \textit{distogram loss} dove la struttura predetta è usata per generare un distogramma (matrice 2D di intervalli di probabilità di distanze) da confrontare con la "realtà di base" (\textit{ground truth}, ovvero dalle strutture PDB con più accuratezza).

\par Un'altra loss interessante è l'\textit{MSA masking}. Ad ogni step al modello viene fornita una MSA con alcuni simboli "mascherati" e gli viene chiesto di predirli. È un modo per crearsi da sé un apprendimento supervisionato, come reso popolare da BERT, ed è usato sia nella fase di training che di inferenza.\\

\par Un altro dettaglio è la \textit{self-distillation}. L'architettura di AF2 è in grado di allenarsi con grande accuratezza solamente tramite il \textit{supervised learning} sui dati provenienti dal PDB. È stato però trovato il modo di aumentare l'accuratezza usando un approccio simile al \textit{noisy student self-distillation}\supercite{xie2020self}. In questa procedura viene usata una rete già addestrata per predire la struttura di circa 350 000 sequenze diverse da Uniclust30; viene poi creato un nuovo set di dati di strutture previste, filtrate in un sottoinsieme ad alta confidenza. Viene poi addestrata di nuovo la stessa architettura da zero utilizzando una combinazione di dati dal PDB e da questo nuovo set di dati di strutture previste come \textit{training data}. Questa procedura ha lo scopo di fare un uso efficace delle sequenze non etichettate e aumenta considerevolmente l'accuratezza della rete risultante (vedi fig. \ref{fig:ablazione1}). 

\subsubsection{Training}

AF2 viene addestrato in un modo apparentemente strano: non su intere proteine, ma su frammenti di alcune, ciò che il team di AF2 chiama \textit{crops}. Di solito i \textit{crops} sono composti da un paio di centinaia di residui, quindi solo una frazione di grandi proteine.

\par Sorprendentemente, mentre AF2 è principalmente addestrato su frammenti fino a 256 residui (successivamente perfezionato a 384), può prevedere strutture proteiche con ben oltre 2 000 residui. Sembrerebbe un task quasi impossibile in apparenza: come si è già visto il contesto globale è fondamentale nel protein folding, due sottosequenze di amminoacidi uguali, in due differenti proteine, non si ripiegano allo stesso modo in genere. Ci sono però due fattori che consentono ad AF2 di affrontare la sensibilità al contesto:

\begin{itemize}
	\item AF2 lavora con MSA o pattern coevolutivi, che codificano informazioni a prescindere dalla separazione nella catena
	\item durante la fase di inferenza AF2 usa l'intera sequenza
\end{itemize}

Quest'idea di disaccoppiare cose generalmente accoppiate stride con i modelli comuni nel ML dove le fasi di training e inferenza vengono tenute molto simili, basandosi sull'idea che più i due processi sono simili migliore sarà la predizione finale. In questo caso nella fase di allenamento è importante che il modello acquisisca informazioni con aggiornamenti dei gradienti. Nonostante AF2 non sia l'unica architettura ad aver adottato questa strategia (es. modelli generativi) essa è un’implementazione robusta dell’idea. È possibile che sia stata progettata in questo modo solamente per efficienza di memoria (sarebbe impossibile allenare su proteine intere una struttura della grandezza di AF2) ma tale scelta si è rivelata una buona idea anche dal punto di vista biofisico.

\subsection{Analisi via ablazione}

Per \textit{ablazione} si intende la valutazione delle prestazioni del sistema rimuovendo uno o più componenti non essenziali o combinazioni di questi. Questo studio è interessante perché può rivelare l'importanza e l'efficacia di determinati componenti. Non si possono trarre conclusioni generali ma si può ipotizzare quali siano le parti più importanti per generare predizioni di qualità.

\par Il modello \textit{baseline} è il modello come descritto nel paper ad eccezione del meccanismo di \textit{self-distillation}. Viene utilizzato come base per i confronti in questi studi di ablazione. 

\begin{figure}[!htb]
	\centering
	\includegraphics[scale=0.6]{images/ablazione.png}
	\caption{Risultati di ablazione di vari componenti su due target set: insieme di domini del CASP14 (n=87), PDB test set di catene con copertura di identità minore del 30\% (n=2.261). Fonte: \cite{jumper2021highly}}
	\label{fig:ablazione1}
\end{figure}

\begin{figure}[!htb]
	\centering
	\includegraphics[scale=0.7]{images/ablazione2.png}
	\caption{Accuratezza in esperimenti di ablazione relativi alla baseline per differenti valori di profondità dell'MSA su recenti insiemi di proteine dal PDB, filtrati da una copertura da template <30\%(n=2.261). Fonte \cite{supplementaryjumper2021highly}}
	\label{fig:ablazione-2}
\end{figure}

\par Ad esempio una caratteristica che può saltare all'occhio è l'apparente bassa influenza dell'IPA, nonostante sia una struttura complessa a cui il team  ha dedicato molto tempo. Senza IPA lo structure module si basa solo sulla rappresentazione 1D per la generazione della struttura. Il punto fondamentale è che le performance non cambiano molto se si toglie l'IPA, a patto che il recycling venga lasciato. Quando entrambi sono tolti si può vedere (fig. \ref{fig:ablazione1}) che le performance calino nettamente. Tuttavia se viene rimosso il recycling ma lasciata l'IPA le performance non subiscono grandi differenze, ed è importante notare ciò in quanto mostra che l'IPA è una struttura incredibilmente efficiente rispetto all'evoformer: con il recycling vengono triplicati i 48 blocchi dell'evoformer mentre l'IPA è composta di soli 8 layer.

\par In figura \ref{fig:ablazione-2} si può notare l'accuratezza di AF2 quando vengono tolti alcuni componenti, su sequenze con MSA poco profonde. Un dettaglio che risalta è l'importanza della funzione di loss relativa al mascheramento dell'MSA e di mantenere almeno uno fra IPA e recycling. Grazie a questi accorgimenti AF2 riesce ad avere ottima accuratezza anche quando l'MSA è poco profonda ed è forse proprio questo uno dei più grandi raggiungimenti di DeepMind.


\subsection{Differenze con AF1}

AlphaFold1 lavora sulla premessa che data una sequenza proteica, è possibile costruire un potenziale appreso e specifico per le proteine, addestrando una rete neurale profonda (DNN) per fare previsioni accurate sulla struttura e per prevedere la struttura stessa riducendo al minimo il potenziale mediante discesa del gradiente.

\par Le caratteristiche usate nella DNN sono caratteristiche MSA generate eseguendo HHblits e PSI-BLAST su database di sequenze. La DNN prevede gli angoli di torsione della backbone e la distanza a coppie tra i residui. Quindi, la distanza prevista e le distribuzioni di probabilità di torsione insieme alle interazioni di  van der Walls vengono combinate per formare un potenziale specifico della proteina. Infine, viene eseguita la discesa del gradiente sul potenziale specifico della proteina per ottenere il modello proteico finale. 

\par I dati di addestramento per il modello di AF1 vengono estratti dai domini PDB e più specificamente CATH in cui sono state utilizzate 29 427 proteine ​​per l'addestramento e 1 820 proteine ​​vengono utilizzate per i test. Le buone prestazioni di AlphaFold sono attribuite all'accuratezza delle previsioni di distanza\supercite{pakhrin2021deep}.

\par L'idea di ridurre al minimo il potenziale mediante la discesa del gradiente piuttosto che utilizzare l'assemblaggio dei frammenti e la successiva raffinazione del modello è piuttosto nuova.

\par La prima macro differenza tra i due sistemi è che AlphaFold 1 (AF1) conteneva moduli addestrati separatamente, mentre AlphaFold2 lo ha sostituito con un sistema di sottoreti accoppiate insieme in un sistema di deep learning end-to-end formato come un'unica struttura integrata.

\begin{figure}[!htb]
	\centering
	\includegraphics[scale=0.5]{images/af1.png}
	\caption{ Struttura di AF1. Length L = 155. (a) Step della predizione della struttura. (b) La rete neurale predice l'intero distogramma L×L basato su caratteristiche dell'MSA, accumulando predizioni separate per regioni di residui 64×64. Fonte\cite{senior2020improved}}
	\label{fig:af1}
\end{figure}

\par Rispetto alla prima iterazione di AlphaFold apparsa nel CASP13, guidata dalla previsione della \textit{distance map }basata su CNN, uno dei principali nuovi sviluppi di AlphaFold2 è l'architettura della rete neurale basata sull'\textit{attenzione} che interviene arbitrariamente sull'intera MSA. 

\par Inoltre, invece di utilizzare l'ottimizzazione della discesa del gradiente per costruire modelli basati sui vincoli di distanza previsti, come ha fatto AlphaFold in CASP13, AlphaFold2 utilizza un sistema di addestramento completo end-to-end dalla sequenza ai modelli di struttura, utilizzando il raffinamento strutturale iterativo basato sulla stima dell'errore locale. AF2 sostituisce le tradizionali simulazioni di ripiegamento con un modulo strutturale composto da reti neurali di transformer equivarianti 3D, che trattano ciascun amminoacido come un gas di corpi rigidi 3D e costruiscono direttamente la backbone proteica e le catene laterali.

\section{DeepMind}

{
	AlphaFold è un sistema di \textit{Artificial Intelligence }(AI) sviluppato da DeepMind che, come già visto, realizza predizioni allo stato dell'arte sulla struttura delle proteine basandosi sulle loro sequenze amminoacidiche.
	
	\par DeepMind è un'azienda inglese di Intelligenza Artificiale sussidiaria di Alphabet Inc.\footnote{In altre parole DeepMind è una società controllata: Alphabet Inc. detiene la maggioranza dei voti nell'assemblea ordinaria o un'influenza dominante sull'amministrazione.}. La missione a lungo termine di AlphaFold è avanzare il progresso scientifico risolvendo problemi scientifici fondamentali attraverso l'uso di sistemi di AI.
	
	\par DeepMind è stata fondata nel 2010 da Demis Hassabis, Shane Legg e Mustafa Suleyman. La società ha sede a Londra con centri di ricerca in Canada, Francia e Stati Uniti \supercite{deepMindWiki}.
	
	\par Può risultare interessante osservare la correlazione fra i primi lavori di DeepMind e la vita di Demis Hassabis, una vita ricca di sfaccettature: bambino prodigio nel gioco degli scacchi, programmatore di videogiochi (dai 17 anni) passando per una laurea in \textit{Computer Science}, alla fondazione del proprio studio videoludico (Elixir Studios) per poi ritornare nel mondo accademico per ottenere il suo PhD in neuroscienze cognitive nel 2009, campo nel quale ha coautorato numerosi articoli influenti su memoria e amnesia (es. rappresentazione della memoria episodica tramite \textit{scene construction} \supercite{Hassabis2007Jul}) \supercite{hassabisWiki}. Per arrivare infine a fondare DeepMind e la nuovissima società Isomorphic Labs, sempre sussidiaria di Alphabet Inc. che si pone come obiettivo quello di reimmaginare il processo di \textit{drug discovery} con un approccio basato principalmente sull'AI.
	
	\par DeepMind iniziò infatti a focalizzarsi sull'insegnare ad un sistema di AI come giocare a vecchi videogiochi anni '70, '80 (es. Pong, Breakout, Space Invaders), per poi passare al gioco del Go, al protein folding e recentemente alla programmazione competitiva automatizzata \supercite{competitiveProgrDeepMind}.
	DeepMind è stata acquistata da Google nel 2014 per 500 milioni di dollari \supercite{Guardian2014}.
	
	\subsubsection{Etica}
	Dopo l'acquisizione di Google l'azienda ha stabilito un'\textit{AI ethics board}. DeepMind è uno dei membri fondatori di \textit{Partnership on AI} insieme ad Amazon, Google, Facebook, IBM e Microsoft, un'organizzazione dedicata all'interfaccia società-AI \supercite{partnershiponai}.
	
	\par DeepMind ha anche aperto una nuova unità denominata DeepMind Ethics and Society e si è concentrata sulle questioni etiche e sociali sollevate dall'intelligenza artificiale avendo come consulente il famoso filosofo Nick Bostrom. Nell'ottobre 2017, DeepMind ha lanciato un nuovo gruppo di ricerca per studiare l'etica dell'IA.
	
	\subsubsection{Alphabet}
	Alphabet è un'azienda statunitense fondata nel 2015 dagli stessi fondatori di Google (Larry Page e Sergey Brin) come \textit{holding} a cui fa capo Google LLC e altre società sussidiarie: oltre a DeepMind vi sono Calico, CapitalG, Waymo, Wing, Intrinsic, Nest Labs, Sidewalk Labs, Isomorphic Labs, ecc.\\ 
	Da dicembre 2019 il CEO di Alphabet è Sundar Pichai \supercite{cnbc}.
	La fondazione di Alphabet a partire da Google è stata una scelta finalizzata a rendere più trasparenti le attività inerenti a Google e concedere una maggiore autonomia alle società del gruppo che operano in settori diversi da quello dei servizi internet.
}

\clearpage















\chapter{Uso di AlphaFold e visualizzazione}

\clearpage
\chapter{Scenari aperti e conclusioni}

One should remember, though, that effi-
cient protein structure prediction is just a means to an end; the real challenge of structural
biology has always been (and still is) to deduce the functions of proteins on the basis of their
sequences and/or structural data\supercite{kessel}


- drug design [pal 18.3 p.474]


ai-fueled paradigm shift
Perrakis et al., 2021- AI revolutions in biology-Alphafold

% bibliography, glossary and index would go here.
\printbibliography

\end{document}