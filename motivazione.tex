Trasformare l’esperienza dell’università in qualcosa di positivo, di progressivo, che può alimentare il fuoco delle mie passioni
Fai qualcosa di specifico, renditi esperto.\newline

Guida il lettore da 0 ad Alphafold facendolo meravigliare davanti alla bellezza della bioinformatica, e della vita.\newline

Medita e poi scrivi qui: non passare da fonti terze. Non perdere il flusso.\\
Tu stai scrivendo qualcosa per te, non per il mondo. Scrivi, poi confrontati. Se ti confronti è normale che ti vedi inferiore. Come puoi invece essere inferiore a te stesso? \newline

Ciò che conta è fare, fare, fare, mettere in pratica.\newline

Hai scelto tu di uscire dall’informatica. Hai paura di risultare ignorante in biologia? Hai paura di esserti immischiato in un campo a te esterno e di sembrare “capiscione”? \\
1. Non ne sa quasi nulla nessuno dei prof 2.Non interessa loro 3. ho Mario Pirchio a cui chiedere aiuto 4. voglio uscire dall’informatica pura. Non mi fido. Non mi interessa. Qui mi interesso 5.affronta la responsabilità. Ho la responsabilità di creare la mia strada e crederci, di laurearmi per mio padre e la mia famiglia.\newline

Mentre disegnavo ho notato che ciò che mi spingeva a a migliorare il disegno era riuscire a intravedere il risultato finale in quello che stavo facendo. Non stavo tracciando una linea su un foglio. Stavo facendo piccoli passi per mettere su carta ciò che vedevo dentro di me (non nella mente, ma nel cuore).\\ Realizzavo una piccola parte di me al di fuori di me. E vedere che ciò che stavo creando si stava avvicinando a ciò che avevo in mente mi dava una soddisfazione immensa. E questa felicità mi spingeva tantissimo a continuare e a migliorarmi.\\
Voglio scrivere questo documento per realizzare una piccola parte di me all’esterno di me.\\
L’obiettivo del disegno era realizzare un ritratto di Thich Nhat Hanh, per esprimere la mia gratitudine nei suoi confronti.\newline

Obiettivo finale: realizzare un documento riguardante il background della bioinformatica e lo studio di AlphaFold per esprimere la mia speranza che l’informatica possa essere usata per il bene della Vita, che ci possa avvicinare ad una comprensione maggiore di essa e di quanto ogni fenomeno sia interrelato.\newline

La tesi serve a dimostrare una ipotesi che avete elaborato dall’inizio, non a mostrare che voi sapete tutto

Ludo non dimenticarti quanta luce hai, sei ricco di una bellezza tanto speciale, non lasciare che altri te la nascondano\\
... \textit{ti ringrazio per ciò che sei}...\\
Tutto ciò che sei, che dici, che fai è meraviglioso\\
Dovresti sentirti in colpa con te stesso se invece abbandonassi tutto e tornassi a casa per paura di sbagliare\\
Ludo non hai bisogno di me, nè di tua madre nè di nessun altro. Tu sei una persona davvero meravigliosa, sei forte, hai tanta luce in te. Una cosa che penso di sapere è che potresti fare qualsiasi cosa, andare da qualsiasi parte. E se lo vorrai io ci sarò in ogni caso, non hai bisogno del mio appoggio per raggiungere quello che vuoi, ma io sono qui, e ci resto per tutto il tempo che vorrai.\\
E potessi starti vicina ogni notte e risvegliarmi accanto a te la mattina farei il tifo per te direttamente dalla prima fila ;")\\
ci credo davvero nel risultato positivo che potrai scoprire tra un po', non demordere prima o poi arriverà esattamente quella cosa che stavi aspettando e tutto andrà a posto da sè [..Sophie..]\\ \\

Una cosa per volta. Svuota il cervello. Adesso il mondo ti sembra pieno di problemi. Ne esiste solo uno per te: il tuo obiettivo. Se pensi a tutte le possibilità rimani fermo. Va solo in direzione dell'obiettivo. Poi al prossimo ci si penserà una volta raggiunto. Nessuno ti mette i bastoni fra le ruote, siamo con te. [papà] \\ \\


Non solo hai dato voce alla tua vita, ma sei stato in grado di renderla poesia, sono veramente orgoglioso di te e di come ti ho rivisto dopo tanto tempo perché, te lo dico con tutto il cuore, hai fatto dei passi da gigante, dei passi enormi e veramente complimenti. [..diego] \\ \\
Quando l'ansia bussa alla porta ringhio contro di lei: "ti affronto". Sono qui, avanti. Mi fermo e la guardo negli occhi. Affronto la vita, senza scappatoie. \\

Non abbiate paura di rischiare per non sbagliare. Mordete la vita. Sporcatevi le mani [Mattarella] \\

in Lui, le doti della mente e del cuore armonizzavano singolarmente.\\

L'amore all'indagine fu veramente la sola grande passione della Sua vita\\

Egli è rimasto vivo [..] ad insegnarci ancora come sulla vetta del sapere debba ardere il puro fuoco dell'entusiasmo. \\

tanto lo scienziato e l'uomo erano fusi in Lui; tanto la scienza era entrata ad invadergli la vita; tanto la sua umanità si era interamente votata alla
scienza. Al rigorismo scientifico disposava un caldo sentimento di artista, un'ispirazione profonda, intuitiva, che Lo guidava nel trovare i temi di ricerca, che Lo confortava nella dura fatica del suo lavoro quotidiano. \\

 era nemico nei suoi studi dell'arida arte meccanica. Se un giovane faceva ricorso a Lui desideroso di approfondire problemi o di battere la via della ricerca scientifica, Egli era pronto a divenirgli non solamente maestro, ma, più che maestro, padre. Si rallegrava con tanta spontanea compiacenza dei successi dei suoi allievi, come se questi fossero altrettanti suoi figliuoli: ed era pronto a
sorreggerli non solamente dinanzi agli ostacoli degli studi, ma ancora di fronte alle difficoltà della vita. Dal conforto che Egli recava al cuore dei suoi allievi ritraeva il suo medesimo conforto.\\

Confessava della sua scienza l'impareggiabile funzione morale: destare nei giovani l'amore alle cose naturali, rivelarne la bellezza, discoprirne 1' ordine e le leggi voleva dire per Lui potenziare ed accrescere in loro l'ordine morale interiore


[Lambertini in onore ad Angelo Ruffini, 1930]

\clearpage